\documentclass[11pt]{article}

\usepackage[francais]{babel} %
% \usepackage[T1]{fontenc} %
\usepackage[latin1]{inputenc} %
% \usepackage[applemac]{inputenc} %
% \usepackage{a4wide} %

% \setlength{\parskip}{0.3\baselineskip}

\begin{document}

\title{Le titre de votre rapport \\
Pas plus du recto-verso pour cette fiche}

\author{Votre nom, celui de votre encadrant, le nom de son
  �quipe/labo}

\date{La date}

\maketitle

\pagestyle{empty} %
\thispagestyle{empty}

%% Attention: pas plus d'un recto-verso!
% Ne conservez pas les questions


\subsection*{Le contexte g�n�ral}

De quoi s'agit-il ? 
D'o� vient-il ? 
Quels sont les travaux d�j� accomplis dans ce domaine dans le monde ?

\subsection*{Le probl�me �tudi�}

Quelle est la question que vous avez abord�e ? 
Pourquoi est-elle importante, � quoi cela sert-il d'y r�pondre ?  
Est-ce un nouveau probl�me ?
Si oui, pourquoi �tes-vous le premier chercheur de l'univers � l'avoir pos�e ?
Si non, pourquoi pensiez-vous pouvoir apporter une contribution originale ?

\subsection*{La contribution propos�e}

Qu'avez vous propos� comme solution � cette question ? 
Attention, pas de technique, seulement les grandes id�es ! 
Soignez particuli�rement la description de la d�marche \emph{scientifique}.

\subsection*{Les arguments en faveur de sa validit�}

Qu'est-ce qui montre que cette solution est une bonne solution ?
Des exp�riences, des corollaires ? 
Commentez la \emph{robustesse} de votre proposition : 
comment la validit� de la solution d�pend-elle des hypoth�ses de travail ?

\subsection*{Le bilan et les perspectives}

Et apr�s ? En quoi votre approche est-elle g�n�rale ? 
Qu'est-ce que votre contribution a apport� au domaine ? 
Que faudrait-il faire maintenant ? 
Quelle est la bonne \emph{prochaine} question ?

\end{document}



