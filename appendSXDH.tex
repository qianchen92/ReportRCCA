To compare with our construction, we also give in this section an instantiation of Chase \etal~\cite{DBLP:conf/eurocrypt/ChaseKLM12} based on the SXDH assumption.

\begin{description}
\item[\boldmath$RCCA3.\Setup(\lambda)$:] This algorithm generates the public and secret keys of our RCCA encryption scheme.
  \begin{enumerate}
  \item Pick bilinear groups $(\G, \hat{\G}, \G_T)$ of prime order $p$ and the bilinear map $e$ on this pair of groups. We choose a group generator $f \in \G$ and a random value $x \sample \mathbb{Z}_p$, then compute $g = f^x$.
  \item Choose a random group generator $d \in \G$.
  \item Choose a random group generator $\hat{d} \in \hat{\G}$.
  \item Choose $g_1,g_2 \sample \G$ and $\hat{g}_1, \hat{g}_2 \in \hat{\G}$ then set $\vec{u}_1 = (g_1, g_2) \in \G^2$, $\vec{u}_2 \sample \G^2$, $\hat{\vec{u}}_1 = (\hat{g}_1, \hat{g}_2) \in \hat{\G}^2$ and $\hat{\vec{u}}_2 \sample \hat{\G}^2 $
  \item Set up the keys for the underlying encryption scheme $\pk_{enc} = (f,g)$ and $\sk_{enc} = (x)$.
  \item Choose two random group generators $(g_z, g_r)$ and $6$ random exponents $\{\chi_i, \gamma_i\}_{i = 1}^2, \zeta, \rho\sample \Z_p$, then compute $g_i = g_z^{\chi_i}g_r^{\gamma_i})$ for $ i \in \{1,2\}$ and $\alpha = g_z^\zeta g_r^\rho$
  \item Set up the keys for the underlying signature scheme
    $$\vk_{sig} = (g_z, g_r, g_1, g_2, \alpha)$$
    and
    $$\sk_{sig} = (\vk_{sig}, \zeta, \rho, \{\chi_i, \gamma_i\}_{i = 1}^2).$$
  \item $\PK = (d,\pk_{enc}, \vk_{sig}, \sigma_{crs})$.
  \item $\SK = (x)$
  \end{enumerate}

\item[\boldmath$RCCA3.\Enc(\PK,m)$:] This algorithm takes as input a message and the public key of the underlying encryption scheme, outputs the corresponding ciphertext of our RCCA encryption scheme.
  \begin{enumerate}
  \item Choose a random exponents $\theta \sample \Z_p$ and set $r = \theta$.
  \item Compute the ciphertext $\vec{C} = (C_1, C_2)$:
    \begin{align*}
      C_1 &= f^{\theta} & C_2 &= m \cdot g^{\theta}
    \end{align*}

  \item Prove the knowledge of the witness $\vec{w} = (m, r, \vec{D}, S, \vec{\sigma})$ which verifies that
    \begin{align*}
      Enc_{BBS}(pk_{enc}, m; r) = \vec{C} \vee (\vec{C} = ReRand(\vec{D}; S) \wedge Verify(vk_{sig}, \vec{D}) = \True)
    \end{align*}
    
  \item Define the bit $b = 1$ and a commitment $\vec{C}_b = (1,d^b)\cdot \vec{g}_1^{r_b} \cdot \vec{g}_2^{s_b}$, $\hat{\vec{C}}_b = (1,\hat{d}^b) \cdot \hat{\vec{g}}_1^{r_{\hat{b}}} \cdot \hat{\vec{g}}_2^{s_{\hat{b}}}$ and also a $WI-NIZK$ proof $(\pi_{b,1}, \pi_{b,2}) \in (\G^2 \times \hat{\G}^2)^2$ of the pairing product equation 
  
  \begin{align}
    e(d, \boxed{\hat{d}^b}) &= e(\boxed{d^b}, \boxed{\hat{d}^b}) \tag{1}\\
    e(\boxed{d^b}, d) &= e(\boxed{d^b}, \boxed{\hat{d}^b}) \tag{2}
  \end{align}
  
  which ensures that $b \in \{0,1\}$.

    %  \item Then generate commitements $\{\vec{C}_{\Gamma_i}\}_{i = 1}^3$ of the variables $\Gamma_i = C_i^b$ for $i \in \{1,2,3\}$ and the corresponding proofs:
    %    \begin{align}
    %      e(C_i,\boxed{h^b}) &= e(h, \boxed{\Gamma_i}) ,&
    %      \forall i \in \{1,2,3\}
    %    \end{align}
  \item We first prove the left side of the OR statement, we generate commitments $(\vec{C}_{\hat{R}})$ of the variable $\hat{R} = \hat{d}^{\theta b}$ and $\vec{C}_{M}$ commitment of $M = m^b$ and commitments $\{\vec{C}_{\Delta_i}\}_{i=1}^2$ of the variables $\{\Delta_i = C_i^b\}_{i=1}^2$. Recall that $\{C_i\}_{i=1}^2$, $R$ and $\{\Delta_i\}_{i=1}^2$ verify the following equations:
    \begin{align}
      e(C_i, \boxed{\hat{d}^b}) &= e(\boxed{\Delta_i}, \hat{d})  &\forall i \in \{1,2\} \tag{3,4}\\
      e(\boxed{\Delta_1}, \hat{d}) &= e(f, \boxed{\hat{R}}) \tag{5}\\
      e(\boxed{\Delta_2}, \hat{d}) \cdot e(\boxed{M}, \hat{d}^{-1}) &= e(g, \boxed{\hat{R}}) \tag{6}
    \end{align}



    %signature part
  \item Then we prove the right side of the OR statement:
    \begin{enumerate}  
    \item The $ReRand$ component: define $(D_1, D_2) = (1_\G, 1_\G)$ and $S = 1_\G$.
    \item Remind that actually these variables are of the following forms in the security proof, but in the case $b=1$ they all become $1_\G$.
      $$(D_1, D_2) = (f^{(\theta_1+\theta_1')\cdot (1-b)}, m^{1-b} \cdot g^{(\theta_2+\theta_2')\cdot (1-b)}$$
      and $$S_1 = d^{\theta_1'\cdot (1-b)}$$
    \item Then compute the commitments $\{\vec{C}_{D_i}\}_{i=1}^2$ of $\{D_i\}_{i= 1}^2$ and $\vec{C}_{\hat{S}}$ commitments of $\hat{S}$.
    \item And also compute the proofs of following equations:
      \begin{align}
        e(C_1/\boxed{\Delta_1}, \hat{d}) &= e (\boxed{D_1}, \hat{d}) \cdot e(f^{-1}, \boxed{\hat{S}}) \tag{7}\\
        e(C_2/\boxed{\Delta_2}, \hat{d}) &= e (\boxed{D_2}, \hat{d}) \cdot e(g^{-1}, \boxed{\hat{S}}) \tag{8}
      \end{align}

    \item Then the signature component (Remind that $b = 1$): define
      $$\vec{\sigma}  = (\Sigma_1, \Sigma_2) = (z^{1-b}, r^{1-b}) = (1_\G, 1_\G),$$
      then compute their commitments $(\vec{C}_{\Sigma_1}, \vec{C}_{\Sigma_2})$.
    \item We generate the proof of the following linear pairing equations:
      \begin{align} 
        e(\alpha, d/\boxed{d^b}) &= e(g_z, \boxed{\Sigma_1}) \cdot e(g_r, \boxed{\Sigma_2}) \cdot \prod_{i=1}^3 e(g_i, \boxed{D_i}) \tag{9}
      \end{align}

    \end{enumerate}

  \item To allow the re-randomization of the ciphertext, we need to compute the commitments $\vec{C}_F$, $\vec{C}_G$ to the variables :

    \begin{align*}
    F &= f^b, & G&=g^b
    \end{align*}

    and their corresponding proofs:
    \begin{align}
      e(\boxed{F},d) &= e(f,\boxed{d^b}) & e(\boxed{G}, d) &= e(g, \boxed{d^b})\tag{10,11}
    \end{align}

    
  \item We put all these proofs together to get $\vec{\pi}$.
  \item The ciphertext of the RCCA-scheme is
    $$(\vec{C} = (C_1, C_2), \vec{C}_{d^b}, \vec{C}_{\hat{d}^b}, \vec{C}_{M}, \vec{C}_{\hat{R}}, \{\vec{C}_{D_i}\}_{i = 1}^2, \vec{C}_{\hat{S}}, \{\vec{C}_{\Sigma_i}\}_{i = 1}^2,\{\vec{C}_{\Delta_i}\}_{i=1}^2, \vec{C}_F, \vec{C}_G, \vec{\pi}) \in 49\G \times 20 \hat{\G}$$
    
  \end{enumerate}

\item[\boldmath{$RCCA3.\Dec(\PK,\SK, \vec{C})$}]:
  \begin{enumerate}
  \item Parse $\vec{C}$ as $(C_1, C_2), \vec{C}_{d^b}, \vec{C}_{\hat{d}^b}, \vec{C}_{M}, \vec{C}_{\hat{R}}, \{\vec{C}_{D_i}\}_{i = 1}^2, \vec{C}_{\hat{S}}, \{\vec{C}_{\Sigma_i}\}_{i = 1}^2,\{\vec{C}_{\Delta_i}\}_{i=1}^2, \vec{C}_F, \vec{C}_G, \vec{\pi})$.
  \item Parse $\SK$ as $x$
  \item Verify that all proofs are correct.
  \item If any proof fails then return $\bot$. otherwise return $C_2/(C_1^x)$ 
  \end{enumerate}

\item[\boldmath{$RCCA3.\Rerand(\PK, \vec{C}, )$}]:  
  For the randomization, we will proceed in two stages. Firstly we sample two random values $\theta' \gets \mathbb{Z}_p$. The new variables are $C_1' = C_1 \cdot f^{\theta}$ and $C_2' = C_2 \cdot h^{\theta'}$. Then using the proof of the equations $(10, 11)$, to adapt the new proofs corresponding to the new ciphertext instance $\vec{C}' = (C_1', C_2')$.

  %  For the randomization we first generate randomness for each commitment. For a variable $X$, we generate the new randomness $(\tilde{r}_X, \tilde{s}_X, \tilde{t}_X)$, the new commitment will be $\tilde{\vec{C}}_X = \iota(X) \cdot \vec{g}_1^{r_X+\tilde{r}_X} \cdot \vec{g}_2^{s_X+\tilde{s}_X} \cdot \vec{g}_3^{t_X+\tilde{t}_X}$.
  
For the second stage, we randomize all the commitments and the GS proofs without changing the ciphertext part $\vec{C}' = (C_1', C_2')$.

In this algorithm, for the variable $X$, we denote its commitment by $\vec{C}_X = (1, X) \cdot \vec{g}_1^{r_X} \cdot \vec{g}_2^{s_X}$ its new commitment from the first stage by $\vec{C}_X' = \vec{C}_X \cdot \vec{g}_1^{r'_X} \cdot \vec{g}_2^{s'_X}$ and denote the new randomness introduced in the second step by $(\tilde{r}_X, \tilde{s}_X)$.



\end{description}


%\input{proofs}

