
\begin{proof}
  The proof proceeds with a sequence of games that begins with the real game and ends with a game where no advantage is left to the adversary. In each game, we call $W_i$ the event that the experiment outputs $1$. 


  \begin{description}
  \item[\textsf{Game} $0$:] This is the real game. The adversary is given the public key $\PK$ which contains 
    vectors 
    $(\vec{u}_1, \vec{u}_2)$ such that
    \begin{eqnarray} \label{vec-PK} 
      \vec{u}_1 &=& (\hat{g}, \hat{h}) \in \hat{\G}^2  \\ \nonumber 
      \vec{u}_2 &=&  (\hat{g}^{\rho_u}, \hat{h}^{\rho_u'}) \in \hat{\G}^2,
    \end{eqnarray}
    where $\hat{g},\hat{h} \sample \hat{\G}$,  $\rho_u,\rho_u' \sample \Z_p$.
    In the challenge phase, 
    it chooses two messages $M_0,M_1 \in \G$ and obtains a challenge ciphertexts 
    $$  \vec{C}^\star = (\SVK^\star, \com^\star, \open^\star, C_0^\star, C_1^\star, C_2^\star, \vec{C}_{\theta}^\star, \vec{\pi}^\star, \vec{\sigma}^\star)  $$
    where 
    \begin{align*}
      C_0^\star &= M_{\beta} \cdot X^{\theta^\star}, & C_1^\star &= g_1^{\theta^\star}, & C_2^\star &= g_2^{\theta^\star},
    \end{align*}
    for some random bit $\beta \sample \{0,1\}$, and 
    \begin{eqnarray*}
      \com^\star  &=& \hat{C}^\star  = \hat{g}^{\zeta_2^\star} \cdot \prod_{i = 1}^{\ell }\hat{X}_i^{\chi_i^\star } \cdot \hat{X}_{\ell+1}^{w_z^\star} \cdot \hat{X}_{\ell+2}^{a^\star}  \\
      \open^\star &=& \big( D^\star , g_z^\star, g_1^\star, \dots, g_\ell^\star, A^\star  , \hat{Z}^\star, \hat{R}^\star \big)  \\ 
      & =& \big(  g^{\zeta_2^\star },~ g^{w_z^\star}, ~ g^{\chi_1^\star}, ~\ldots , ~ g^{\chi_\ell^\star} , ~ g^{\chi_z^\star} , ~ g^{a^\star}, ~ 		 
      \hat{g}^{\zeta_1^\star} ,  ~\hat{g}^{a^\star-\zeta_1^\star w_z^\star }  \cdot  \prod_{i=1}^{6} {\hat{M}_i^\star~ }^{\chi_i^\star} 
      \big), \\
      \vec{C}_{\theta}^\star &=& \vec{u}_{\com^\star}^{\theta^\star} \cdot (\vec{u}_1)^{r^\star}\\
      \vec{\pi}^\star &=& (\pi_1^\star,\pi_2^\star) =(g_1^{r^\star},g_2^{r^\star}) 
    \end{eqnarray*}
    with $(\hat{M}_1^\star,\ldots,\hat{M}_\ell^\star)=(\hat{g}_1^\star,\ldots,\hat{g}_5^\star,\hat{A}^\star) $ and 
    $\vec{u}_{\com^\star} = \vec{u}_2\cdot (1, \com^\star)$.
    \indent The adversary's decryption queries are always faithfully answered by the challenger. When the adversary halts, it outputs   
    $\beta' \in \{0,1\}$ and wins if $\beta' =\beta$. In this case, the experiment outputs $1$. Otherwise, it outputs $0$.  
    The adversary's advantage is thus $|\Pr[W_0]-1/2|$. \smallskip \smallskip 

  \item[\textsf{Game} $1$:] In this game, we modify the generation of the public key and define
    \begin{eqnarray} \label{vec-PK-sim} 
      \vec{u}_1 &=& (\hat{g}, \hat{h}) \in \hat{\G}^2  \\ \nonumber 
      \vec{u}_2 &=&  (\hat{g}^{\rho_u}, \hat{h}^{\rho_u'}) \cdot (1,{\hat{C}^\star ~}^{-1})  \in \hat{\G}^2.
    \end{eqnarray}
    instead of computing $(\vec{u}_1,\vec{u}_2)$ as in (\ref{vec-PK}) (note that we may assume w.l.o.g. that $\SVK^\star$ and $\com^\star=\hat{C}^\star$ are 
    generated 
    at the outset of the game).  However, this modification does not affect the adversary's view since $\vec{u}_2$ remains uniformly distributed 
    over $\hat{\G}^2$. We have $\Pr[W_1]=\Pr[W_0]$.   \smallskip \smallskip 

  \item[\textsf{Game} $2$:] This game is like Game $1$ with the difference that, if the adversary makes a pre-challenge decryption query 
    $ \vec{C} = (\SVK, \com, \open, C_0, C_1, C_2, \vec{C}_{\theta}, \vec{\pi}, \vec{\sigma})  $ such that $\com=\com^\star$, the expermiment halts and 
    outputs a random bit. Since Game $2$ is identical to Game $1$ until this event $F_2$ occurs, we have 
    $|\Pr[W_2]-\Pr[W_1]| \leq \Pr[F_2]$. Moreover, since $\com^\star$ was chosen uniformly in $\hat{\G}$ and remains independent of $\A$'s view until 
    the challenge phase, we have $ |\Pr[W_2]-\Pr[W_1]| \leq \Pr[F_2] \leq q_D/p$. \smallskip \smallskip 


  \item[\textsf{Game} $3$:] This game is like Game $2$ but we modify the decryption oracle. Namely, if the adversary makes a post-challenge decryption query 
    for a valid ciphertext
    $$ \vec{C} = (\SVK, \com, \open, C_0, C_1, C_2, \vec{C}_{\theta}, \vec{\pi}, \vec{\sigma})  $$
    such that $\com=\com^\star$ but $\open \neq \open^\star$, the experiment halts and outputs a random bit. If we call $F_3$ the latter event, 
    we have $|\Pr[W_3]-\Pr[W_2]| \leq \Pr[F_3]$. Moreover, event $F_3$ clearly implies an adversary $\B$ against the target collision-resistance of the 
    structure-preserving trapdoor commitment in Section \ref{trap-com}, which contradicts the Double Pairing assumption. Hence, we have  
    $|\Pr[W_3]-\Pr[W_2]| \leq \mathbf{Adv}_\B^{\mathsf{TCR}\textsf{-}\mathsf{CR}}(\lambda) \leq   \mathbf{Adv}_\B^{\mathrm{DP}}(\lambda)$.   \smallskip \smallskip 


  \item[\textsf{Game} $4$:] We modify again the decryption oracle. After the phase, if the adversary queries the decryption of a ciphertext 
    $ \vec{C} = (\SVK, \com, \open, C_0, C_1, C_2, \vec{C}_{\theta}, \vec{\pi}, \vec{\sigma})  $ such that $(\com,\open)=(\com^\star,\open^\star)$
    but $(C_0,C_1,C_2,\pi_1,\pi_2) \neq (C_0^\star,C_1^\star,C_2^\star,\pi_1^\star,\pi_2^\star)$, the experiment halts and outputs a random 
    bit. If we call $F_4$ this event, we have the inequality  $|\Pr[W_4]-\Pr[W_3]| \leq \Pr[F_4]$ since Game $4$ is identical to Game $3$ until $F_4$ occurs.
    Moreover,   $F_4$ would contradict the strong unforgeability of the one-time structure-preserving signature  and thus the DP assumption. This implies  
    $|\Pr[W_4]-\Pr[W_3]| \leq \mathbf{Adv}_\B^{\mathsf{SUF}\textsf{-}\mathsf{OTS}}(\lambda)   \leq   \mathbf{Adv}_\B^{\mathrm{DP}}(\lambda)$.
    \smallskip \smallskip 

  \item[\textsf{Game} $5$:] We introduce yet another modification in the decryption oracle.  We let the decryption oracle reject all ciphertexts $ \vec{C} = (\SVK, \com, \open, C_0, C_1, C_2, \vec{C}_{\theta}, \vec{\pi}, \vec{\sigma})  $ 
    such that
    \begin{eqnarray} \label{event-F5}
      (\com,\open)=(\com^\star,\open^\star)  \quad \wedge \quad (C_0,C_1,C_2,\pi_1,\pi_2) = (C_0^\star,C_1^\star,C_2^\star,\pi_1^\star,\pi_2^\star) 
      \quad \wedge \quad \vec{C}_{\theta} \neq \vec{C}_{\theta}^\star .
    \end{eqnarray}  
    Let $F_5$ be the event that the decryption oracle rejects a ciphertext that would not have been rejected in Game $4$.   
    We argue that $\Pr[W_5] = \Pr[W_4]$ since Game $5$ is identical to Game $4$ until event $F_5$ occurs and we have $\Pr[F_5]=0$. 
    Indeed, 
    %since the vectors $\vec{u}_{\com^\star}$ and $\vec{u}_1$ are linearly independent, they span the entire vector space $\hat{\G}^2$ which 
    %means that, 
    for a given $(C_1^\star,C_2^\star,\pi_1^\star,\pi_2^\star) \in \G^4$, there exists only one commitment $\vec{C}_{\theta}^\star \in \hat{\G}^2$ that satisfies the equalities (\ref{ver-eq}). This follows from the fact that, since  
    $(C_1^\star,C_2^\star,\pi_1^\star,\pi_2^\star)=(g_1^{\theta^\star},g_2^{\theta^\star},g_1^{r^\star},g_2^{r^\star})$, relations 
    (\ref{ver-eq}) can be written
    \begin{eqnarray*}  
      E(g_1,\vec{C}_{\theta}^\star) &=& E(g_1^{\theta^\star} , \vec{u}_{\com}) \cdot E(g_1^{r^\star},\vec{u}_1) = E(g_1 , \vec{u}_{\com}^{\theta^\star}) \cdot E(g_1,\vec{u}_1^{r^\star}) \\ \nonumber
      E(g_2,\vec{C}_{\theta}^\star) &=& E(g_2^{\theta^\star} , \vec{u}_{\com}) \cdot E(g_2^{r^\star},\vec{u}_1) =E(g_2 , \vec{u}_{\com}^{\theta^\star}) \cdot E(g_2,\vec{u}_1^{r^\star})
    \end{eqnarray*}
    which uniquely determines the  only commitment $\vec{C}_{\theta}^\star=\vec{u}_{\com}^{\theta^\star} \cdot \vec{u}_1^{r^\star} \in \hat{\G}^2$ that satisfies (\ref{ver-eq}). 
    This shows that $\Pr[F_5] = 0$, as claimed. 
    \smallskip \smallskip 





  \item[\textsf{Game} $6$:] In this game, we modify the distribution of the public key.  Namely, instead of generating 
    the vectors $(\vec{u}_1,\vec{u}_2)$ as in (\ref{vec-PK-sim}), we 
    set 
    \begin{eqnarray} \label{vec-PK-sim-bis} 
      \vec{u}_1 &=& (\hat{g}, \hat{h}) \in \hat{\G}^2  \\ \nonumber 
      \vec{u}_2 &=&  (\hat{g}^{\rho_u}, \hat{h}^{\rho_u}) \cdot (1,{\hat{C}^\star~}^{-1})  \in \hat{\G}^2.
    \end{eqnarray}
    Said otherwise,  $\vec{u}_2$ is now the product of two terms, the first one of which lives in the 
    one-dimensional subspace spanned by $\vec{u}_1$. Under the DDH assumption in $\hat{\G}$, this modified  
    distribution of $\PK$ should have not noticeable impact on the adversary's behavior.  
    A straightforward reduction shows 
    that $|\Pr[W_6]-\Pr[W_5] | \leq \mathbf{Adv}^{\mathrm{DDH}}_\B (\lambda)$. Note that, although the vectors $(\vec{u}_{\com^\star},\vec{u}_1) \in \hat{\G}^2$ are 
    no longer linearly independent, $\vec{C}_{\theta}^\star = \vec{u}_1^{\rho_u \cdot \theta^\star +r^\star}$ remains the only commitment 
    that satisfies the verification equations  for a given tuple $(C_1^\star,C_2^\star,\pi_1^\star,\pi_2^\star)$.
    

    \smallskip \smallskip 

  \item[\textsf{Game} $7$:] In this game, we  modify the challenge ciphertext and replace the NIZK proof $\vec{\pi}^\star=(\pi_1^\star,\pi_2^\star) \in \G^2$ by a simulated proof which is produced 
    using $\rho_u \in \Z_p$ as a simulation trapdoor. Namely, $(\vec{C}_\theta^\star,\vec{\pi}^\star)$ is obtained by picking $r \sample \Z_p$ and  computing
    \begin{eqnarray*}
      \vec{C}_{\theta}^\star &=& \vec{u}_1^{r},   \qquad \qquad \quad
      \pi_1^\star  =  g_1^{r} \cdot {C_1^\star }^{-\rho_u} , \qquad \qquad \quad 
      \pi_2^\star  =  g_2^{r} \cdot {C_2^\star }^{-\rho_u}
    \end{eqnarray*}
    Observe that, although $(\vec{C}_\theta^\star,\pi_1^\star,\pi_2^\star)$ are generated without using the witness $\theta^\star = \log_{g_1}(C_1^\star) =
    \log_{g_2}(C_2^\star)$,  the NIZK property of 
    GS proofs ensures that 
    their distribution remains exactly as in Game $6$: indeed, if we define $\tilde{r} =r -\rho_u \cdot \theta^\star$, we have
    \begin{eqnarray*}
      \vec{C}_{\theta}^\star &=& \vec{u}_{\com^\star}^{\theta^\star} \cdot \vec{u}_1^{\tilde{r}},   \qquad \qquad \quad
      \pi_1^\star  =  g_1^{\tilde{r}} , \qquad \qquad \quad 
      \pi_2^\star  =  g_2^{\tilde{r}} ,
    \end{eqnarray*}
    which implies $\Pr[W_7]=\Pr[W_6]$.  
    \smallskip \smallskip


  \item[\textsf{Game} $8$:]  We modify the generation of the challenge ciphertext, which is generated using the private key $\SK=(x_1,x_2)$ instead
    of the public key: Namely, the challenger computes 
    \begin{align*}
      C_1^\star &= g_1^{\theta^\star}, & C_2^\star &= g_2^{\theta^\star},    &  C_0^\star &= M_{\beta} \cdot {C_1^\star}^{x_1} \cdot  {C_2^\star}^{x_2} , 
    \end{align*} 
    while $(\vec{C}_\theta^\star,\pi_1^\star,\pi_2^\star)$ are computed using the NIZK simulation trapdoor $\rho_u \in \Z_p$ as in Game $7$. 
    This modification does not affect the adversary's view since the ciphertext retains exactly the same distribution as in Game $7$. 
    We have $\Pr[W_8]=\Pr[W_7]$.  \smallskip \smallskip

  \item[\textsf{Game} $9$:] We modify again the distribution of the challenge ciphertext which is obtained as 
    \begin{align*}
      C_1^\star &= g_1^{\theta_1^\star}, & C_2^\star &= g_2^{\theta_2^\star},    &  C_0^\star &= M_{\beta} \cdot {C_1^\star}^{x_1} \cdot  {C_2^\star}^{x_2} , 
    \end{align*} 
    for random and independent $\theta_1^\star,\theta_2^\star \sample \Z_p$, 
    while the NIZK proof $(\vec{C}_\theta^\star,\pi_1^\star,\pi_2^\star)$ is simulated using $\rho_u \in \Z_p$ as in Game $8$.  Since 
    the witness $\theta^\star \in \Z_p$ was not used anymore in Game $8$, a straightforward reduction shows that  any noticeable change in $\A$'s output distribution implies a DDH distinguisher in $\G$. We have 
    $|\Pr[W_9]-\Pr[W_8]| \leq   \mathbf{Adv}_{\B,\G}^{\mathrm{DDH}}(\lambda)$. \medskip 

    We remark that, although we now have  
    $\log_{g_1}(C_1^\star) \neq \log_{g_2}(C_2^\star)$ with overwhelming probability, the signed ciphertext components $(C_1^\star,C_2^\star,\pi_1^\star,\pi_2^\star)$ 
    still uniquely determine $\vec{C}_{\theta}^\star \in \hat{\G}^2$ as the only commitment that satisfies the verification equations of 
    $(\vec{C}_{\theta}^\star,\pi_1^\star,\pi_2^\star)$:
    indeed, the equalities 
    \begin{eqnarray*}
      E(g_1,\vec{C}_\theta^\star) &=& E(C_1^\star, \vec{u}_{\com^\star}) \cdot E(\pi_1^\star ,\vec{u}_1)  \\ 
      E(g_2,\vec{C}_\theta^\star) &=& E(C_2^\star, \vec{u}_{\com^\star}) \cdot E(\pi_2^\star ,\vec{u}_1) 
    \end{eqnarray*} 
    can be written 
    \begin{eqnarray*}
      E(g_1,\vec{C}_\theta^\star) &=& E(C_1^\star , \vec{u}_1^{\rho_u} ) 
      \cdot E(  g_1^{r} \cdot {C_1^\star }^{-\rho_u} , \vec{u}_1)  = E(g_1^r,\vec{u}_1)  \\
      E(g_2,\vec{C}_\theta^\star) &=& E(C_2^\star , \vec{u}_1^{\rho_u} ) 
      \cdot E(  g_2^{r} \cdot {C_2^\star }^{-\rho_u} , \vec{u}_1) =E(g_2^r,\vec{u}_1),
    \end{eqnarray*}
    which implies that $\vec{C}_{\theta}^\star = \vec{u}_1^r$ is the only commitment satisfying (\ref{ver-eq}). Hence, at any step of the sequence of 
    games, $\vec{C}_\theta^\star$ 
    is always uniquely determined by $(C_1^\star,C_2^\star,\pi_1^\star,\pi_2^\star)$ and does not have to be signed. 
    \smallskip \smallskip

    

  \end{description}

  In the final game, it is easy to see that $\Pr[W_9]=1/2$ since the challenge ciphertext does not carry any information about $\beta \in \{0,1\}$. 
  Indeed, we have 
  \begin{align*}
    C_1^\star &= g_1^{\theta_1^\star}, & C_2^\star &= g_2^{\theta_1^\star + \theta_1'},    &  C_0^\star &= M_{\beta} \cdot X^{\theta_1^\star} 
    \cdot  {g_2}^{\theta_1' \cdot x_2} , 
  \end{align*} 
  for some random $\theta_1' \in_R \Z_p$, which implies that 		the term ${g_2}^{\theta_1' \cdot x_2}$ perfectly hides $M_\beta$ in the expression of $C_0^\star$. This follows from the fact
  that $x_2 \in \Z_p$ is perfectly independent of the adversary's view.	Indeed, the public key leaves $x_2 \in \Z_p$ completely undetermined as it only reveals $X=g_1^{x_1} g_2^{x_2}$. During the game, decryption queries are guaranteed not to reveal anything about $x_2$ since all 
  NIZK proofs $(\vec{C}_{\theta},\pi_1,\pi_2)$ take place on   Groth-Sahai CRSes $(\vec{u}_{\com},\vec{u}_1)$ which are perfectly sound (as 
  they span the entire vector space $\hat{\G}^2$)
  whenever $\com \neq \com^\star$. This implies that, although the adversary can see a simulated NIZK proof $(\vec{C}_\theta^\star,\pi_1^\star,\pi_2^\star)$ for 
  a false statement in the challenge phase, it remains  unable to trick the decryption oracle into accepting a  ciphertext 
  $ \vec{C} = (\SVK, \com, \open, C_0, C_1, C_2, \vec{C}_{\theta}, \vec{\pi}, \vec{\sigma})  $ such that $\log_{g_1}(C_1) \neq \log_{g_2}(C_2)$. 
  As a consequence, the adversary does not learn anything about $x_2$ from responses of the decryption oracle. 
  
\end{proof}






