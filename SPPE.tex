Our idea of constructing a RCCA scheme is to combine the encryption scheme and a proof system.
Intuitively, we observe that in the CCA2 constructions, usually the component which contains the message is re-randomizable(without considering other elements),
those elements non-rerandomizable works as a proof of the validity of the ciphertext part.
Thus if we encrypt a message with a CCA2 encryption scheme,
then replace the verification part by an rerandomizable NIZK proof, then we can achieve computational unlinkable RCCA scheme.

However, this approach needs the underlying public key encryption scheme to be structure-preserving and publicly verifiable.
Current existing scheme with such properties is not very efficient($381\G$ group elements~\cite{DBLP:conf/pkc/AbeDKNO13}).

In this section, we use the all-but-one hash proof techniques proposed by~\cite{DBLP:conf/tcc/LibertY12}. We combine it with the previous trapdoor commitment scheme and a strongly unforgeable signature scheme,using the property CMTCR (Chosen-Message Target Collision Resistant) of the trapdoor commitment scheme, we can construct a wanted CCA-2 structure preserving publicly verifiable encryption scheme.
\begin{description}

\item[\boldmath{$SPCCA.\KeyGen(1^\lambda)$}]:
  \begin{enumerate}
  \item Choose a asymmetric pairing group system $(\G, \hat{\G}, \G_T)$, groups of prime order $p > 2^\lambda$.
  \item Set $\PPP$ as $(\G, \hat{\G}, \G_T)$.
  \item Choose also group generators $g_1, g_2 \sample \G$ and random values $x_1, x_2 \sample \mathbb{Z}_p$.
  \item Generate group generator $\hat{g} \sample \hat{\G}$
  \item Set $X = g_1^{x_1}g_2^{x_2}$.
  \item Choose random values $\rho_u,\rho_u' \sample \Z_p$ and random group generators $(\hat{g}, \hat{h}) \sample \hat{\G}^2$.
  \item Set $(\vec{u}_1, \vec{u}_2)$ as $\vec{u}_1 = (\hat{g}, \hat{h}) \in \hat{\G}^2$ and $\vec{u}_2 =  (\hat{g}^{\rho_u}, \hat{h}^{\rho_u'}) \in \hat{\G}^2$. Note that $\vec{u}_1$ and $\vec{u}_2$ are linearly independent with overwhelming probability.
  \item Set $\PPP_{TC} = (\PPP, g_1, \hat{g}, \ell = 6)$.
  \item Generate the commitment key $\vec{\ck} \in \hat{\G}^8$ and $\vec{\tk} \in \mathbb{Z}_p^8$
  \item Define $\SK = (x_1, x_2)$ and $\PK = (g_1, g_2, \vec{u}_1, \vec{u}_2, X, \PPP_{TC}, \vec{\ck})$
  \end{enumerate}
\item[\boldmath{$SPCCA.\Enc(M,\PK)$}]:
  \begin{enumerate}
  \item Generate the one-time signature keys $(\SSK, \SVK) \gets OT1.\KeyGen(\PPP)$ with $\SSK = (\{\chi_i, \gamma_i\}_{i=1}^5, \zeta, \rho) \in \mathbb{Z}_p^{12}$ and $\SVK =  (\{\hat{g}_i\}_{i =1}^5,  \hat{A} ) \in \hat{\G}^6$.
  \item Choose $\theta \sample \mathbb{Z}_p$ and compute
    \begin{align*}
      C_0 &= M\cdot X^{\theta}, & C_1 &= g_1^{\theta}, & C_2 &= g_2^{\theta}.
    \end{align*}
  \item Generate a commitment to $\SVK = (\{\hat{g}_i\}_{i =1}^5, \hat{a})$ and let 
    $$(\com, \open) \gets TC.\Com(\PPP_{TC}, \vec{ck}, \SVK) \in \hat{\G} \times (\G^9 \times \hat{\G}^2)$$
    be the resulting commitment/decommitment pair.
  \item Define vector $\vec{u}_{\com} = \vec{u}_2\cdot (1, \com)$ and the Groth-Sahai CRS $\mathbf{u}_{\com}=(\vec{u}_{\com},\vec{u}_1)$. 
  \item Pick $r \sample \mathbb{Z}_p$. Compute $\vec{C}_{\theta} = \vec{u}_{\com}^{\theta} \cdot (\vec{u}_1)^r$.
  \item Using the randomness of the commitment $\vec{C}_{\theta}$, generate  proof elements $\vec{\pi}=(\pi_1,\pi_2)=(g_1^r,g_2^r) \in \G^2$ showing that the committed $\theta \in \Z_p$ satisfies the multi-exponentiation equations
    \begin{align*}
      C_1 &= g_1^{\theta} & C_2 &= g_2^{\theta}
    \end{align*}
  \item Output the ciphertext
    \begin{align*}
      \vec{C} = (\SVK, \com, \open, C_0, C_1, C_2, \vec{C}_{\theta}, \vec{\pi}, \vec{\sigma}) \in \G^{16} \times \hat{\G}^{11}
    \end{align*}
    
    
    in which $\vec{\sigma} = OT1.\Sig(\SSK, (C_0, C_1, C_2, {\pi}_1,\pi_2)) \in \G^2$.

    Notice that we don't sign the commitments because in the Groth-Sahai proof system and in this very special case, there is only one valid commitment for given proofs.
    
  \end{enumerate}
  
\item[\boldmath{$SPCCA.\Dec(\PK, \vec{C}, \SK)$}]:
  \begin{enumerate}
  \item Parse $\PK$ with $(\vec{g}_1, \vec{g}_2, X, \PPP_{TC}, \ck)$ and $\SK$ with $(x_1, x_2)$.
  \item Parse $\vec{C}$ with $ (\SVK, \com, \open, C_0, C_1, C_2, \vec{C}_{\theta}, \vec{\pi}, \vec{\sigma})$.
  \item Verify the signature is valid $OT1.\Verif(\PPP,  (C_0, C_1, C_2, \pi_1,\pi_2), \sigma) = \True$.
  \item Using the commitment verification algorithm to verify that $TC.\Verif(\ck, \com, \SVK, \open) = \True$
  \item Verify that $\vec{\pi}=(\pi_1,\pi_2)$ is a valid Groth-Sahai proof \wrt  $(C_1, C_2, \vec{C}_{\theta}, \com)$. Namely, 
    it should satisfy 
    \begin{eqnarray} \label{ver-eq} 
      E(g_1,\vec{C}_{\theta}) &=& E(C_1 , \vec{u}_{\com}) \cdot E(\pi_1,\vec{u}_1) \\ \nonumber
      E(g_2,\vec{C}_{\theta}) &=& E(C_2 , \vec{u}_{\com}) \cdot E(\pi_2,\vec{u}_1)
    \end{eqnarray}
  \item If any of the verification fails then halt and return $\bot$, otherwise, output $M=C_0/(C_1^{x_1}\cdot C_2^{x_2})$.
  \end{enumerate}
  
\end{description}


\begin{myTh}
  The scheme provides IND-CCA2 security under the SXDH assumption.
\end{myTh}


The proof of this theorem is very similar than the proof of the next section, we give a full description of this proof in the appendix~\ref{ProofSPPE}.
