\begin{myTh}
  The scheme $SPPE$ provides IND-CCA2 security under the SXDH assumption.
\end{myTh}


\begin{proof}

  This will be a game based proof. From the first game which is the definition of the CCA2 security game to the last game, in which the adversary can trivially not have any advantage. In the $i$-th game, we define the advantage of the adversary by $S_i$.

  \begin{description}
  \item[\textsf{Game} $0$ :] This is the real game, the adversary is against the CCA2 security game. We give the adversary the public key $\PK$ of the encryption scheme which contains the proof vectors $(\vec{u}_1, \vec{u}_2)$ which verifies:
    \begin{align*}
      \vec{u}_1 &= (\hat{g}, \hat{h}) \in \hat{\G}^2\\
      \vec{u}_2 &= (\hat{g}^{\rho_u}, \hat{h}^{\rho_u'}) \in \hat{\G}^2
    \end{align*}

    The adversary has the access to the decryption oracle.
    
    Then during the challenge phase, the adversary chooses two messages $(m_0, m_1) \in \G^2$, then submits to the Challenger and obtains a challenge ciphertext
    \begin{align*}
      \vec{C}^* = (\SVK^*, \com^*, \open^*, C_0^*, C_1^*, C_2^*, \vec{C}_{\theta}^*, \vec{\pi}^*, \vec{\sigma}^*)
    \end{align*}
    especially we have:
    \begin{align*}
      C_0^* &= m_b \cdot X^{\theta^*} & C_1^* &= g_1^{\theta^*} & C_2^* &= g_2^{\theta^*}.
    \end{align*}

    The adversary has the access to the decryption oracle expect the ciphertext $\vec{C}^*$ after the challenge phase.

    At the end, the adversary outputs a bit $b'$, its advantage against the RCCA security game is defined by the winning probability of $S_0 = |\PR[b' = b] - \frac{1}{2}|$.

  \item[\textsf{Game} $1$ :] In this game, the challenger generate the signing key and verification key pair and the commitment $\com^*$ of the signature's verification key at the beginning of the game. Since this does not change the view of the adversary then we have $S_0 = S_1$.

  \item[\textsf{Game} $2$ :] In the $\KeyGen$ algorithm, we modify the generation of the public key. Instead of generate $(\vec{u}_1, \vec{u}_2)$ as in the \textsf{Game} 0, we define:
    \begin{align*}
      \vec{u}_1 &= (\hat{g}, \hat{h}) \in \hat{\G}^2\\
      \vec{u}_2 &= (\hat{g}^{\rho_u}, \hat{h}^{\rho_u'}) \cdot (1, (\com^*)^{-1}) \in \hat{\G}^2
    \end{align*}

    Since $\vec{u}_2$ always distributed uniformly over $\hat{\G}^2$, this change does not affect the view of the adversary. Thus we have $S_2 = S_1$.


  \item[\textsf{Game} $3$ :] In this game, we define a failure event $F_4$: the ciphertext submitted by the adversary to the decryption oracle during the first query phase(before the challenge phase) contains the commitment $\com$ which verifies $\com = \com^*$.

    If the event $F_3$ happens, then the experiment halts and outputs a random bit. Since $\com^*$ is chosen uniformly in the space $\hat{\G}$, and remains independent from the adversary's view until the challenge phase, then we have $|S_3 - S_2| \leq \PR[F_3] \leq q_D/p$ where $q_D$ represents the number of decryption queries before the challenge phase $p$ is the order of the group $\hat{\G}$.

  \item[\textsf{Game} $4$ :] In this game, we modify the decryption oracle during the second query phase, let us denote the event $F_4$: the ciphertext submitted by the adversary to the decryption oracle during the second query phase contains the commitment and the open $(\com, \open)$ which verifies $\com = \com^*$ but $\open \neq \open^*$.

    The experiment halts if $F_4$ occurs and outputs a random bit. Thus we have $|S_4- S_3| \leq \PR[F_4]$.

    And if $F_4$ occurs, we can easily construct an adversary $\mathcal{B}$ of the target collision-resistance of the underlying structure-preserving trapdoor commitment $SPC$ which contradicts the Double Pairing assumption.

    Thus we have $|S_4 - S_3| \leq \PR[F_4] \leq adv_{\mathcal{B}}^{TCR-CR}(\lambda) \leq adv_{\mathcal{B}}^{DP}(\lambda)$.

  \item[\textsf{Game} $6$ :] We modify again the decryption oracle during the second query phase.
    During the second query phase, the ciphertext submitted by the adversary contains $(\com, \open)$ such that $\com = \com^*$ and $\open = \open^*$ but $(C_0, C_1, C_2, \pi_1, \pi_2) \neq (C_0^*, C_1^*, C_2^*, \pi_1^*, \pi_2^*)$, we denote this event $F_6$. If $F_6$ occurs, the experiment halts and outputs a random bit. Thus we have $|S_6 - S_5| \leq \PR[F_6]$. The event $F_6$ is contradict the strong unforgeability of the underlying one-time signature. Thus we have $|S_6 - S_5| \leq adv_{\mathcal{B}}^{SUF-OTS}(\lambda) \leq adv_{\mathcal{B}}^{DP}(\lambda)$.
    

  \item[\textsf{Game} $6$ :] We modify again the decryption oracle. For the decryption query
    $$\vec{C} = (\SVK, \com, \open, C_0, C_1, C_2, \vec{C}_{\theta}, \vec{\pi}, \vec{\sigma}),$$
    let us denote the event $F_6$ as the following condition happens but $F_5$ does not occur:
    \begin{eqnarray} \label
      (\com,\open)=(\com^*,\open^*)  \quad \wedge \quad (C_0,C_1,C_2,\pi_1,\pi_2) = (C_0^*,C_1^*,C_2^*,\pi_1^*,\pi_2^*) 
      \quad \wedge \quad \vec{C}_{\theta} \neq \vec{C}_{\theta}^* .
    \end{eqnarray}

    We have $|S_6- S_5| = \PR[F_6]$. We now prove that $\PR[F_6] = 0$. For a given $(C_1^*,C_2^*,\pi_1^*,\pi_2^*) \in \G^4$, there exists only one commitment $\vec{C}_{\theta}^* \in \hat{\G}^2$ that satisfies the equalities:
    \begin{eqnarray*}
      E(g_1,\vec{C}_{\theta}) &=& E(C_1 , \vec{u}_{\com}) \cdot E(\pi_1,\vec{u}_1) \\
      E(g_2,\vec{C}_{\theta}) &=& E(C_2 , \vec{u}_{\com}) \cdot E(\pi_2,\vec{u}_1)
    \end{eqnarray*}
    Because the two equations give us the equalities:
    \begin{eqnarray*}  
      E(g_1,\vec{C}_{\theta}^\star) &=& E(g_1^{\theta^\star} , \vec{u}_{\com}) \cdot E(g_1^{r^\star},\vec{u}_1) = E(g_1 , \vec{u}_{\com}^{\theta^\star}) \cdot E(g_1,\vec{u}_1^{r^\star}) \\
      E(g_2,\vec{C}_{\theta}^\star) &=& E(g_2^{\theta^\star} , \vec{u}_{\com}) \cdot E(g_2^{r^\star},\vec{u}_1) =E(g_2 , \vec{u}_{\com}^{\theta^\star}) \cdot E(g_2,\vec{u}_1^{r^\star})
    \end{eqnarray*}

    Thus we have $|S_6- S_5| = \PR[F_6] = 0$
    
    
  \item[\textsf{Game} $7$ :] In this game, we modify the distribution of the public keys. We compute the public keys $(\vec{u}_1, \vec{u}_2)$ in the following way:
    \begin{align*}
      \vec{u}_1 &= (\hat{g}, \hat{h}) \in \hat{\G}^2\\
      \vec{u}_1 &= (\hat{g}^{\rho_u}, \hat{h}^{\rho_u}) \cdot (1, \com^{-1}) \in \hat{\G}^2
    \end{align*}

    Since the Challenger does not use $\rho_u$ or $\rho_u'$ in the security game, thus an adversary who can make difference between \textsf{Game} $6$ and \textsf{Game} $7$, is an adversary against the DDH assumption in $\hat{\G}$. Thus we have $|S_7 - S_6| \leq adv_{\mathcal{B}}^{DDH}(\lambda)$.

  \item[\textsf{Game} $8$ :] In this game, instead of generate the proof $(\vec{\pi}_{\theta^*}, \vec{\pi}_{\theta^*})$ using the witness $\theta^*$, we generate a random value $r \sample \mathbb{Z}_p^*$ and generate $(\vec{C}_{\theta^*}, \vec{\pi}_{\theta^*}, \vec{\pi}_{\theta^*})$ as following:
    \begin{align*}
      \vec{C}_{\theta^*} &= \vec{u}_1^r & \vec{\pi}_{\theta^*} &= g_1^r \cdot C_1^{*-\rho_u} & \vec{\pi}_{\theta^*} &= g_2^r \cdot C_2^{*-\rho_u}
    \end{align*}

    Notice that even the proof elements are generated without using the witness $\theta^* = log_{g_1}(C_1^*) = log_{g_2}(C_2^*)$, the distribution of the proof is still remain the same as in the original proof. In fact, let us define $\tilde{r} = r - \rho_u \cdot \theta^*$, we have:
    \begin{align*}
      \vec{C}_{\theta}^* &= \vec{u}_{\com}^{\theta^*} \cdot \vec{u}_1^{\tilde{r}} &  \vec{\pi}_{\theta^*,1,1} &= g_1^{\tilde{r}} & \vec{\pi}_{\theta^*,2,1} &= g_2^{\tilde{r}}
    \end{align*}

    Thus we have $S_8 = S_7$.

  \item[\textsf{Game} $9$ :] In this game, we modify the ciphertext generation in the challenge phase. Instead of compute the ciphertext using the public key $(X, g_1, g_2)$, we generate it with the secret key $(x_1, x_2)$:
    \begin{align*}
      C_1^* &= g_1^{\theta^*} & C_2^* &= g_2^{\theta^*} & C_0 &= M_b\cdot C_1^{*x_1} \cdot C_2^{*x_2}  
    \end{align*}
    Since the ciphertext remains exactly the same as in the \textsf{Game} $8$. Thus this modification does not change the view of the adversary, which means $S_9 = S_8$.


  \item[\textsf{Game} $10$ :] In this game, we modify again the ciphertext generation in the challenge phase. Recall that since the \textsf{Game} $9$, we don't use anymore $\theta^{*}$ to generate $(\vec{C}_{\theta^*}, \vec{\pi}_{\theta^*}, \vec{\pi}_{\theta^*})$, then we generate two random values $(\theta_1, \theta_2) \sample \mathbb{Z}_p^2$ and compute the ciphertext as following:
    \begin{align*}
      C_1 &= g_1^{\theta_1} & C_2 &= g_2^{\theta_2} & C_0 &= m_b \cdot C_1^{x_1} \cdot C_2^{x_2}
    \end{align*}

    As we don't use anymore $\theta_1$ nor $\theta_2$ in the whole game, we can easily construct a reduction from an adversary who can make difference between \textsf{Game} $10$ and \textsf{Game} $9$ to an adversary against the DDH assumption. Thus we have $|S_{10} - S_9| \leq adv_{\mathcal{B}}^{DDH}(\lambda)$.
    
  \end{description}

  Notice that in the final game, the ciphertext component is as follows:
  \begin{align*}
    C_1 &= g_1^{\theta^*} & C_2 &= g_2^{\theta^*+ \theta'} & C_0 &= m_b \cdot X_1^{\theta^*} \cdot g_2^{x_2 \cdot \theta'}
  \end{align*}

  As $x_2$ is completely independent of the adversary's view, $C_0$ can be seen as a one-time pad of the message $m_b$. Thus the adversary does not have any information about the bit $b$. Then we have $S_{10} = 0$.
For summary, we have
\begin{align*}
  adv_{\mathcal{A}}^{CCA2}(\lambda) &= S_0\\
  &\leq S_{10} + 2 \cdot adv_{\mathcal{B}}^{DDH}(\lambda) + 2 \cdot adv_{\mathcal{B}}^{DP}(\lambda) + q_D/p\\
  &= 2 \cdot adv_{\mathcal{B}}^{DDH}(\lambda) + 2 \cdot adv_{\mathcal{B}}^{DP}(\lambda) + q_D/p \in negl(\lambda)
\end{align*}

\end{proof}
