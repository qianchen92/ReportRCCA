\documentclass[11pt]{article}

\usepackage[backend=bibtex]{biblatex}
\bibliography{main}
\usepackage{fullpage}
%\usepackage[top=1.5cm, bottom=1.5cm, left=1.5cm, right=1.5cm]{geometry}
\usepackage[english]{babel}
\usepackage[utf8]{inputenc}
%\usepackage[latin1]{inputenc} 
\usepackage[T1]{fontenc}
\usepackage{lmodern}
\usepackage[onelanguage,boxed]{algorithm2e}
\usepackage{graphicx}
\usepackage{float}
\usepackage{amsmath}
\usepackage{amsfonts}
\usepackage{amsthm}
\usepackage{color}
\usepackage[usenames,dvipsnames]{xcolor}
\usepackage{listings}
\usepackage{tikz}
\usepackage{subfigure}
\usepackage{wrapfig}
\usepackage{hyperref}
\hypersetup{colorlinks,linkcolor=black,urlcolor=blue}
\usepackage[font=small,labelfont=bf]{caption}
\newtheorem {myDef} {Definition}
\newtheorem {myTh}{Theorem}
\lstset
{
  language=[Objective]Caml,
  basicstyle=\footnotesize,       % the size of the fonts that are used for the code
  numbers=left,                   % where to put the line-numbers
  numberstyle=\footnotesize,      % the size of the fonts that are used for the line-numbers
  stepnumber=1,                   % the step between two line-numbers. If it is 1 each line will be numbered
  numbersep=5pt,                  % how far the line-numbers are from the code
  backgroundcolor=\color{white},  % choose the background color. You must add \usepackage{color}
  showspaces=false,               % show spaces adding particular underscores
  showstringspaces=false,         % underline spaces within strings
  showtabs=false,                 % show tabs within strings adding particular underscores
  frame=leftline,           % adds a frame around the code
  tabsize=2,          % sets default tabsize to 2 spaces
  captionpos=b,           % sets the caption-position to bottom
  breaklines=true,        % sets automatic line breaking
  breakatwhitespace=false,    % sets if automatic breaks should only happen at whitespace
  escapeinside={\%*}{*)},          % if you want to add a comment within your code
  basicstyle=\ttfamily,
  keywordstyle=\color{BurntOrange}\ttfamily,
  stringstyle=\color{red}\ttfamily,
  commentstyle=\color{LimeGreen}\ttfamily,
  morecomment=[l][\color{RoyalBlue}]{\#}
}
\lstset
{
  emph={Stack},
  emphstyle={\color{DarkOrchid}}
}



\newcommand{\A}{\mathcal{A}}
\newcommand{\B}{\mathcal{B}}
\newcommand{\D}{\mathcal{D}}
\newcommand{\G}{\mathbb{G}}
\newcommand{\Z}{\mathbb{Z}}
\newcommand{\PR}{\operatorname{Pr}}
\newcommand{\PP}{\mathsf{P}}  
\newcommand{\VV}{\mathsf{V}}  
\newcommand{\K}{\mathsf{K}}  
\newcommand{\SIM}{\mathsf{S}}  
\newcommand{\lbl}{\mathsf{lbl}} 
\newcommand{\PPE}{\mathrm{PPE}} 
\newcommand{\SK}{\mathsf{SK}}
\newcommand{\PK}{\mathsf{PK}}
\newcommand{\VK}{\mathsf{VK}}
\newcommand{\SSK}{\mathsf{SSK}}
\newcommand{\SVK}{\mathsf{SVK}}
\newcommand{\sk}{\mathsf{sk}}
\newcommand{\ck}{\mathsf{ck}}
\newcommand{\tk}{\mathsf{tk}}
\newcommand{\msk}{\mathsf{msk}}
\newcommand{\vk}{\mathsf{vk}}
\newcommand{\ovk}{\mathsf{ovk}}
\newcommand{\pk}{\mathsf{pk}}
\newcommand{\opk}{\mathsf{opk}}
\newcommand{\osk}{\mathsf{osk}}
\newcommand{\com}{\mathsf{com}}
\newcommand{\open}{\mathsf{open}}
\newcommand{\True}{\mathsf{True}}
\newcommand{\False}{\mathsf{False}}
\newcommand{\BF}{\mathbf}
\newcommand{\sample}{\stackrel{{\scriptscriptstyle \mkern4mu R}}{\gets}}
\newcommand{\etal}{\textit{el. al.}}
\newcommand{\eg}{\textrm{e.g.} }
\newcommand{\ie}{\textrm{i.e.} }
\newcommand{\wrt}{\textrm{w.r.t.} }
\providecommand{\tprod}{{\textstyle\prod}}
\newcommand{\Setup}{{\mathsf{Setup}}}
\newcommand{\KeyGen}{{\mathsf{KeyGen}}}
\newcommand{\Enc}{{\mathsf{Enc}}}
\newcommand{\Dec}{{\mathsf{Dec}}}
\newcommand{\Sig}{{\mathsf{Sign}}}
\newcommand{\Verif}{{\mathsf{Verify}}}
\newcommand{\Prove}{{\mathsf{Prove}}}
\newcommand{\Com}{{\mathsf{Commit}}}
\newcommand{\PPP}{\mathsf{PP}}
\newcommand{\Forge}{\mathsf{Forge}}
\newcommand{\Adv}{\mathcal{A}}


\begin{document}

\title{Efficient RCCA rerandomizable encryption scheme}

\author{Qian Chen \and Supervisors: Beno\^it Libert, Fabien Laguillaumie \and Aric Lip ENS Lyon}
\date{August 21, 2016}

\maketitle

\thispagestyle{empty}

%% Attention: pas plus d'un recto-verso!
% Ne conservez pas les questions


\section{Introduction}
\subsection*{General Context}

For the simple needs of communicate safely and privately, cryptography is very important in our current life. 
Start with Shanon in 1949 in his paper Communication theory of secrecy systems~\cite{shannon-otp},
we begin to formally define the properties we wanted for the cryptographic protocols and prove these properties based on some hardness assumptions or complexity assumptions.
As one of most useful cryptographic protocol, the encryption scheme is widely used in the construction or more complex cryptographic system.
Thus we are motivated to define the most adapted security notion for the encryption scheme. 
From the very basic One-Wayness Chosen-Plaintext Attack(OW-CPA) to the most secure Indistinguishable Chosen-Ciphertext Attack(IND-CCA) model. 

One of the most important property of the encryption scheme is the malleability, 
which means with a valid ciphertext we can produce another ciphertext of a plaintext which is related to the original one without knowing it.
This property necessarily produce some information leakages,
thus it is forbidden by the most secure definition(IND-CCA).
But recently, these properties are seen to be a potentially useful feature that can be exploited.

\subsection*{Problem studied}
My internship focused on the encryption scheme which only allowed to be rerandomizable.
This property can be formally defined as resistant to Replayable Chosen-Ciphertext Attack(RCCA).
Some work on construct more complex cryptographic system like mix network proposed by Golle~\etal~\cite{DBLP:conf/ctrsa/GolleJJS04}
RCCA encryption scheme are required in such schemes.
However, the previous constructions~\cite{DBLP:conf/ctrsa/GolleJJS04} are suffered from the Chosen-Chipertext attack or based on some non standard assumptions~\cite{DBLP:conf/crypto/PrabhakaranR07} or not very efficient~\cite{DBLP:conf/eurocrypt/ChaseKLM12}.
The aim of my internship is to construct and prove efficient encryption scheme which are suitable for the above schemes.
The one of the main motivation is that the previous works on constructing such scheme are more or less not efficient.
We try to improve there efficiency to get some usable protocol based on the standard assumptions in the practical point of view.

\subsection*{Proposed Contributions}
The contributions of my internship are the following:
We first give an efficient instantiation of the general controlled-malleable encryption scheme proposed by~\cite{DBLP:conf/eurocrypt/ChaseKLM12} which ciphertext has $93\G$ elements.
Then we use another approach to get a very efficient computational RCCA encryption in which the the ciphertext size is only $39\G + 20 \hat{\G}$ elements.
As a sub-result, we also have constructed a public verifiable structure-preserving CCA encryption which is more efficient than the existing construction $16\G + 11\hat{\G}$ against $321\G$~\cite{DBLP:conf/pkc/AbeDKNO13}.

\subsection*{Arguments Supporting Their Validity}
For the validity of our construction,
every construction has been proven for the security model with standard complexity assumptions which are well studied and general believed.
And we also give their efficiency by counting their ciphertext size and compare with existing schemes to show that we achieve efficiency improvement.

\subsection*{Summary and Future Work}
During my internship, I have proposed several efficiency improvements for the construction of the cryptographic scheme,
This contribution can be considered as improvement both in the efficiency and the construction of the new scheme with some practical properties for the further construction of more complex cryptographic system.

However, several questions are left open. 
We especially studied the re-randomizable encryption scheme, which is a subset of homomorphic encryption scheme,
can we use the similar idea of the efficiency improvement for a wider class of homomorphic encryption scheme.
And, even in our RCCA scheme, the re-randomization is computational.
A natural open question is can we have RCCA scheme which rerandomization which is statistical unlinkable.



\newpage

\section{Preliminaries}
In this section, we briefly present some standard computation assumptions, we will use in this report.  
Then we present the security model and proprieties we want achieve we also give some building blocks.

\begin{subsection}{Assumptions}

  In the rest of this work, a negligible function $\varepsilon(\lambda)$ is a positive function which is asymptoticaly smaller than $2^{-\lambda}$.
  
  In cryptography, one of the most studied assumption is Diffie-Hellman assumption. In this work, we consider a slightly stronger variant of DDH assumption Symmetric external Diffie-Hellman(SXDH).

  \begin{myDef}{Decisional Diffie-Hellman(DDH)}
    For a security parameter $\lambda$, we say a group $\G$ of prime order $p>2^{\lambda}$ verifies the Diffie-Hellman assumption, if given a group generator $g \in \G$ and two triples of group elements $(g^a, g^b, g^{ab})$ and $(g^a, g^b, g^c)$ in which $(a, b, c)$ are random values generated randomly $(a, b, c) \gets \mathbb{Z}_p^3$, we define the advantage of an adversary $\Adv$ against the DDH problem by:

    \begin{align*}
      adv(\Adv) = |\PR(\Adv(g, g^a, g^b, g^{ab}) = 1) = \PR(\Adv(g, g^a, g^b, g^c) = 1)| < \varepsilon(\lambda).
    \end{align*}

    where $\varepsilon(\lambda)$ is a negligible function of the security parameter $\lambda$.
  \end{myDef}

  \begin{myDef}{Symmetric external Diffie-Hellman(SXDH)}
    For a security parameter $\lambda$ and for a asymmetric pairing group setting, three groups of prime order $p>2^\lambda$: $(\G, \hat{\G}, \G_T)$ and the pairing $e: \G \times \hat{\G} \to \G_T$. If there is no adversary with non-negligible advantage against the DDH problem both on the group $\G$ and $\hat{\G}$, then we say that the group triple $(\G, \hat{\G}, \G_T)$ verifies the SXDH assumption.
  \end{myDef}

  We also introduce Double Pairing assumption which can be applied by the SXDH assumption
  
  \begin{myDef}{Double Pairing(DP)}
    Given a asymmetric pairing setting $(\G, \hat{G}, \G_T, e:\G \times \hat{\G} \to \G_T)$. Given two random non-zero generators $(g_z, g_r) \in \G$,
    \begin{align*}
      \PR[(z, r) \gets \Adv| (z, r)\in \hat{\G}^2 \wedge e(g_z, z) \cdot e(g_r, r) = 1] \in negl(\lambda)
    \end{align*}
  \end{myDef}

  
  For the simplicity of the description of the algorithm, we also use symmetric pairing setting, in which the DDH and SXDH assumption are not verified, we introduce the DLIN assumption
  \begin{myDef}{Decisional Linear(DLIN)}
    For a security parameter $\lambda$, we say a group $\G$ of prime order $p>2^{\lambda}$ verifies the Decisional Linear assumption, if given three group generators $(f, g, h) \in \G^3$ and two triples of group elements $(f^a, g^b, h^{a+b})$ and $(f^a, g^b, h^c)$ in which $(a, b, c)$ are random values generated randomly $(a, b, c) \gets \mathbb{Z}_p^3$, we define the advantage of an adversary $\Adv$ against the DLIN problem by:

    \begin{align*}
      adv(\Adv) = |\PR(\Adv(f, g, h, f^a, g^b, h^{a+b}) = 1) = \PR(\Adv(f, g, h, f^a, g^b, h^c) = 1)| < \varepsilon(\lambda).
    \end{align*}

    where $\varepsilon(\lambda)$ is a negligible function of the security parameter $\lambda$.
  \end{myDef}


  

\end{subsection}




\begin{subsection}{Building Blocks}

  \begin{myDef}{Linearly Homomorphic Structure Preserving Signature based on SXDH assumption~\cite{DBLP:conf/crypto/LibertPJY13}}
    \begin{description}
    \item[\boldmath{$LHSPS.\Setup(1^{\lambda})$}]:
      \begin{enumerate}
      \item We generate a bilinear group system $(\G, \hat{\G}, \G_T, e:\G \times \hat{\G} \to \G_T)$.
      \item Choose random group generators $(\hat{g}_z, \hat{g}_r) \sample \hat{\G}^2$.
      \item Choose random group generator $g \sample \G$.
      \item Output $\PPP = (\hat{g}_z, \hat{g}_r, g)$.
      \end{enumerate}
    \item[\boldmath{$LHSPS.\KeyGen(\PPP)$}]:
      \begin{enumerate}
      \item Generate $(\{\hat{\chi}_i, \hat{\gamma}_i\}_{i =1 }^k, \hat{\zeta}, \hat{\rho}) \sample \mathbb{Z}_p^{2k+2}$.
      \item Compute for $i \in \{1, \dots, k\}$, $\hat{g}_i \gets \hat{g}_z^{\hat{\chi}_i}\hat{g}_r^{\hat{\gamma}_i}$.
      \item Output $\vk = (\{\hat{g}_i\}_{i=1}^k) \in \hat{\G}^{k}$ and $\sk = (\{\hat{\chi}_i, \hat{\gamma}_i\}_{i=1}^k) \in \mathbb{Z}_p^{2k}$.
      \end{enumerate}
    \item[\boldmath{$LHSPS.\Sig(\sk, \{m_1, \dots, m_k\})$}]: where $(m_1, \dots, m_k) \in \G^k$
      \begin{enumerate}
      \item Parse $\sk$ with $(\{\hat{\chi}_i, \hat{\gamma}_i\}_{i = 1}^k)$.
      \item Compute
        \begin{align*}
          z &= \prod_{i=1}^km_i^{\hat{\chi}_i} & r &= \prod_{i =1}^km_i^{\hat{\gamma}_i} 
        \end{align*}
      \item Output the signature $\sigma = (z, r)$
      \end{enumerate}

    \item[\boldmath{$LHSPS.\Verif(\vk, \sigma, \{m_1, \dots, m_k\})$}]:
      \begin{enumerate}
      \item Parse the signature $\sigma$ with $\sigma = (z, r)$ and the verification key $\vk$ with $\vk = (\hat{g}_1, \dots, \hat{g}_k)$
      \item Verify the pairing equation:
        \begin{align*}
          e(z, \hat{g}_z) \cdot e(r, \hat{g}_r) &=  \cdot \prod_{i = 1}^ke(m_i, \hat{g}_i)
        \end{align*}
      \end{enumerate}
    \end{description}
  \end{myDef}


  \begin{myDef} A LHSPS for the message vectors of size $k$ is strongly unforgeable if no PPT(probabilistic polynomial Turing machine) adversary has non-negligible advantage in the following game:
    \begin{description}
    \item[Init phase: ]The challenger use $\Setup$ and $\KeyGen$ to generate the public parameters $\PPP$, verification key $\VK$ and signing key$\SK$. Then send $\PPP$ and $\VK$ to the challenger.
    \item[Signing queries: ] The adversary has access of a signing oracle, he can require a polynomial number of messages $\{\vec{M}_i\}_{i = 1}^q$ to sign. 
    \item[Challenge phase: ] The adversary outputs a message and signature pair $(\vec{M}^*, \sigma^*)$.
    \end{description}

    \begin{align*}
      adv(\Adv) &= \PR[LHSPS.\Verif(\VK, \vec{M}^* , \sigma) = \True \wedge \vec{M}^* \not \in Span(\{\vec{M}_i\}_{i=1}^q)]
    \end{align*}

  \end{myDef}

  The following theorem is proved in~\cite{DBLP:journals/dcc/LibertPJY15}.
  
  \begin{myTh}
    The previous constructed LHSPS is strongly unforgeable.
  \end{myTh}
  
  %  The security of the One-Time Linearly Homomorphic Structure Preserving Signature is defined by the following games

  In the construction of the structure preserving publicly verifiable encryption scheme, we also need a trapdoor commitment proposed by Abe \etal~\cite{DBLP:conf/eurocrypt/AbeKOT15}.

  \begin{description}
  \item[\boldmath{$TC.\Setup(1^{\lambda},\ell)$}]: Generate the public parameters for the trapdoor commitment scheme for security parameter $\lambda$ and message vector length $\ell$. 
    \begin{enumerate}
    \item Choose a random prime $p< 2^{\lambda}$.
    \item Generate a asymmetric pairing groups $(\G, \hat{\G}, \G_T)$ of prime order $p$ and a pairing function $e : \G \times \hat{\G} \to \G_T$.
    \item Generate the group generators $(g, \hat{g}) \in \G \times \hat{\G}$.	
    \item $\PPP = (p, \G, \hat{\G}, \G_T, e, g, \hat{g},\ell)$.
    \end{enumerate}

  \item[\boldmath{$TC.\KeyGen(\PPP)$}]:
    \begin{enumerate}
    \item For $i = 1, \dots, \ell+2$, generate random values $\rho_i \sample \mathbb{Z}_p^*$, then compute $\hat{X}_i \gets \hat{g}^{\rho_i}$.
    \item Set $\ck \gets \{\hat{X}_i\}_{i = 1}^{\ell+2}$ and $tk \gets \{\rho_i\}_i^{\ell+2}$.
    \end{enumerate}


  \item[\boldmath{$TC.\Com(\PPP, \ck, \vec{M})$}]: where $\vec{\hat{M}} = (\hat{M}_1, \dots, \hat{M}_\ell) \in \hat{\G}^\ell$.
    \begin{enumerate}
    \item Choose a random value $w_z \in \mathbb{Z}_p^*$ then compute $g_z = g^{w_z}$.
    \item For $i = 1, \dots, \ell$, generate random values $\chi_i \sample \mathbb{Z}_p$ and compute $g_i = g^{\chi_i}$.
    \item Set $\vk_{pots} \gets (g_z, g_1, \dots, g_\ell) \in \G^{\ell+1}$ and $\sk_{pots} \gets (w_z,\chi_1, \dots, \chi_\ell)$.
    \item Choose randomly $a \sample \mathbb{Z}_p$, then set $\ovk_{pots} =A = g^{a}$ and $\osk_{pots} = a$.
    \item Using the signing key $\sk_{pots}$ to generate signatures of the message $\vec{\hat{M}}$ \wrt to the one-time signature's secret key $\osk_{pots}$:
      \begin{enumerate}
      \item Generate random value $\zeta_1 \in \mathbb{Z}_p$
      \item Compute the signature $(\hat{Z}, \hat{R}) \in \hat{\G}^2$ for $\vec{\hat{M}}$:
	\begin{align*}
	  \hat{Z} &= \hat{g}^{\zeta_1} & \hat{R} = \hat{g}^{a-\zeta_1 w_z}\prod_{i=1}^{\ell} \hat{M}_i^{\chi_i} 
	\end{align*}
      \end{enumerate}
    \item We use the commitment key to generate the commitment for the message.
      \begin{enumerate}
      \item Set $(m_1, \dots, m_{\ell+2}) \gets (\chi_1, \dots, \chi_\ell, w_z, a)$ 
      \item Parse $\vec{\ck}$ as $(\hat{X}_1, \dots, \hat{X}_{\ell+2})$.
      \item Generate a random value $\zeta_2 \gets \mathbb{Z}_p^*$ and compute:
	\begin{align*}
	  \hat{C} &= \hat{g}^{\zeta_2}\prod_{i = 1}^{\ell+2}\hat{X}_i^{m_i} & D &= g^{\zeta_2}
	\end{align*}
      \end{enumerate}
    \item We set the commitment as $\com  = \hat{C}$ and $\open = (D, g_z, g_1, \dots, g_\ell, \ovk_{pots} = g^a, \hat{Z}, \hat{R}) \in \G^{\ell+3} \times \hat{\G}^{2}$.
    \end{enumerate}
    
  \item[\boldmath{$TC.\Verif(\ck, \com, \vec{\hat{M}}, \open)$}] :
    \begin{enumerate}
    \item Parse $\vec{\hat{M}}$ with $(\hat{M}_1, \dots, \hat{M}_\ell)$ and $\open$ with $(D, g_z, g_1, \dots, g_\ell, \ovk_{pots} = g^a, \hat{Z}, \hat{R})$.
    \item Set $\vec{N} = (N_1, \dots, N_{\ell+2}) = (g_1, \dots, g_\ell, g_z, \ovk_{pots})$
    \item Using $\ovk_{pots} = A \in \G$, verify the following equations:
      \begin{align*}
	e(g, \hat{C}) &= e(D, \hat{g}) \prod_{i = 1}^{\ell+2} e(N_i, \hat{X}_i) & e(A, \hat{g}) &= e(g_z, \hat{Z}) \cdot e(g, \hat{R}) \cdot \prod_{i = 1}^\ell e(g_i, \hat{M}_i) 
      \end{align*}
    \end{enumerate}
  \end{description}

  \begin{myDef}{Chosen-Message Target Collision Resistance} This property of a Trapdoor Commitment scheme $TC$ is defined by the winning probability of the adversary against the following security game:
    \begin{description}
    \item[Init phase:] The challenger generate the public parameters $\PPP_{TC} \gets TC.\Setup$.
    \item[Query phase:] The adversary has oracle access of the commitment algorithm $\Com$. For each query of message $m$, the challenger compute $(\com, \open) \gets TC.\Com(m)$. Record them in the hash table $Q$. and outputs $(\com, \open)$ to the adversary.
    \item[Challenge phase:] The adversary outputs message-commitment triple $(m^*, \com^*, \open^*)$.
    \end{description}
    The advantage of the adversary is defined by the following probability:
    \begin{align*}
      \PR[(m^*, \com^*, \open^*) \gets \Adv | \com^* \in Q \wedge (\com^*, m^*) \not \in Q \wedge TC.\Verif(\vec{ck}, \com^*, m^*, \open^*) = \True].
    \end{align*}
  \end{myDef}

  The following theorem is proved in~\cite{DBLP:conf/eurocrypt/AbeKOT15}.

  \begin{myTh}
    The previous Trapdoor Commitment scheme is Chosen-Massage Target Collision Resistant.
  \end{myTh}

  

\end{subsection}


\begin{subsection}{Security Notions}

  
  An public key encryption scheme is a quadruple of algorithms $\mathcal{E} = (\Setup, \KeyGen, \Enc, \Dec)$.

  \begin{myDef}{CCA-2}
  The CCA-2 security of the public key encryption is defined by the wining probability of the adversary in the following game:
  \begin{description}
  \item[Init phase]:
    We generate the public parameter \wrt the secure parameter $\lambda$. The Challenger use $\KeyGen$ algorithm to generate a pair of public key $\pk$ and secrect key $\sk$, then give the public key to the adversary.
  \item[Quary phase 1]: The adversary has the oracle access to the decryption oracle, he can decrypt polynomialy many ciphertext by the decryption oracle.
  \item[Challenge phase]: The adversary choose two message $(m_0, m_1)$, then submits them to the challenger. The challenger choose randomly a bit $b \in \{0,1\}$, then encrypts the message $m_b$ using the public encryption key $\pk$ and return the result ciphertext $c_b$ to the adversary.
  \item[Quary phase 2]: The adversary has the oracle access to the decryption oracle expect that he can not require the oracle to decrypt the ciphertext $c_b$.
  \item[Gussing phase]: The adversary output a bit $b'$.
  \end{description}

  The encryption scheme $\mathcal{E}$ is CCA-2 secure iff $adv(\Adv) = |\PR(b = b') - \frac{1}{2}| < negl(\lambda)$ where $negl(\lambda)$ is a negligible function \wrt $\lambda$.
  \end{myDef}

  We also define the Replayable-CCA encryption scheme(RCCA)
  \begin{myDef}{RCCA}
    The RCCA security of the public key encryption is defined by the wining probability of the adversary in the following game:
    \begin{description}
    \item[Init phase]:
      We generate the public parameter \wrt the secure parameter $\lambda$. The Challenger use $\KeyGen$ algorithm to generate a pair of public key $\pk$ and secrect key $\sk$, then give the public key to the adversary.
    \item[Query phase 1]: The adversary has the oracle access to the decryption oracle, he can decrypt polynomialy many ciphertext by the decryption oracle.
    \item[Challenge phase]: The adversary choose two message $(m_0, m_1)$, then submits them to the challenger. The challenger choose randomly a bit $b \in \{0,1\}$, then encrypts the message $m_b$ using the public encryption key $\pk$ and return the result ciphertext $c_b$ to the adversary.
    \item[Query phase 2]: The adversary has the oracle access to the decryption oracle expect that when the oracle receive a ciphertext, if the result of the decryption is equal to $m_0$ or $m_1$, then the oracle returns $Replay$.
    \item[Guessing phase]: The adversary output a bit $b'$.
  \end{description}

  The encryption scheme $\mathcal{E}$ is CCA-2 secure iff $adv(\Adv) = |\PR(b = b') - \frac{1}{2}| < negl(\lambda)$ where $negl(\lambda)$ is a negligible function \wrt $\lambda$.
  \end{myDef}

  We also define the unlinkability of the RCCA encryption scheme which is first proposed by Prabhakaran \etal~\cite{DBLP:conf/crypto/PrabhakaranR07}.
  \begin{myDef}{Unlinkability}
    Let RCCA encryption be the following four algorithm $(\Setup, \KeyGen, \Enc, \Dec, \Rerand)$
    The unlinkability of RCCA encrypton scheme is defined by the wining probability of the adversary in the following game:
    \begin{description}
    \item[Init phase]:
      We generate the public parameter \wrt the secure parameter $\lambda$. The Challenger use $\KeyGen$ algorithm to generate a pair of public key $\pk$ and secrect key $\sk$, then give the public key to the adversary.
    \item[Query phase 1]: The adversary has the oracle access to the decryption oracle, he can decrypt polynomialy many ciphertext by the decryption oracle.
    \item[Challenge phase]: The adversary choose outputs a ciphertext $C$, then submits it to the challenger. If $\Dec(C)\neq \bot$, the challenger choose randomly a bit $b \in \{0,1\}$, Then if $b = 0$, the challenger outputs $\Enc(\Dec(C))$, otherwise he outputs $\Rerand(C)$.
    \item[Query phase 2]: The adversary has again the oracle access to the decryption oracle, he can decrypt polynomialy many ciphertext by the decryption oracle.
    \item[Guessing phase]: The adversary output a bit $b'$.
    \end{description}

    The RCCA scheme is computational(\resp statistical) unlinkable if all PPT(\resp omnipotent) adversary has negligible advantage against the previous game. 
  \end{myDef}
  

\end{subsection}



\section{Efficient instantiation of the re-randomization encryption of generic construction~\cite{DBLP:conf/eurocrypt/ChaseKLM12}}
For the simplicity, we choose the symmetric setting of the bilinear group and our construction is based on the DLIN assumption.

\begin{description}
\item[\boldmath$RCCA1.\Setup(\lambda)$:] This algorithm generates the public and secret keys of our RCCA encryption scheme.
  \begin{enumerate}
  \item Pick bilinear group $(\G, \G_T)$ of prime order $p$ and the bilinear map $e$ on this pair of groups with generators $(f,g,h) \sample \G^3$ which verify $f^x = g^y = h$.
  \item Choose a random group generator $d \in \G$.
  \item Choose $g_1,g_2 \sample \G$ and set $\vec{g}_1 = (g_1,1,g) \in \G^3$, $\vec{g}_2 = (1,g_2,g) \in \G^3$ and $\vec{g}_3 \sample \G^3$.
  \item Set up the keys for the underlying encryption scheme $\pk_{enc} = (f,g,h)$ and $\sk_{enc} = (x,y)$.
  \item Choose four random group generators $(g_z, g_r, h_z, h_u)$ and twelve random exponents $\{\chi_i, \gamma_i, \delta_i\}_{i = 1}^3, \zeta, \rho, \phi \sample \Z_p$, then compute $(g_i,h_i) = (g_z^{\chi_i}g_r^{\gamma_i}, h_z^{\chi_i}h_u^{\delta_i})$ for $ i \in \{1,2,3\}$ and $(\alpha,\beta) = (g_z^\zeta g_r^\rho, h_z^\zeta h_u^\phi)$
  \item Set up the keys for the underlying signature scheme
    $$\vk_{sig} = (g_z, h_z, g_r, h_u, \{g_i, h_i\}_{i = 1}^3, \alpha ,\beta)$$
    and
    $$\sk_{sig} = (\vk_{sig}, \zeta, \rho, \phi, \{\chi_i, \gamma_i, \delta_i\}_{i = 1}^3).$$
  \item $\PK = (d,\pk_{enc}, \vk_{sig}, \sigma_{crs})$.
  \item $\SK = (x, y)$
  \end{enumerate}

\item[\boldmath$RCCA1.\Enc(\PK,m)$:] This algorithm takes as input a message and the public key of the underlying encryption scheme, outputs the corresponded ciphertext of our RCCA encryption scheme.
  \begin{enumerate}
  \item Choose two random exponents $(\theta_1, \theta_2) \sample \Z_p$ and compute $ \vec{r} = (\theta_1, \theta_2)$.
  \item Compute the ciphertext $\vec{C} = (C_1, C_2, C_3)$:
    \begin{align*}
      C_1 &= f^{\theta_1} & C_2 &= g^{\theta_2} & C_3 &= m \cdot h^{\theta_1+\theta_2}
    \end{align*}

  \item Recall that we want prove the knowledge of the witness $\vec{w} = (m, \vec{r}, \vec{D}, \vec{S}, \vec{\sigma})$ which verifies that
    \begin{align*}
      Enc_{BBS}(pk_{enc}, m; \vec{r}) = \vec{C} \vee (\vec{C} = ReRand(\vec{D}; \vec{S}) \wedge Verify(vk_{sig}, \vec{D}) = \True)
    \end{align*}
    
  \item Define the bit $b = 1$ and a Groth-Sahai commitment $\vec{C}_b = (1,1,d^b)\cdot \vec{g}_1^{r_b} \cdot \vec{g}_2^{s_b} \cdot \vec{g}_3^{t_b}$ and also a $NIWI$ proof $\pi_b \in \G^6$ of the pairing product equation 
  
  \begin{align}
  e(d,\boxed{d^b}) &= e(\boxed{d^b},\boxed{d^b}) \tag{1}
  \end{align}
  
  which ensures that $b \in \{0,1\}$.

    %  \item Then generate commitements $\{\vec{C}_{\Gamma_i}\}_{i = 1}^3$ of the variables $\Gamma_i = C_i^b$ for $i \in \{1,2,3\}$ and the corresponded proofs:
    %    \begin{align}
    %      e(C_i,\boxed{h^b}) &= e(h, \boxed{\Gamma_i}) ,&
    %      \forall i \in \{1,2,3\}
    %    \end{align}
  \item We first prove the left side of the OR statement, we generate commitments $(\vec{C}_{R_1}, \vec{C}_{R_2})$ of the variables $(R_1 = d^{\theta_1b}, R_2 = d^{\theta_2b})$ and $\vec{C}_{M}$ commitment of $M = m^b$ and commitments $\{\vec{C}_{\Delta_i}\}_{i=1}^3$ of the variables $\{\Delta_i = C_i^b\}_{i=1}^3$. Recall that $\{C_i\}_{i=1}^3$, $\{R_1,R_2\}$ and $\{\Delta_i\}_{i=1}^3$ verify the following equations:
    \begin{align}
      e(C_i,\boxed{d^b}) &= e(\boxed{\Delta_i}, d)  &\forall i \in \{1,2,3\} \tag{2,3,4}\\
      e(\boxed{\Delta_1},d) &= e(f, \boxed{R_1}) \tag{5}\\
      e(\boxed{\Delta_2},d) &= e(g, \boxed{R_2}) \tag{6}\\
      e(\boxed{\Delta_3},d) \cdot e(\boxed{M},d^{-1}) &= e(\boxed{R_1}, h) \cdot e(\boxed{R_2},h) \tag{7}
    \end{align}



    %signature part
  \item Then we prove the right side of the OR statement:
    \begin{enumerate}  
    \item The $ReRand$ component: define $(D_1, D_2, D_3) = (1_\G, 1_\G, 1_\G)$ and $(S_1,S_2) = (1_\G, 1_\G)$.
    \item Remind that actually these variables are of the following forms in the security proof, but in the case $b=1$ they all become $1_\G$.
      $$(D_1, D_2, D_3) = (f^{(\theta_1+\theta_1')\cdot (1-b)},g^{(\theta_2+\theta_2')\cdot (1-b)},m^{1-b} \cdot h^{(\theta_1+\theta_1'+\theta_2+\theta_2')\cdot (1-b)})$$
      and $$(S_1, S_2) = (d^{\theta_1'\cdot (1-b)}, d^{\theta_2'\cdot (1-b)})$$
    \item Then compute the commitments $\{\vec{C}_{D_i}\}_{i=1}^3$ of $\{D_i\}_{i= 1}^3$ and $(\vec{C}_{S_1},\vec{C}_{S_2})$ commitments of $S_1,S_2$.
    \item And also compute the proofs of following equations:
      \begin{align}
        e(C_1/\boxed{\Delta_1}, d) &= e (\boxed{D_1},d) \cdot e(f^{-1}, \boxed{S_1}) \tag{8}\\
        e(C_2/\boxed{\Delta_2}, d) &= e (\boxed{D_2},d) \cdot e(g^{-1}, \boxed{S_2}) \tag{9}\\
        e(C_3/\boxed{\Delta_3}, d) &= e (\boxed{D_3},d) \cdot e(\boxed{S_1},h^{-1}) \cdot e(\boxed{S_2},h^{-1}) \tag{10}
      \end{align}

    \item Then the signature component (Remind that $b = 1$): define
      $$\vec{\sigma}  = (\Sigma_1, \Sigma_2, \Sigma_3) = (z^{1-b},r^{1-b},u^{1-b}) = (1_\G, 1_\G, 1_\G),$$
      then compute their commitments $(\vec{C}_{\Sigma_1}, \vec{C}_{\Sigma_2}, \vec{C}_{\Sigma_3})$.
    \item We generate the proof of the following linear pairing equations:
      \begin{align} 
        e(\alpha, d/\boxed{d^b}) &= e(g_z, \boxed{\Sigma_1}) \cdot e(g_r, \boxed{\Sigma_2}) \cdot \prod_{i=1}^3 e(g_i, \boxed{D_i}) \tag{11}\\
        e(\beta, d/\boxed{d^b}) &= e(h_z, \boxed{\Sigma_1}) \cdot e(h_u, \boxed{\Sigma_3}) \cdot \prod_{i=1}^3 e(h_i, \boxed{D_i}) \tag{12}
      \end{align}

    \end{enumerate}

  \item To allow the re-randomization of the ciphertext, we need to compute the commitments $\vec{C}_F$, $\vec{C}_G$ to the variables :

    \begin{align*}
    H &= h^b & F &= f^b, & G&=g^b
    \end{align*}

    and their corresponding proofs:
    \begin{align}
      e(\boxed{d^b}, h) &= e(\boxed{H},d) & e(\boxed{F},d) &= e(f,\boxed{d^b}) & e(\boxed{G}, d) &= e(g, \boxed{d^b})\tag{13, 14, 15}
    \end{align}

    
  \item We put all these proofs together to get $\vec{\pi}$.
  \item The ciphertext of the RCCA-scheme is
    $$(\vec{C} = (C_1, C_2, C_3), \vec{C}_{H}, \vec{C}_{d^b}, \vec{C}_{M}, \{\vec{C}_{R_1}\}_{i= 1}^2, \{\vec{C}_{D_i}\}_{i = 1}^3, \{\vec{C}_{S_i}\}_{i = 1}^2, \{\vec{C}_{\Sigma_i}\}_{i = 1}^3,\{\vec{C}_{\Delta_i}\}_{i=1}^3, \vec{C}_F, \vec{C}_G, \vec{\pi})$$
    
  \end{enumerate}

\item[\boldmath{$RCCA1.\Dec(\PK,\SK, \vec{C})$}]:
  \begin{enumerate}
  \item Parse $\vec{C}$ as $(C_1, C_2, C_3), \vec{C}_{H}, \vec{C}_{d^b}, \vec{C}_{M}, \{\vec{C}_{R_1}\}_{i= 1}^2, \{\vec{C}_{D_i}\}_{i = 1}^3, \{\vec{C}_{S_i}\}_{i = 1}^2, \{\vec{C}_{\Sigma_i}\}_{i = 1}^3,\{\vec{C}_{\Delta_i}\}_{i=1}^3, \vec{C}_F, \vec{C}_G, \vec{\pi})$.
  \item Parse $\SK$ as $(x,y)$
  \item Verify that all proofs are correct.
  \item If any proof fails then return $\bot$. otherwise return $C_3/(C_1 \cdot C_2)$ 
  \end{enumerate}
  
\item[\boldmath{$RCCA1.\Rerand(\PK, C)$}]:
  For the randomization, we will proceed in two stages. Firstly we sample two random values $(\theta_1', \theta_2') \gets \mathbb{Z}_p$. The new variables are $C_1' = C_1 \cdot f^{\theta_1'}$, $C_2' = C_2 \cdot g^{\theta_2'}$ and $C_3' = C_3 \cdot h^{\theta_1'+\theta_2'}$. Then using the proof of the equations $(13, 14, 15)$, to adapt the new proofs corresponding to the new ciphertext instance $\vec{C}' = (C_1', C_2', C_3')$.

%  For the randomization we first generate randomness for each commitment. For a variable $X$, we generate the new randomness $(\tilde{r}_X, \tilde{s}_X, \tilde{t}_X)$, the new commitment will be $\tilde{\vec{C}}_X = \iota(X) \cdot \vec{g}_1^{r_X+\tilde{r}_X} \cdot \vec{g}_2^{s_X+\tilde{s}_X} \cdot \vec{g}_3^{t_X+\tilde{t}_X}$.

For the second stage, we randomize all the commitments and the GS proofs without changing the ciphertext part $\vec{C}' = (C_1', C_2', C_3')$.

In this algorithm, for the variable $X$, we denote its commitment by $\vec{C}_X = (1, 1, X) \cdot \vec{g}_1^{r_X} \cdot \vec{g}_2^{s_X} \cdot \vec{g}_3^{t_X}$ its new commitment from the first stage by $\vec{C}_X' = \vec{C}_X \cdot \vec{g}_1^{r'_X} \cdot \vec{g}_2^{s'_X} \cdot \vec{g}_3^{t'_X}$ and denote the new randomness introduced in the second step by $(\tilde{r}_X, \tilde{s}_X, \tilde{t}_X)$.


The detailed proof elements are in the appendix~\ref{Rerandomization}. With this instantiation, the ciphertext of the RCCA encryption scheme
 $$(\vec{C} = (C_1, C_2, C_3), \vec{C}_{H}, \vec{C}_{d^b}, \vec{C}_{M}, \{\vec{C}_{R_1}\}_{i= 1}^2, \{\vec{C}_{D_i}\}_{i = 1}^3, \{\vec{C}_{S_i}\}_{i = 1}^2, \{\vec{C}_{\Sigma_i}\}_{i = 1}^3,\{\vec{C}_{\Delta_i}\}_{i=1}^3, \vec{C}_F, \vec{C}_G, \vec{\pi})$$
has $93\G$ group elements.


\end{description}


%\input{proofs}


\section{Using the structure-preserving commitment scheme to construct efficient structure preserving publicly verifiable CCA-2 encryption scheme}
In this section, we use the previous trapdoor commitment scheme to commit the verification key, as they are constructed to verify so called CMTCR (Chosen-Message Target Collision Resistant) property, this will leads us to a wanted CCA-2 encryption scheme.
\begin{description}

\item[\boldmath{$SPCCA.\KeyGen(1^\lambda)$}]:
  \begin{enumerate}
  \item Choose a asymmetric pairing group system $(\G, \hat{\G}, \G_T)$, groups of prime order $p > 2^\lambda$.
  \item Set $\PPP$ as $(\G, \hat{\G}, \G_T)$.
  \item Choose also group generators $g_1, g_2 \sample \G$ and random values $x_1, x_2 \sample \mathbb{Z}_p$.
  \item Generate group generator $\hat{g} \sample \hat{\G}$
  \item Set $X = g_1^{x_1}g_2^{x_2}$.
  \item Choose random values $\rho_u,\rho_u' \sample \Z_p$ and random group generators $(\hat{g}, \hat{h}) \sample \hat{\G}^2$.
  \item Set $(\vec{u}_1, \vec{u}_2)$ as $\vec{u}_1 = (\hat{g}, \hat{h}) \in \hat{\G}^2$ and $\vec{u}_2 =  (\hat{g}^{\rho_u}, \hat{h}^{\rho_u'}) \in \hat{\G}^2$. Note that $\vec{u}_1$ and $\vec{u}_2$ are linearly independent with overwhelming probability.
  \item Set $\PPP_{TC} = (\PPP, g_1, \hat{g}, \ell = 6)$.
  \item Generate the commitment key $\vec{\ck} \in \hat{\G}^8$ and $\vec{\tk} \in \mathbb{Z}_p^8$
  \item Define $\SK = (x_1, x_2)$ and $\PK = (g_1, g_2, \vec{u}_1, \vec{u}_2, X, \PPP_{TC}, \vec{\ck})$
  \end{enumerate}
\item[\boldmath{$SPCCA.\Enc(M,\PK)$}]:
  \begin{enumerate}
  \item Generate the one-time signature keys $(\SSK, \SVK) \gets OT1.\KeyGen(\PPP)$ with $\SSK = (\{\chi_i, \gamma_i\}_{i=1}^5, \zeta, \rho) \in \mathbb{Z}_p^{12}$ and $\SVK =  (\{\hat{g}_i\}_{i =1}^5,  \hat{A} ) \in \hat{\G}^6$.
  \item Choose $\theta \sample \mathbb{Z}_p$ and compute
    \begin{align*}
      C_0 &= M\cdot X^{\theta}, & C_1 &= g_1^{\theta}, & C_2 &= g_2^{\theta}.
    \end{align*}
  \item Generate a commitment to $\SVK = (\{\hat{g}_i\}_{i =1}^5, \hat{a})$ and let 
    $$(\com, \open) \gets TC.\Com(\PPP_{TC}, \vec{ck}, \SVK) \in \hat{\G} \times (\G^9 \times \hat{\G}^2)$$
    be the resulting commitment/decommitment pair.
  \item Define vector $\vec{u}_{\com} = \vec{u}_2\cdot (1, \com)$ and the Groth-Sahai CRS $\mathbf{u}_{\com}=(\vec{u}_{\com},\vec{u}_1)$. 
  \item Pick $r \sample \mathbb{Z}_p$. Compute $\vec{C}_{\theta} = \vec{u}_{\com}^{\theta} \cdot (\vec{u}_1)^r$.
  \item Using the randomness of the commitment $\vec{C}_{\theta}$, generate  proof elements $\vec{\pi}=(\pi_1,\pi_2)=(g_1^r,g_2^r) \in \G^2$ showing that the committed $\theta \in \Z_p$ satisfies the multi-exponentiation equations
    \begin{align*}
      C_1 &= g_1^{\theta} & C_2 &= g_2^{\theta}
    \end{align*}
  \item Output the ciphertext
    \begin{align*}
      \vec{C} = (\SVK, \com, \open, C_0, C_1, C_2, \vec{C}_{\theta}, \vec{\pi}, \vec{\sigma}) \in \G^{16} \times \hat{\G}^{11}
    \end{align*}
    
    
    in which $\vec{\sigma} = OT1.\Sig(\SSK, (C_0, C_1, C_2, {\pi}_1,\pi_2)) \in \G^2$.

    Notice that we don't sign the commitments because in the Groth-Sahai proof system and in this very special case, there is only one valid commitment for given proofs.
    
  \end{enumerate}
  
\item[\boldmath{$SPCCA.\Dec(\PK, \vec{C}, \SK)$}]:
  \begin{enumerate}
  \item Parse $\PK$ with $(\vec{g}_1, \vec{g}_2, X, \PPP_{TC}, \ck)$ and $\SK$ with $(x_1, x_2)$.
  \item Parse $\vec{C}$ with $ (\SVK, \com, \open, C_0, C_1, C_2, \vec{C}_{\theta}, \vec{\pi}, \vec{\sigma})$.
  \item Verify the signature is valid $OT1.\Verif(\PPP,  (C_0, C_1, C_2, \pi_1,\pi_2), \sigma) = \True$.
  \item Using the commitment verification algorithm to verify that $TC.\Verif(\ck, \com, \SVK, \open) = \True$
  \item Verify that $\vec{\pi}=(\pi_1,\pi_2)$ is a valid Groth-Sahai proof \wrt  $(C_1, C_2, \vec{C}_{\theta}, \com)$. Namely, 
    it should satisfy 
    \begin{eqnarray} \label{ver-eq} 
      E(g_1,\vec{C}_{\theta}) &=& E(C_1 , \vec{u}_{\com}) \cdot E(\pi_1,\vec{u}_1) \\ \nonumber
      E(g_2,\vec{C}_{\theta}) &=& E(C_2 , \vec{u}_{\com}) \cdot E(\pi_2,\vec{u}_1)
    \end{eqnarray}
  \item If any of the verification fails then halt and return $\bot$, otherwise, output $M=C_0/(C_1^{x_1}\cdot C_2^{x_2})$.
  \end{enumerate}
  
\end{description}


\begin{myTh}
  The scheme provides IND-CCA2 security under the SXDH assumption.
\end{myTh}
\iffalse


\begin{proof}
  The proof proceeds with a sequence of games that begins with the real game and ends with a game where no advantage is left to the adversary. In each game, we call $W_i$ the event that the experiment outputs $1$. 


  \begin{description}
  \item[\textsf{Game} $0$:] This is the real game. The adversary is given the public key $\PK$ which contains 
    vectors 
    $(\vec{u}_1, \vec{u}_2)$ such that
    \begin{eqnarray} \label{vec-PK} 
      \vec{u}_1 &=& (\hat{g}, \hat{h}) \in \hat{\G}^2  \\ \nonumber 
      \vec{u}_2 &=&  (\hat{g}^{\rho_u}, \hat{h}^{\rho_u'}) \in \hat{\G}^2,
    \end{eqnarray}
    where $\hat{g},\hat{h} \sample \hat{\G}$,  $\rho_u,\rho_u' \sample \Z_p$.
    In the challenge phase, 
    it chooses two messages $M_0,M_1 \in \G$ and obtains a challenge ciphertexts 
    $$  \vec{C}^\star = (\SVK^\star, \com^\star, \open^\star, C_0^\star, C_1^\star, C_2^\star, \vec{C}_{\theta}^\star, \vec{\pi}^\star, \vec{\sigma}^\star)  $$
    where 
    \begin{align*}
      C_0^\star &= M_{\beta} \cdot X^{\theta^\star}, & C_1^\star &= g_1^{\theta^\star}, & C_2^\star &= g_2^{\theta^\star},
    \end{align*}
    for some random bit $\beta \sample \{0,1\}$, and 
    \begin{eqnarray*}
      \com^\star  &=& \hat{C}^\star  = \hat{g}^{\zeta_2^\star} \cdot \prod_{i = 1}^{\ell }\hat{X}_i^{\chi_i^\star } \cdot \hat{X}_{\ell+1}^{w_z^\star} \cdot \hat{X}_{\ell+2}^{a^\star}  \\
      \open^\star &=& \big( D^\star , g_z^\star, g_1^\star, \dots, g_\ell^\star, A^\star  , \hat{Z}^\star, \hat{R}^\star \big)  \\ 
      & =& \big(  g^{\zeta_2^\star },~ g^{w_z^\star}, ~ g^{\chi_1^\star}, ~\ldots , ~ g^{\chi_\ell^\star} , ~ g^{\chi_z^\star} , ~ g^{a^\star}, ~ 		 
      \hat{g}^{\zeta_1^\star} ,  ~\hat{g}^{a^\star-\zeta_1^\star w_z^\star }  \cdot  \prod_{i=1}^{6} {\hat{M}_i^\star~ }^{\chi_i^\star} 
      \big), \\
      \vec{C}_{\theta}^\star &=& \vec{u}_{\com^\star}^{\theta^\star} \cdot (\vec{u}_1)^{r^\star}\\
      \vec{\pi}^\star &=& (\pi_1^\star,\pi_2^\star) =(g_1^{r^\star},g_2^{r^\star}) 
    \end{eqnarray*}
    with $(\hat{M}_1^\star,\ldots,\hat{M}_\ell^\star)=(\hat{g}_1^\star,\ldots,\hat{g}_5^\star,\hat{A}^\star) $ and 
    $\vec{u}_{\com^\star} = \vec{u}_2\cdot (1, \com^\star)$.
    \indent The adversary's decryption queries are always faithfully answered by the challenger. When the adversary halts, it outputs   
    $\beta' \in \{0,1\}$ and wins if $\beta' =\beta$. In this case, the experiment outputs $1$. Otherwise, it outputs $0$.  
    The adversary's advantage is thus $|\Pr[W_0]-1/2|$. \smallskip \smallskip 

  \item[\textsf{Game} $1$:] In this game, we modify the generation of the public key and define
    \begin{eqnarray} \label{vec-PK-sim} 
      \vec{u}_1 &=& (\hat{g}, \hat{h}) \in \hat{\G}^2  \\ \nonumber 
      \vec{u}_2 &=&  (\hat{g}^{\rho_u}, \hat{h}^{\rho_u'}) \cdot (1,{\hat{C}^\star ~}^{-1})  \in \hat{\G}^2.
    \end{eqnarray}
    instead of computing $(\vec{u}_1,\vec{u}_2)$ as in (\ref{vec-PK}) (note that we may assume w.l.o.g. that $\SVK^\star$ and $\com^\star=\hat{C}^\star$ are 
    generated 
    at the outset of the game).  However, this modification does not affect the adversary's view since $\vec{u}_2$ remains uniformly distributed 
    over $\hat{\G}^2$. We have $\Pr[W_1]=\Pr[W_0]$.   \smallskip \smallskip 

  \item[\textsf{Game} $2$:] This game is like Game $1$ with the difference that, if the adversary makes a pre-challenge decryption query 
    $ \vec{C} = (\SVK, \com, \open, C_0, C_1, C_2, \vec{C}_{\theta}, \vec{\pi}, \vec{\sigma})  $ such that $\com=\com^\star$, the expermiment halts and 
    outputs a random bit. Since Game $2$ is identical to Game $1$ until this event $F_2$ occurs, we have 
    $|\Pr[W_2]-\Pr[W_1]| \leq \Pr[F_2]$. Moreover, since $\com^\star$ was chosen uniformly in $\hat{\G}$ and remains independent of $\A$'s view until 
    the challenge phase, we have $ |\Pr[W_2]-\Pr[W_1]| \leq \Pr[F_2] \leq q_D/p$. \smallskip \smallskip 


  \item[\textsf{Game} $3$:] This game is like Game $2$ but we modify the decryption oracle. Namely, if the adversary makes a post-challenge decryption query 
    for a valid ciphertext
    $$ \vec{C} = (\SVK, \com, \open, C_0, C_1, C_2, \vec{C}_{\theta}, \vec{\pi}, \vec{\sigma})  $$
    such that $\com=\com^\star$ but $\open \neq \open^\star$, the experiment halts and outputs a random bit. If we call $F_3$ the latter event, 
    we have $|\Pr[W_3]-\Pr[W_2]| \leq \Pr[F_3]$. Moreover, event $F_3$ clearly implies an adversary $\B$ against the target collision-resistance of the 
    structure-preserving trapdoor commitment in Section \ref{trap-com}, which contradicts the Double Pairing assumption. Hence, we have  
    $|\Pr[W_3]-\Pr[W_2]| \leq \mathbf{Adv}_\B^{\mathsf{TCR}\textsf{-}\mathsf{CR}}(\lambda) \leq   \mathbf{Adv}_\B^{\mathrm{DP}}(\lambda)$.   \smallskip \smallskip 


  \item[\textsf{Game} $4$:] We modify again the decryption oracle. After the phase, if the adversary queries the decryption of a ciphertext 
    $ \vec{C} = (\SVK, \com, \open, C_0, C_1, C_2, \vec{C}_{\theta}, \vec{\pi}, \vec{\sigma})  $ such that $(\com,\open)=(\com^\star,\open^\star)$
    but $(C_0,C_1,C_2,\pi_1,\pi_2) \neq (C_0^\star,C_1^\star,C_2^\star,\pi_1^\star,\pi_2^\star)$, the experiment halts and outputs a random 
    bit. If we call $F_4$ this event, we have the inequality  $|\Pr[W_4]-\Pr[W_3]| \leq \Pr[F_4]$ since Game $4$ is identical to Game $3$ until $F_4$ occurs.
    Moreover,   $F_4$ would contradict the strong unforgeability of the one-time structure-preserving signature  and thus the DP assumption. This implies  
    $|\Pr[W_4]-\Pr[W_3]| \leq \mathbf{Adv}_\B^{\mathsf{SUF}\textsf{-}\mathsf{OTS}}(\lambda)   \leq   \mathbf{Adv}_\B^{\mathrm{DP}}(\lambda)$.
    \smallskip \smallskip 

  \item[\textsf{Game} $5$:] We introduce yet another modification in the decryption oracle.  We let the decryption oracle reject all ciphertexts $ \vec{C} = (\SVK, \com, \open, C_0, C_1, C_2, \vec{C}_{\theta}, \vec{\pi}, \vec{\sigma})  $ 
    such that
    \begin{eqnarray} \label{event-F5}
      (\com,\open)=(\com^\star,\open^\star)  \quad \wedge \quad (C_0,C_1,C_2,\pi_1,\pi_2) = (C_0^\star,C_1^\star,C_2^\star,\pi_1^\star,\pi_2^\star) 
      \quad \wedge \quad \vec{C}_{\theta} \neq \vec{C}_{\theta}^\star .
    \end{eqnarray}  
    Let $F_5$ be the event that the decryption oracle rejects a ciphertext that would not have been rejected in Game $4$.   
    We argue that $\Pr[W_5] = \Pr[W_4]$ since Game $5$ is identical to Game $4$ until event $F_5$ occurs and we have $\Pr[F_5]=0$. 
    Indeed, 
    %since the vectors $\vec{u}_{\com^\star}$ and $\vec{u}_1$ are linearly independent, they span the entire vector space $\hat{\G}^2$ which 
    %means that, 
    for a given $(C_1^\star,C_2^\star,\pi_1^\star,\pi_2^\star) \in \G^4$, there exists only one commitment $\vec{C}_{\theta}^\star \in \hat{\G}^2$ that satisfies the equalities (\ref{ver-eq}). This follows from the fact that, since  
    $(C_1^\star,C_2^\star,\pi_1^\star,\pi_2^\star)=(g_1^{\theta^\star},g_2^{\theta^\star},g_1^{r^\star},g_2^{r^\star})$, relations 
    (\ref{ver-eq}) can be written
    \begin{eqnarray*}  
      E(g_1,\vec{C}_{\theta}^\star) &=& E(g_1^{\theta^\star} , \vec{u}_{\com}) \cdot E(g_1^{r^\star},\vec{u}_1) = E(g_1 , \vec{u}_{\com}^{\theta^\star}) \cdot E(g_1,\vec{u}_1^{r^\star}) \\ \nonumber
      E(g_2,\vec{C}_{\theta}^\star) &=& E(g_2^{\theta^\star} , \vec{u}_{\com}) \cdot E(g_2^{r^\star},\vec{u}_1) =E(g_2 , \vec{u}_{\com}^{\theta^\star}) \cdot E(g_2,\vec{u}_1^{r^\star})
    \end{eqnarray*}
    which uniquely determines the  only commitment $\vec{C}_{\theta}^\star=\vec{u}_{\com}^{\theta^\star} \cdot \vec{u}_1^{r^\star} \in \hat{\G}^2$ that satisfies (\ref{ver-eq}). 
    This shows that $\Pr[F_5] = 0$, as claimed. 
    \smallskip \smallskip 





  \item[\textsf{Game} $6$:] In this game, we modify the distribution of the public key.  Namely, instead of generating 
    the vectors $(\vec{u}_1,\vec{u}_2)$ as in (\ref{vec-PK-sim}), we 
    set 
    \begin{eqnarray} \label{vec-PK-sim-bis} 
      \vec{u}_1 &=& (\hat{g}, \hat{h}) \in \hat{\G}^2  \\ \nonumber 
      \vec{u}_2 &=&  (\hat{g}^{\rho_u}, \hat{h}^{\rho_u}) \cdot (1,{\hat{C}^\star~}^{-1})  \in \hat{\G}^2.
    \end{eqnarray}
    Said otherwise,  $\vec{u}_2$ is now the product of two terms, the first one of which lives in the 
    one-dimensional subspace spanned by $\vec{u}_1$. Under the DDH assumption in $\hat{\G}$, this modified  
    distribution of $\PK$ should have not noticeable impact on the adversary's behavior.  
    A straightforward reduction shows 
    that $|\Pr[W_6]-\Pr[W_5] | \leq \mathbf{Adv}^{\mathrm{DDH}}_\B (\lambda)$. Note that, although the vectors $(\vec{u}_{\com^\star},\vec{u}_1) \in \hat{\G}^2$ are 
    no longer linearly independent, $\vec{C}_{\theta}^\star = \vec{u}_1^{\rho_u \cdot \theta^\star +r^\star}$ remains the only commitment 
    that satisfies the verification equations  for a given tuple $(C_1^\star,C_2^\star,\pi_1^\star,\pi_2^\star)$.
    

    \smallskip \smallskip 

  \item[\textsf{Game} $7$:] In this game, we  modify the challenge ciphertext and replace the NIZK proof $\vec{\pi}^\star=(\pi_1^\star,\pi_2^\star) \in \G^2$ by a simulated proof which is produced 
    using $\rho_u \in \Z_p$ as a simulation trapdoor. Namely, $(\vec{C}_\theta^\star,\vec{\pi}^\star)$ is obtained by picking $r \sample \Z_p$ and  computing
    \begin{eqnarray*}
      \vec{C}_{\theta}^\star &=& \vec{u}_1^{r},   \qquad \qquad \quad
      \pi_1^\star  =  g_1^{r} \cdot {C_1^\star }^{-\rho_u} , \qquad \qquad \quad 
      \pi_2^\star  =  g_2^{r} \cdot {C_2^\star }^{-\rho_u}
    \end{eqnarray*}
    Observe that, although $(\vec{C}_\theta^\star,\pi_1^\star,\pi_2^\star)$ are generated without using the witness $\theta^\star = \log_{g_1}(C_1^\star) =
    \log_{g_2}(C_2^\star)$,  the NIZK property of 
    GS proofs ensures that 
    their distribution remains exactly as in Game $6$: indeed, if we define $\tilde{r} =r -\rho_u \cdot \theta^\star$, we have
    \begin{eqnarray*}
      \vec{C}_{\theta}^\star &=& \vec{u}_{\com^\star}^{\theta^\star} \cdot \vec{u}_1^{\tilde{r}},   \qquad \qquad \quad
      \pi_1^\star  =  g_1^{\tilde{r}} , \qquad \qquad \quad 
      \pi_2^\star  =  g_2^{\tilde{r}} ,
    \end{eqnarray*}
    which implies $\Pr[W_7]=\Pr[W_6]$.  
    \smallskip \smallskip


  \item[\textsf{Game} $8$:]  We modify the generation of the challenge ciphertext, which is generated using the private key $\SK=(x_1,x_2)$ instead
    of the public key: Namely, the challenger computes 
    \begin{align*}
      C_1^\star &= g_1^{\theta^\star}, & C_2^\star &= g_2^{\theta^\star},    &  C_0^\star &= M_{\beta} \cdot {C_1^\star}^{x_1} \cdot  {C_2^\star}^{x_2} , 
    \end{align*} 
    while $(\vec{C}_\theta^\star,\pi_1^\star,\pi_2^\star)$ are computed using the NIZK simulation trapdoor $\rho_u \in \Z_p$ as in Game $7$. 
    This modification does not affect the adversary's view since the ciphertext retains exactly the same distribution as in Game $7$. 
    We have $\Pr[W_8]=\Pr[W_7]$.  \smallskip \smallskip

  \item[\textsf{Game} $9$:] We modify again the distribution of the challenge ciphertext which is obtained as 
    \begin{align*}
      C_1^\star &= g_1^{\theta_1^\star}, & C_2^\star &= g_2^{\theta_2^\star},    &  C_0^\star &= M_{\beta} \cdot {C_1^\star}^{x_1} \cdot  {C_2^\star}^{x_2} , 
    \end{align*} 
    for random and independent $\theta_1^\star,\theta_2^\star \sample \Z_p$, 
    while the NIZK proof $(\vec{C}_\theta^\star,\pi_1^\star,\pi_2^\star)$ is simulated using $\rho_u \in \Z_p$ as in Game $8$.  Since 
    the witness $\theta^\star \in \Z_p$ was not used anymore in Game $8$, a straightforward reduction shows that  any noticeable change in $\A$'s output distribution implies a DDH distinguisher in $\G$. We have 
    $|\Pr[W_9]-\Pr[W_8]| \leq   \mathbf{Adv}_{\B,\G}^{\mathrm{DDH}}(\lambda)$. \medskip 

    We remark that, although we now have  
    $\log_{g_1}(C_1^\star) \neq \log_{g_2}(C_2^\star)$ with overwhelming probability, the signed ciphertext components $(C_1^\star,C_2^\star,\pi_1^\star,\pi_2^\star)$ 
    still uniquely determine $\vec{C}_{\theta}^\star \in \hat{\G}^2$ as the only commitment that satisfies the verification equations of 
    $(\vec{C}_{\theta}^\star,\pi_1^\star,\pi_2^\star)$:
    indeed, the equalities 
    \begin{eqnarray*}
      E(g_1,\vec{C}_\theta^\star) &=& E(C_1^\star, \vec{u}_{\com^\star}) \cdot E(\pi_1^\star ,\vec{u}_1)  \\ 
      E(g_2,\vec{C}_\theta^\star) &=& E(C_2^\star, \vec{u}_{\com^\star}) \cdot E(\pi_2^\star ,\vec{u}_1) 
    \end{eqnarray*} 
    can be written 
    \begin{eqnarray*}
      E(g_1,\vec{C}_\theta^\star) &=& E(C_1^\star , \vec{u}_1^{\rho_u} ) 
      \cdot E(  g_1^{r} \cdot {C_1^\star }^{-\rho_u} , \vec{u}_1)  = E(g_1^r,\vec{u}_1)  \\
      E(g_2,\vec{C}_\theta^\star) &=& E(C_2^\star , \vec{u}_1^{\rho_u} ) 
      \cdot E(  g_2^{r} \cdot {C_2^\star }^{-\rho_u} , \vec{u}_1) =E(g_2^r,\vec{u}_1),
    \end{eqnarray*}
    which implies that $\vec{C}_{\theta}^\star = \vec{u}_1^r$ is the only commitment satisfying (\ref{ver-eq}). Hence, at any step of the sequence of 
    games, $\vec{C}_\theta^\star$ 
    is always uniquely determined by $(C_1^\star,C_2^\star,\pi_1^\star,\pi_2^\star)$ and does not have to be signed. 
    \smallskip \smallskip

    

  \end{description}

  In the final game, it is easy to see that $\Pr[W_9]=1/2$ since the challenge ciphertext does not carry any information about $\beta \in \{0,1\}$. 
  Indeed, we have 
  \begin{align*}
    C_1^\star &= g_1^{\theta_1^\star}, & C_2^\star &= g_2^{\theta_1^\star + \theta_1'},    &  C_0^\star &= M_{\beta} \cdot X^{\theta_1^\star} 
    \cdot  {g_2}^{\theta_1' \cdot x_2} , 
  \end{align*} 
  for some random $\theta_1' \in_R \Z_p$, which implies that 		the term ${g_2}^{\theta_1' \cdot x_2}$ perfectly hides $M_\beta$ in the expression of $C_0^\star$. This follows from the fact
  that $x_2 \in \Z_p$ is perfectly independent of the adversary's view.	Indeed, the public key leaves $x_2 \in \Z_p$ completely undetermined as it only reveals $X=g_1^{x_1} g_2^{x_2}$. During the game, decryption queries are guaranteed not to reveal anything about $x_2$ since all 
  NIZK proofs $(\vec{C}_{\theta},\pi_1,\pi_2)$ take place on   Groth-Sahai CRSes $(\vec{u}_{\com},\vec{u}_1)$ which are perfectly sound (as 
  they span the entire vector space $\hat{\G}^2$)
  whenever $\com \neq \com^\star$. This implies that, although the adversary can see a simulated NIZK proof $(\vec{C}_\theta^\star,\pi_1^\star,\pi_2^\star)$ for 
  a false statement in the challenge phase, it remains  unable to trick the decryption oracle into accepting a  ciphertext 
  $ \vec{C} = (\SVK, \com, \open, C_0, C_1, C_2, \vec{C}_{\theta}, \vec{\pi}, \vec{\sigma})  $ such that $\log_{g_1}(C_1) \neq \log_{g_2}(C_2)$. 
  As a consequence, the adversary does not learn anything about $x_2$ from responses of the decryption oracle. 
  
\end{proof}

\fi








\section{Combine Structure Perserving Public Encryption scheme with LHSPS and WI-NIZK proof}
\begin{description}
\item[\boldmath{$RCCA2.\KeyGen(1^\lambda)$}]:
  \begin{enumerate}
  \item Choose a asymmetric pairing group system $(\G, \hat{\G}, \G_T)$, groups of prime order $p > 2^\lambda$.
  \item Set $\PPP$ as $(\G, \hat{\G}, \G_T)$.
  \item Choose also group generators $g_1, g_2 \sample \G$ and random values $x_1, x_2 \sample \mathbb{Z}_p$.
  \item Generate group generator $\hat{g} \sample \hat{\G}$
  \item Set $X = g_1^{x_1}g_2^{x_2}$.
  \item Generate random values $(\rho_u, \rho_u') \sample \mathbb{Z}_p^2$ and random group generators $(\hat{g}, \hat{h}) \sample \hat{\G}^2$.
  \item Set $(\vec{u}_1, \vec{u}_2)$ as $\vec{u}_1 = (\hat{g}, \hat{h}) \in \hat{\G}^2$ and $\vec{u}_2 = (\vec{u}_1)^{\rho_u} = (\hat{g}^{\rho_u}, \hat{h}^{\rho_u'}) \in \hat{\G}^2$. Note that $\vec{u}_1$ and $\vec{u}_2$ are linearly independent with overwhelming probability.
  \item $\PPP_{TC} = (\PPP, g_1, \hat{g}, \ell = 6)$.
  \item Generate the commitment key $\vec{\ck} \in \hat{\G}^8$ and $\vec{\tk} \in \mathbb{Z}_p^8$.
  \item Generate a pair of Groth-Sahai commitment parameters$(\PPP_{GS}) = (\vec{g}_r, \vec{g}_s)$.
  \item We output the public parameters $\PPP =  (g_1, g_2, \PPP_{TC}, \vec{\ck}, \PPP_{GS})$.
  \item $\SK = (x_1, x_2)$
  \item $\PK = (\vec{u}_1, \vec{u}_2, X)$
  \end{enumerate}
\item[\boldmath{$RCCA2.\Enc(\PPP,M,\PK)$}]:
  \begin{enumerate}
  \item Generate the LHSPS signature keys $(\SSK, \SVK) \gets LHSPS.\KeyGen(\PPP)$ with $\SSK = (\{\chi_i, \gamma_i\}_{i=1}^3) \in \mathbb{Z}_p^{8}$ and $\SVK = (\{\SVK_i\}_{i=1}^4) = (\{\hat{g}_i\}_{i =1}^4) \in \hat{\G}^4$.
  \item Choose $\theta \sample \mathbb{Z}_p$ and compute
    \begin{align*}
      C_0 &= M\cdot X^{\theta}, & C_1 &= g_1^{\theta}, & C_2 &= g_2^{\theta}.
    \end{align*}
  \item Generate commitment and open of the verification key $\SVK$, 
    $$(\com, \open) \gets TC.\Com(\PPP_{TC}, \vec{ck}, \SVK) \in \hat{\G} \times (\G^7 \times \hat{\G}^2)$$
  \item Construct the proof vector $\vec{u}_{\com} = \vec{u}_2\cdot (1, \com)$.
  \item Generate the commitment $\vec{C}_{\SVK}, \vec{C}_{\com}, \vec{C}_{\open}$ of $\SVK, \com, \open$ with respect to $\PPP_{GS}$.
  \item Generate $(r,s) \sample \mathbb{Z}_p$. Compute $\vec{C}_{\theta} = \vec{u}_{\com}^{\theta} \cdot (\vec{u}_1)^r$ and $\vec{C}_{1} = \vec{u}_{\com} \cdot (\vec{u}_1)^s$.
  \item Note that we have the commitment of $\com$ is $\vec{C}_{\com} = (1, \com) \cdot \vec{g}_r^{r_{\com}} \cdot \vec{g}_s^{s_{\com}}$.
  \item Using $\vec{C}_{\theta}$ to get two GS proofs $(\vec{\pi}_{\theta}, \vec{\pi}_1)$:

    \begin{align*}
      \vec{\pi}_{\theta} &= (\vec{\pi}_{\theta,1}, \vec{\pi}_{\theta,2})\\
      &= ((\pi_{\theta,1,1}, \pi_{\theta,1,2}, \pi_{\theta,1,3}), (\pi_{\theta,2,1}, \pi_{\theta,2,2}, \pi_{\theta,2,3}))\\
      &= ((g_1^r, C_1^{r_{\com}}, C_1^{s_{\com}}),(g_2^r, C_2^{r_{\com}}, C_2^{s_{\com}}))\\     
      \vec{\pi}_{1} &= (\pi_{1,1,1}, \pi_{1,1,2})\\
      &= ((\pi_{1,1,1}, \pi_{1,1,2}, \pi_{1,1,3}), (\pi_{1,2,1}, \pi_{1,2,2}, \pi_{1,2,3}))\\
      &= ((g_1^s, g_1^{r_{\com}}, g_1^{s_{\com}}),(g_2^s, g_2^{r_{\com}}, g_2^{s_{\com}}))
    \end{align*}

    of
    
    \begin{align*}
      C_1 &= g_1^{\theta} & C_2 &= g_2^{\theta}
    \end{align*}

    which verifies:
    \begin{align*}
      E(g_1, \vec{C}_{\theta}) &= E(C_1, \vec{u}_2) \cdot E(C_1, \boxed{\vec{C}_{\com}}) \cdot E(\pi_{\theta,1,1}, \vec{u}_1) \cdot E(\pi_{\theta,1,2}, \vec{g}_r) \cdot E(\pi_{\theta,1,3}, \vec{g}_s)\\
      E(g_2, \vec{C}_{\theta}) &= E(C_2, \vec{u}_2) \cdot E(C_2, \boxed{\vec{C}_{\com}}) \cdot E(\pi_{\theta,2,1}, \vec{u}_1) \cdot E(\pi_{\theta,2,2}, \vec{g}_r) \cdot E(\pi_{\theta,2,3}, \vec{g}_s)\\
      E(g_1, \vec{C}_{1}) &= E(g_1, \vec{u}_2) \cdot E(g_1, \boxed{\vec{C}_{\com}}) \cdot E(\pi_{1,1,1}, \vec{u}_1) \cdot E(\pi_{1,1,2}, \vec{g}_r) \cdot E(\pi_{1,1,3}, \vec{g}_s)\\
      E(g_2, \vec{C}_{1}) &= E(g_2, \vec{u}_2) \cdot E(g_2, \boxed{\vec{C}_{\com}}) \cdot E(\pi_{1,2,1}, \vec{u}_1) \cdot E(\pi_{1,2,2}, \vec{g}_r) \cdot E(\pi_{1,2,3}, \vec{g}_s)
    \end{align*}
  \item We also compute the signature $\vec{\sigma}_1 = (z_1, r_1)$ of the vector $(g, X, g_1, g_2)$ and the signature $\vec{\sigma}_m = (z_m, r_m)$ of the vector $(1, C_0, C_1, C_2)$.
  \item Generate the proof $(\vec{\pi}_{\com}, \vec{\pi}_{\vec{\sigma}_1}, \vec{\pi}_{\vec{\sigma}_m})$ of the following equations:
    \begin{align*}
      TC.\Verif(\ck, \com, \SVK, \open) &= \True\\
      LHSPS.\Verif(\SVK, (g, X, g_1, g_2), \vec{\sigma}_1) &= \True\\
      LHSPS.\Verif(\SVK, (1, C_0, C_1, C_2), \vec{\sigma}_m) &= \True\\
    \end{align*}
    
    More explicitly, We parse $\open$ as $(D, g_z, g_1, g_2, g_3, g_4, \ovk_{POS}, \hat{Z}, \hat{R}) \in \G^7 \times \hat{\G}^2$, then we have $\vec{C}_{\open} \in \G^{14} \times \hat{\G}^4$, then we proof the following equations:
    \begin{align*}
      e(g, \boxed{\hat{C}}) &= e(\boxed{D}, \hat{g}) \prod_{i = 1}^{4} e(\boxed{\SVK_i}, \boxed{\hat{X}_i}) \cdot e(\boxed{g_z}, \boxed{\hat{X}_5}) \cdot e(\boxed{\ovk_{POS}}, \boxed{\hat{X}_6}) \\
      e(\boxed{\ovk_{POS}}, \hat{g}) &= e(\boxed{g_z}, \boxed{\hat{Z}}) \cdot e(g, \boxed{\hat{R}}) \cdot \prod_{i = 1}^4 e(\boxed{g_i}, \boxed{\SVK_i}) 
    \end{align*}
    Then we have $\pi_{\com} \in \G^2 \times \hat{\G}^2$.

  \item Then we prove the following equations:
    \begin{align}
      e(z_1, \hat{g}_z) \cdot e(r_1, \hat{g}_r) &= e(g, \boxed{\SVK_1}) \cdot e(X, \boxed{\SVK_2}) \cdot e(g_0, \boxed{\SVK_3}) \cdot e(g_1, \boxed{\SVK_4}) \label{z1}\\
      e(z_m, \hat{g}_z) \cdot e(r_m, \hat{g}_r) &= e(1, \boxed{\SVK_1}) \cdot e(C_0, \boxed{\SVK_2}) \cdot e(C_1, \boxed{\SVK_3}) \cdot e(C_2, \boxed{\SVK_4}) \label{zm}
    \end{align}

    Thus we have $(\vec{\pi}_{\vec{\sigma}_1}, \vec{\pi}_{\vec{\sigma}_m}) \in \G^2 \times \G^2$
  
  \item Output the ciphertext
    \begin{align*}
      \vec{C} = (C_0, C_1, C_2, \vec{C}_{\theta}, \vec{\pi}_{\theta}, \vec{\pi}_{1}, \vec{\sigma}_1, \vec{\sigma}_m, \vec{C}_{\SVK}, \vec{C}_{\com}, \vec{C}_1, \vec{C}_{\open},  \vec{\pi}_{\com}, \vec{\pi}_{\vec{\sigma}_1}, \vec{\pi}_{\vec{\sigma}_m}, \vec{\pi}_{\theta}) \in \G^{39} \times \hat{\G}^{20}
    \end{align*}
    

    
  \end{enumerate}
  
\item[\boldmath{$RCCA2.\Dec(\PK, \vec{C}, \SK)$}]:
  \begin{enumerate}
  \item Parse $\PK$ with $(\vec{g}_1, \vec{g}_2, X, \PPP_{TC}, \ck)$ and $\SK$ with $(x_1, x_2)$.
  \item Parse $\vec{C}$ with $\vec{C} = (C_0, C_1, C_2, \vec{C}_{\theta}, \vec{\pi}, \vec{\sigma}_1, \vec{\sigma}_m, \vec{C}_{\SVK}, \vec{C}_{\com}, \vec{C}_{\open},  \vec{\pi}_{\com}, \vec{\pi}_{\vec{\sigma}_1}, \vec{\pi}_{\vec{\sigma}_m})$.
  \item Verify that $\vec{\pi}_{\vec{\sigma}_1}$ is a valid Groth-Sahai proof \wrt the commitments $(\vec{C}_{\SVK})$ for the equation \ref{z1}.
  \item Verify that $\vec{\pi}_{\vec{\sigma}_m}$ is a valid Groth-Sahai proof \wrt the commitments $(\vec{C}_{\SVK})$ for the equation \ref{zm}.
  \item If any of the verification fails then halt and return $\bot$, otherwise, output $C_0/(C_1^{x_1}\cdot C_2^{x_2})$.
  \end{enumerate}


\item[\boldmath{$RCCA2.\Rerand$}]:
  \begin{enumerate}
  \item We choose randomly a value $r' \sample \mathbb{Z}_p$.
  \item We update $\vec{C}$ by $\vec{C}'  = (C_1 \cdot f^{r'}, C_2 \cdot g^{r'}, C_3 \cdot h^{r'})$.
  \item Compute the new signature $\vec{\sigma}_m' = \vec{\sigma}_m \cdot \vec{\sigma}_1^{r'}$
  \item Compute the new proof $\vec{\pi}_{\vec{\sigma}_m}' = \vec{\pi}_{\vec{\sigma}_m} \cdot \vec{\pi}_{\vec{\sigma}_1}^{r'}$
  \item Update the commitment $\vec{C}_{\theta}' = \vec{C}_{\theta} \cdot \vec{C}_{1}^{r'}$. We also update the proof $\vec{\pi}_{\theta}' = \vec{\pi}_{\theta} \cdot \vec{\pi}_{1}^{r'}$.
  \item Then we randomize all the commitments and GS proofs.
  \end{enumerate}
  
\end{description}


\begin{myTh}
  $RCCA2$ scheme is secure against RCCA security under SXDH assumption.
\end{myTh}

\begin{proof}

  This will be a game based proof. From the first game which is the definition of the RCCA security game to the last game, in which the adversary can trivially not have any advantage. In the $i$-th game, we define the advantage of the adversary by $S_i$.

  \begin{description}
  \item[\textsf{Game} $0$ :] This is the real game, the adversary is against the RCCA security game. We give the adversary the public key $\PK$ of the encryption scheme which contains the proof vectors $(\vec{u}_1, \vec{u}_2)$ which verifies:
    \begin{align*}
      \vec{u}_1 &= (\hat{g}, \hat{h}) \in \hat{\G}^2\\
      \vec{u}_2 &= (\hat{g}^{\rho_u}, \hat{h}^{\rho_u'}) \in \hat{\G}^2
    \end{align*}

    The adversary has the access to the decryption oracle.
    
    Then during the challenge phase, the adversary choose two messages $(m_0, m_1) \in \G^2$ submits to the Challenger and obtains a challenge ciphertext
    \begin{align*}
      \vec{C}^* = (C_0^*, C_1^*, C_2^*, \vec{C}_{\theta}^*, \vec{\pi}_{\theta}^*, \vec{\pi}_{1}^*, \vec{\sigma}_1^*, \vec{\sigma}_m^*, \vec{C}_{\SVK}^*, \vec{C}_{\com}^*, \vec{C}_1^*, \vec{C}_{\open}^*, \vec{\pi}_{\com}^*, \vec{\pi}_{\vec{\sigma}_1}^*, \vec{\pi}_{\vec{\sigma}_m}^*, \vec{\pi}_{\theta}^*)
    \end{align*}
    especially we have:
    \begin{align*}
      C_0^* &= m_b \cdot X^{\theta^*} & C_1^* &= g_1^{\theta^*} & C_2^* &= g_2^{\theta^*}.
    \end{align*}

    The adversary has the access to the decryption oracle expect the ciphertext $\vec{C}$ corresponded to the ciphertext $m_0$ or $m_1$.

    At the end, the adversary outputs a bit $b'$, it's advantage against the RCCA security game is defined by the wining probability of $S_0 = |\PR[b' = b] - \frac{1}{2}|$.

  \item[\textsf{Game} $1$ :] In this game, the challenger generate the signing key and verification key pair and the commitment $\com^*$ of the signature's verification key at the beginning of the game. Since this does not change the view of the adversary then we have $S_0 = S_1$.

  \item[\textsf{Game} $2$ :] In the $\KeyGen$ algorithm, we modify the generation of the public key. Instead of generate $(\vec{u}_1, \vec{u}_2)$ as in the \textsf{Game} 0, we define:
    \begin{align*}
      \vec{u}_1 &= (\hat{g}, \hat{h}) \in \hat{\G}^2\\
      \vec{u}_2 &= (\hat{g}^{\rho_u}, \hat{h}^{\rho_u'}) \cdot (1, (\com^*)^{-1}) \in \hat{\G}^2
    \end{align*}

    Since $\vec{u}_2$ always distributed uniformly over $\hat{\G}^2$, this change does not affect the view of the adversary. Thus we have $S_2 = S_1$.

  \item[\textsf{Game} $3$ :] In this game, we change the public parameter $\PPP_{GS}$ of the underlying Groth-Sahai proof system to the perfect binding setting. Due to the witness indistinguishable property, we have $|S_3 - S_2| \leq adv_{\mathcal{B}}^{wit-IND}(\lambda)$.

  \item[\textsf{Game} $4$ :] In this game, we define a failure event $F_4$: the ciphertext submitted by the adversary to the decryption oracle during the first query phase(before the challenge phase) contains the commitment $\vec{C}_{\com}$ which verifies $\com = \com^*$.

    If the event $F_4$ happens, then the experiment halts and outputs a random bit. Since $\com^*$ is chosen uniformly in the space $\hat{\G}$, and remains independent from the adversary's view until the challenge phase, then we have $|S_4 - S_3| \leq \PR[F_4] \leq q_D/p$ where $q_D$ represents the number of decryption queries before the challenge phase $p$ is the order of the group $\hat{\G}$.

  \item[\textsf{Game} $5$ :] In this game, we modify the decryption oracle during the second query phase, let us denote the event $F_5$: the ciphertext submitted by the adversary to the decryption oracle during the second query phase contains the commitments $(\vec{C}_{\com}, \vec{C}_{\open})$ which verifies $\com = \com^*$ but $\open \neq \open^*$.

    The experiment halts if $F_5$ occurs and outputs a random bit. Thus we have $|S_5- S_4| \leq \PR[F_5]$.

    And if $F_5$ occurs, we can easily construct an adversary $\mathcal{B}$ of the target collision-resistance of the underlying structure-preserving trapdoor commitment $TC$ which contradicts the Double Pairing assumption.

    Thus we have $|S_5 - S_4| \leq \PR[F_2] \leq adv_{\mathcal{B}}^{TCR-CR}(\lambda) \leq adv_{\mathcal{B}}^{DP}(\lambda)$. 

  \item[\textsf{Game} $6$ :] We modify again the decryption oracle during the second query phase. During the second query phase, the ciphertext submitted by the adversary contains $(\vec{C}_\com, \vec{C}_\open)$ such that $\com = \com^*$ and $\open = \open^*$ but $(C_0, C_1, C_2)$ does not verify that $(C_0/ C_0^*, C_1/C_1^*, C_2/C_2^*) \in Span((X, g_1, g_2))$ and pass all other verification, we denote this event $F_6$. If $F_6$ occurs, the experiment halts and outputs a random bit. Thus we have $|S_6 - S_5| \leq \PR[F_6]$. The event $F_6$ is contradict the strong unforgeability of the underlying LHSPS signature(The GS proof is in the perfect binding setting, if $F_6$ occurs, we can extract a signature which is not in the subspace $Span((g, C_0, C_1, C_2), (a, X, g_1, g_2))$). Thus we have $|S_6 - S_5| \leq adv_{\mathcal{B}}^{SUF-OTLHS}(\lambda) \leq adv_{\mathcal{B}}^{DP}(\lambda)$.
    
    
  \item[\textsf{Game} $7$ :] In this game, we modify the distribution of the public keys. We compute the public keys $(\vec{u}_1, \vec{u}_2)$ in the following way:
    \begin{align*}
      \vec{u}_1 &= (\hat{g}, \hat{h}) \in \hat{\G}^2\\
      \vec{u}_1 &= (\hat{g}^{\rho_u}, \hat{h}^{\rho_u}) \cdot (1, \com^{-1}) \in \hat{\G}^2
    \end{align*}

    Since the Challenger does not use $\rho_u$ or $\rho_u'$ in the security game, thus an adversary who can make difference between \textsf{Game} $6$ and \textsf{Game} $7$, is an adversary agains the DDH assumption in $\hat{\G}$. Thus we have $|S_6 - S_5| \leq adv_{\mathcal{B}}^{DDH}(\lambda)$.

  \item[\textsf{Game} $8$ :] In this game, instead of generate the proof $(\vec{\pi}_{\theta^*, 1, 1}, \vec{\pi}_{\theta^*, 2, 1})$ using the witness $\theta^*$, we generate a random value $r \sample \mathbb{Z}_p^*$ and generate $(\vec{C}_{\theta^*}, \vec{\pi}_{\theta^*,1,1}, \vec{\pi}_{\theta^*,2,1})$ as following:
    \begin{align*}
      \vec{C}_{\theta^*} &= \vec{u}_1^r & \vec{\pi}_{\theta^*,1,1} &= g_1^r \cdot C_1^{*-\rho_u} & \vec{\pi}_{\theta^*,2,1} &= g_2^r \cdot C_2^{*-\rho_u}
    \end{align*}

    Notice that even the proof elements are generated without using the witness $\theta^* = log_{g_1}(C_1^*) = log_{g_2}(C_2^*)$. The distribution of the proof is remain the same as in the original proof. In fact, let us define $\tilde{r} = r - \rho_u \cdot \theta^*$, we have:
    \begin{align*}
      \vec{C}_{\theta}^* &= \vec{u}_{\com}^{\theta^*} \cdot \vec{u}_1^{\tilde{r}} &  \vec{\pi}_{\theta^*,1,1} &= g_1^{\tilde{r}} & \vec{\pi}_{\theta^*,2,1} &= g_2^{\tilde{r}}
    \end{align*}

    Thus we have $S_8 = S_7$.

  \item[\textsf{Game} $9$ :] In this game, we modify the ciphertext generation in the challenge phase. Instead of compute the ciphertext using the public key $(X, g_1, g_2)$, we generate it with the secret key $(x_1, x_2)$:
    \begin{align*}
      C_1^* &= g_1^{\theta^*} & C_2^* &= g_2^{\theta^*} & C_0 &= M_b\cdot C_1^{*x_1} \cdot C_2^{*x_2}  
    \end{align*}
    Since the ciphertext remains exactly the same as in the \textsf{Game} $8$. Thus this modification does not change the view of the adversary, which means $S_9 = S_8$.


  \item[\textsf{Game} $10$ :] In this game, we modify again the ciphertext generation in the challenge phase. Recall that since the \textsf{Game} $8$, we don't use anymore $\theta^{*}$ to generate $(\vec{C}_{\theta^*}, \vec{\pi}_{\theta^*,1,1}, \vec{\pi}_{\theta^*,2,1})$, then we generate two random values $(\theta_1, \theta_2) \sample \mathbb{Z}_p^2$ and compute the ciphertext as following:
    \begin{align*}
      C_1 &= g_1^{\theta_1} & C_2 &= g_2^{\theta_2} & C_0 &= m_b \cdot C_1^{x_1} \cdot C_2^{x_2}
    \end{align*}

    As we don't use anymore $\theta_1$ nor $\theta_2$ in the whole game, we can easily construct a reduction from an adversary who can make difference between \textsf{Game} $10$ and \textsf{Game} $9$ to an adversary against the DDH assumption. Thus we have $|S_{10} - S_9| \leq adv_{\mathcal{B}}^{DDH}(\lambda)$.
    
  \end{description}

  Notice that in the final game, the ciphertext component is as follows:
  \begin{align*}
    C_1 &= g_1^{\theta^*} & C_2 &= g_2^{\theta^*+ \theta'} & C_0 &= m_b \cdot X_1^{\theta^*} \cdot g_2^{x_2 \cdot \theta'}
  \end{align*}

  As $x_2$ is completely independent of the adversary's view, $C_0$ can be seen as a one-time pad of the message $m_b$. Thus the adversary does not have any information about the bit $b$. Then we have $S_{10} = 0$.

  For summary, we have
  \begin{align*}
    adv_{\mathcal{A}}^{RCCA}(\lambda) &= S_0\\
    &\leq S_{10} + 2 \cdot adv_{\mathcal{B}}^{DDH}(\lambda) + 2 \cdot adv_{\mathcal{B}}^{DP}(\lambda) + q_D/p + adv_{\mathcal{B}}^{wit-IND}(\lambda)\\
    &= 2 \cdot adv_{\mathcal{B}}^{DDH}(\lambda) + 2 \cdot adv_{\mathcal{B}}^{DP}(\lambda) + q_D/p + adv_{\mathcal{B}}^{wit-IND}(\lambda) \in negl(\lambda)
  \end{align*}
\end{proof}

\input{proofRCCA2}


\begin{section}{Conclusion}
  During this internship, we explore many aspect of security proof of the cryptographic protocols
and we have constructed and improved several existing rerandomizable encryption scheme.
And also our RCCA encryption scheme is based on the SXDH assumption which is a well studied and general believed assumption,
this means we don't need the symmetric pairing for our construction which is not very efficient.
However, there are many questions left open. Can we construct efficient protocol for larger family of homomorphic encryption scheme.

\end{section}

\newpage

\printbibliography

\begin{appendices}
  
  \begin{section}{The instantiation of Linearly Homomorphic Structure Preserving Signature(LHSPS) based on the DP assumption}
    
\label{LHSPS}
We give as follows an concrete instantiation of the LHSPS proposed by Libert \etal~\cite{DBLP:conf/crypto/LibertPJY13} which is used in our construction.
  
\begin{description}
\item[\boldmath{$LHSPS.\Setup(1^{\lambda})$}]:
  \begin{enumerate}
  \item We generate a bilinear group system $(\G, \hat{\G}, \G_T, e:\G \times \hat{\G} \to \G_T)$.
  \item Choose random group generators $(\hat{g}_z, \hat{g}_r) \sample \hat{\G}^2$.
  \item Choose random group generator $g \sample \G$.
  \item Output $\PPP = (\hat{g}_z, \hat{g}_r, g)$.
  \end{enumerate}
\item[\boldmath{$LHSPS.\KeyGen(\PPP)$}]:
  \begin{enumerate}
  \item Generate $(\{\hat{\chi}_i, \hat{\gamma}_i\}_{i =1 }^k, \hat{\zeta}, \hat{\rho}) \sample \mathbb{Z}_p^{2k+2}$.
  \item Compute for $i \in \{1, \dots, k\}$, $\hat{g}_i \gets \hat{g}_z^{\hat{\chi}_i}\hat{g}_r^{\hat{\gamma}_i}$.
  \item Output $\vk = (\{\hat{g}_i\}_{i=1}^k) \in \hat{\G}^{k}$ and $\sk = (\{\hat{\chi}_i, \hat{\gamma}_i\}_{i=1}^k) \in \mathbb{Z}_p^{2k}$.
  \end{enumerate}
\item[\boldmath{$LHSPS.\Sig(\sk, \{m_1, \dots, m_k\})$}]: where $(m_1, \dots, m_k) \in \G^k$
  \begin{enumerate}
  \item Parse $\sk$ with $(\{\hat{\chi}_i, \hat{\gamma}_i\}_{i = 1}^k)$.
  \item Compute
    \begin{align*}
      z &= \prod_{i=1}^km_i^{\hat{\chi}_i} & r &= \prod_{i =1}^km_i^{\hat{\gamma}_i} 
    \end{align*}
  \item Output the signature $\sigma = (z, r)$
  \end{enumerate}

\item[\boldmath{$LHSPS.\Verif(\vk, \sigma, \{m_1, \dots, m_k\})$}]:
  \begin{enumerate}
  \item Parse the signature $\sigma$ with $\sigma = (z, r)$ and the verification key $\vk$ with $\vk = (\hat{g}_1, \dots, \hat{g}_k)$
  \item Verify the pairing equation:
    \begin{align*}
      e(z, \hat{g}_z) \cdot e(r, \hat{g}_r) &=  \cdot \prod_{i = 1}^ke(m_i, \hat{g}_i)
    \end{align*}
  \end{enumerate}
\end{description}

  \end{section}

  \begin{section}{The instantiation of Structure-Preserving Commitment(SPC) based on the DP assumption}
    \label{SPC}
    \begin{description}
\item[\boldmath{$SPC.\Setup(1^{\lambda},\ell)$}]: Generate the public parameters for the trapdoor commitment scheme for security parameter $\lambda$ and message vector length $\ell$. 
  \begin{enumerate}
  \item Choose a random prime $p< 2^{\lambda}$.
  \item Generate a asymmetric pairing groups $(\G, \hat{\G}, \G_T)$ of prime order $p$ and a pairing function $e : \G \times \hat{\G} \to \G_T$.
  \item Generate the group generators $(g, \hat{g}) \in \G \times \hat{\G}$.	
  \item $\PPP = (p, \G, \hat{\G}, \G_T, e, g, \hat{g},\ell)$.
  \end{enumerate}

\item[\boldmath{$SPC.\KeyGen(\PPP)$}]:
  \begin{enumerate}
  \item For $i = 1, \dots, \ell+2$, generate random values $\rho_i \sample \mathbb{Z}_p^*$, then compute $\hat{X}_i \gets \hat{g}^{\rho_i}$.
  \item Set $\ck \gets \{\hat{X}_i\}_{i = 1}^{\ell+2}$ and $tk \gets \{\rho_i\}_i^{\ell+2}$.
  \end{enumerate}


\item[\boldmath{$SPC.\Com(\PPP, \ck, \vec{M})$}]: where $\vec{\hat{M}} = (\hat{M}_1, \dots, \hat{M}_\ell) \in \hat{\G}^\ell$.
  \begin{enumerate}
  \item Choose a random value $w_z \in \mathbb{Z}_p^*$ then compute $g_z = g^{w_z}$.
  \item For $i = 1, \dots, \ell$, generate random values $\chi_i \sample \mathbb{Z}_p$ and compute $g_i = g^{\chi_i}$.
  \item Set $\vk_{pots} \gets (g_z, g_1, \dots, g_\ell) \in \G^{\ell+1}$ and $\sk_{pots} \gets (w_z,\chi_1, \dots, \chi_\ell)$.
  \item Choose randomly $a \sample \mathbb{Z}_p$, then set $\ovk_{pots} =A = g^{a}$ and $\osk_{pots} = a$.
  \item Using the signing key $\sk_{pots}$ to generate signatures of the message $\vec{\hat{M}}$ \wrt to the one-time signature's secret key $\osk_{pots}$:
    \begin{enumerate}
    \item Generate random value $\zeta_1 \in \mathbb{Z}_p$
    \item Compute the signature $(\hat{Z}, \hat{R}) \in \hat{\G}^2$ for $\vec{\hat{M}}$:
      \begin{align*}
	\hat{Z} &= \hat{g}^{\zeta_1} & \hat{R} = \hat{g}^{a-\zeta_1 w_z}\prod_{i=1}^{\ell} \hat{M}_i^{\chi_i} 
      \end{align*}
    \end{enumerate}
  \item We use the commitment key to generate the commitment for the message.
    \begin{enumerate}
    \item Set $(m_1, \dots, m_{\ell+2}) \gets (\chi_1, \dots, \chi_\ell, w_z, a)$ 
    \item Parse $\vec{\ck}$ as $(\hat{X}_1, \dots, \hat{X}_{\ell+2})$.
    \item Generate a random value $\zeta_2 \gets \mathbb{Z}_p^*$ and compute:
      \begin{align*}
	\hat{C} &= \hat{g}^{\zeta_2}\prod_{i = 1}^{\ell+2}\hat{X}_i^{m_i} & D &= g^{\zeta_2}
      \end{align*}
    \end{enumerate}
  \item We set the commitment as $\com  = \hat{C}$ and $\open = (D, g_z, g_1, \dots, g_\ell, \ovk_{pots} = g^a, \hat{Z}, \hat{R}) \in \G^{\ell+3} \times \hat{\G}^{2}$.
  \end{enumerate}
  
\item[\boldmath{$SPC.\Verif(\ck, \com, \vec{\hat{M}}, \open)$}] :
  \begin{enumerate}
  \item Parse $\vec{\hat{M}}$ with $(\hat{M}_1, \dots, \hat{M}_\ell)$ and $\open$ with $(D, g_z, g_1, \dots, g_\ell, \ovk_{pots} = g^a, \hat{Z}, \hat{R})$.
  \item Set $\vec{N} = (N_1, \dots, N_{\ell+2}) = (g_1, \dots, g_\ell, g_z, \ovk_{pots})$
  \item Using $\ovk_{pots} = A \in \G$, verify the following equations:
    \begin{align*}
      e(g, \hat{C}) &= e(D, \hat{g}) \prod_{i = 1}^{\ell+2} e(N_i, \hat{X}_i) & e(A, \hat{g}) &= e(g_z, \hat{Z}) \cdot e(g, \hat{R}) \cdot \prod_{i = 1}^\ell e(g_i, \hat{M}_i) 
    \end{align*}
  \end{enumerate}
\end{description}

  \end{section}
  
  
  \begin{section}{Details of the rerandomization algorithm of RCCA encryption by Chase \etal~\cite{DBLP:conf/eurocrypt/ChaseKLM12}}
    \label{Rerandomization}
    \subsection{Construction}

\begin{description}
\item[First Stage]: Rerandomize the ciphertext and update the proofs.
  \begin{enumerate}
  \item To update the proofs we need the old proof elements of the equations $(13, 14, 15)$. We first explicit the proofs elements for these equations:
    \begin{enumerate}
    \item Equation 13:

      $e(\boxed{d^b},h) = e(\boxed{H}, d)$: 
      
      The Verification equation is: $E(\vec{C}_b, h) = E(\vec{C}_{H}, d) \cdot E(\pi_{13,1}, \vec{g}_1)\cdot E((\pi_{13,2}), \vec{g}_2)\cdot E((\pi_{13,3}), \vec{g}_3)$

      with
      \begin{itemize}
      \item[$\pi_{13,1} = $] $h^{r_{d^b}} \cdot d^{-r_{H}}$
      \item[$\pi_{13,2} = $] $h^{s_{d^b}} \cdot d^{-s_{H}}$
      \item[$\pi_{13,3} = $] $h^{t_{d^b}} \cdot d^{-t_{H}}$
      \end{itemize}
      
    \item Equation 14:

      $e(\boxed{F},d) = e(f, \boxed{d^b})$

      The Verification equation is: $E(\vec{C}_F, (d)) = E((f), \vec{C}_{d^b}) \cdot E((\pi_{14,1}), \vec{g}_1)\cdot E((\pi_{14,2}), \vec{g}_2)\cdot E((\pi_{14,3}), \vec{g}_3)$

      with

      \begin{itemize}
      \item[$\pi_{14,1} = $] $d^{r_F} \cdot f^{-r_{d^b}}$    
      \item[$\pi_{14,2} = $] $d^{s_F} \cdot f^{-s_{d^b}}$    
      \item[$\pi_{14,3} = $] $d^{t_F} \cdot f^{-t_{d^b}}$
      \end{itemize}
      
    \item Equation 15:

      $e(\boxed{G},d) = e(g, \boxed{d^b})$

      The Verification equation is: $E(\vec{C}_G, (d)) = E((g), \vec{C}_{d^b}) \cdot E((\pi_{15,1}), \vec{g}_1)\cdot E((\pi_{15,2}), \vec{g}_2)\cdot E((\pi_{15,3}), \vec{g}_3)$

      with

      \begin{itemize}
      \item[$\pi_{15,1} = $] $d^{r_G} \cdot g^{-r_{d^b}}$    
      \item[$\pi_{15,1} = $] $d^{s_G} \cdot g^{-s_{d^b}}$    
      \item[$\pi_{15,1} = $] $d^{t_G} \cdot g^{-t_{d^b}}$
      \end{itemize}

    \end{enumerate}

  \item We generate two new random values $(\theta_1', \theta_2')$, and compute the new ciphertext vector $\vec{C}' = (C_1', C_2', C_3') = (C_1 \cdot f^{\theta_1'}, C_2 \cdot g^{\theta_2'}, C_3 \cdot h^{\theta_1'+\theta_2'})$. We compute the new commitments, then update the proofs for the equation $(2, 3, 4, 5, 6, 7, 8, 9, 10)$. Update the commitments:
    \begin{enumerate}
    \item $\vec{C}_{\Delta_1}' = \vec{C}_{\Delta_1} \cdot \vec{C}_{F}^{\theta_1'}$
    \item $\vec{C}_{\Delta_2}' = \vec{C}_{\Delta_2} \cdot \vec{C}_{G}^{\theta_2'}$
    \item $\vec{C}_{\Delta_3}' = \vec{C}_{\Delta_3} \cdot \vec{C}_{H}^{\theta_1' + \theta_2'}$
    \item $\vec{C}_{R_1}' = \vec{C}_{R_1} \cdot \vec{C}_{d^b}^{\theta_1'}$
    \item $\vec{C}_{R_2}' = \vec{C}_{R_2} \cdot \vec{C}_{d^b}^{\theta_2'}$
    \item $\vec{C}_{S_1}' = \vec{C}_{S_1} \cdot \vec{C}_{d^b}^{\theta_1'}$
    \item $\vec{C}_{S_2}' = \vec{C}_{S_2} \cdot \vec{C}_{d^b}^{\theta_2'}$
    \end{enumerate}
    \begin{enumerate}
      %2
    \item Equation 2:

      $e(\boxed{\Delta_1},d) = e(C_1, \boxed{d^b})$:
      
      The Verification equation is:  $E(\vec{C}_{\Delta_1}, (d)) = E((C_1), \vec{C}_{d^b}) \cdot E((\pi_{2,1}), \vec{g}_1)\cdot E((\pi_{2,2}), \vec{g}_2)\cdot E((\pi_{2,3}), \vec{g}_3)$
      
      with
      \begin{itemize}
      \item[$\pi_{2,1} = $] $d^{r_{\Delta_1}} \cdot C_1^{-r_{d^b}}$
      \item[$\pi_{2,2} = $] $d^{s_{\Delta_1}} \cdot C_1^{-s_{d^b}}$
      \item[$\pi_{2,3} = $] $d^{t_{\Delta_1}} \cdot C_1^{-t_{d^b}}$
      \end{itemize}

      The new proofs of the equations are:
      
      \begin{itemize}
      \item[$\pi_{2,1}' = $] $d^{r_{\Delta_1} + r_{\Delta_1}'} \cdot C_1^{-r_{d^b}} \cdot f^{-r_{d^b} \cdot \theta_1'}$    
      \item[$\pi_{2,2}' = $] $d^{s_{\Delta_1} + s_{\Delta_1}'} \cdot C_1^{-s_{d^b}} \cdot f^{-s_{d^b} \cdot \theta_1'}$    
      \item[$\pi_{2,3}' = $] $d^{t_{\Delta_1} + t_{\Delta_1}'} \cdot C_1^{-t_{d^b}} \cdot f^{-t_{d^b} \cdot \theta_1'}$
      \end{itemize}
      
      Using the proof $\vec{\pi}_{14}$, we can update the proof elements:
      \begin{itemize}
      \item[$\pi_{2,1}' = $] $\pi_{2,1} \cdot \pi_{14,1}^{\theta_1'}$    
      \item[$\pi_{2,2}' = $] $\pi_{2,2} \cdot \pi_{14,2}^{\theta_1'}$    
      \item[$\pi_{2,3}' = $] $\pi_{2,3} \cdot \pi_{14,3}^{\theta_1'}$
      \end{itemize}



      %3
    \item Equation 3:
      
      $e(\boxed{\Delta_2},d) = e(C_2, \boxed{d^b})$:
      
      The Verification equation is:  $E(\vec{C}_{\Delta_2}, (d)) = E((C_2), \vec{C}_{d^b}) \cdot E((\pi_{3,1}), \vec{g}_1)\cdot E((\pi_{3,2}), \vec{g}_2)\cdot E((\pi_{3,3}), \vec{g}_3)$
      
      with
      \begin{itemize}
      \item[$\pi_{3,1} = $] $d^{r_{\Delta_2}} \cdot C_2^{-r_{d^b}}$
      \item[$\pi_{3,2} = $] $d^{s_{\Delta_2}} \cdot C_2^{-s_{d^b}}$
      \item[$\pi_{3,3} = $] $d^{t_{\Delta_2}} \cdot C_2^{-t_{d^b}}$
      \end{itemize}
      
      The new proofs of the equations are:
      
      \begin{itemize}
      \item[$\pi'_{3,1} = $] $d^{r_{\Delta_2} + r_{\Delta_2}'} \cdot C_2^{-r_{d^b}} \cdot g^{-r_{d^b} \cdot \theta_2'}$    
      \item[$\pi'_{3,2} = $] $d^{s_{\Delta_2} + s_{\Delta_2}'} \cdot C_2^{-s_{d^b}} \cdot g^{-s_{d^b} \cdot \theta_2'}$    
      \item[$\pi'_{3,3} = $] $d^{t_{\Delta_2} + t_{\Delta_2}'} \cdot C_2^{-t_{d^b}} \cdot g^{-t_{d^b} \cdot \theta_2'}$
      \end{itemize}

      Using the proof $\vec{\pi}_{15}$, we can update the proof elements:

      \begin{itemize}
      \item[$\pi'_{3,1} = $] $\pi_{3,1} \cdot \pi_{15,1}^{\theta_2'}$    
      \item[$\pi'_{3,2} = $] $\pi_{3,2} \cdot \pi_{15,2}^{\theta_2'}$    
      \item[$\pi'_{3,3} = $] $\pi_{3,3} \cdot \pi_{15,3}^{\theta_2'}$
      \end{itemize}


      
      %4
    \item Equation 4:

      $e(\boxed{\Delta_3},d) = e(C_3, \boxed{d^b})$:
      
      The Verification equation is:  $E(\vec{C}_{\Delta_3}, (d)) = E((C_3), \vec{C}_{d^b}) \cdot E((\pi_{4,1}), \vec{g}_1)\cdot E((\pi_{4,2}), \vec{g}_2)\cdot E((\pi_{4,3}), \vec{g}_3)$
      
      with
      \begin{itemize}
      \item[$\pi_{4,1} = $] $d^{r_{\Delta_3}} \cdot C_3^{-r_{d^b}}$
      \item[$\pi_{4,2} = $] $d^{s_{\Delta_3}} \cdot C_3^{-s_{d^b}}$
      \item[$\pi_{4,3} = $] $d^{t_{\Delta_3}} \cdot C_3^{-t_{d^b}}$
      \end{itemize}

      The new proofs of the equations are:
      \begin{itemize}
      \item[$\pi_{4,1}' = $] $d^{r_{\Delta_3} + r'_{\Delta_3}} \cdot C_3^{-r_{d^b}} \cdot h^{- r_{d^b} \cdot (\theta_1'+ \theta_2')}$
      \item[$\pi_{4,2}' = $] $d^{s_{\Delta_3} + s'_{\Delta_3}} \cdot C_3^{-s_{d^b}} \cdot h^{- s_{d^b} \cdot (\theta_1'+ \theta_2')}$
      \item[$\pi_{4,3}' = $] $d^{t_{\Delta_3} + t'_{\Delta_3}} \cdot C_3^{-t_{d^b}} \cdot h^{- t_{d^b} \cdot (\theta_1'+ \theta_2')}$  
      \end{itemize}
      
      Using the proof $\vec{\pi}_{13}$, we can update the proof elements:

      \begin{itemize}
      \item[$\pi'_{4,1} = $] $\pi_{4,1} \cdot \pi_{13,1}^{-(\theta_1' + \theta_2')}$
      \item[$\pi'_{4,2} = $] $\pi_{4,1} \cdot \pi_{13,2}^{-(\theta_1' + \theta_2')}$
      \item[$\pi'_{4,3} = $] $\pi_{4,1} \cdot \pi_{13,3}^{-(\theta_1' + \theta_2')}$
      \end{itemize}

      %5
    \item Equation 5:

      $e(\boxed{\Delta_1},d) = e(f, \boxed{R_1})$:
      
      The Verification equation is:  $E(\vec{C}_{\Delta_1}, (d)) = E((f), \vec{C}_{R_1}) \cdot E((\pi_{5,1}), \vec{g}_1)\cdot E((\pi_{5,2}), \vec{g}_2)\cdot E((\pi_{5,3}), \vec{g}_3)$
      
      with
      \begin{itemize}
      \item[$\pi_{5,1} = $] $d^{r_{\Delta_1}} \cdot f^{-r_{R_1}}$
      \item[$\pi_{5,2} = $] $d^{s_{\Delta_1}} \cdot f^{-s_{R_1}}$
      \item[$\pi_{5,3} = $] $d^{t_{\Delta_1}} \cdot f^{-t_{R_1}}$
      \end{itemize}

      Using the proof $\vec{\pi}_{14}$ we can update the proof elements:
      \begin{itemize}
      \item[$\pi'_{5,1} = $] $\pi_{5,1} \cdot \pi_{14,1}^{\theta_1'}$
      \item[$\pi'_{5,2} = $] $\pi_{5,2} \cdot \pi_{14,2}^{\theta_1'}$
      \item[$\pi'_{5,3} = $] $\pi_{5,3} \cdot \pi_{14,3}^{\theta_1'}$
      \end{itemize}

      %6
    \item Equation 6:

      $e(\boxed{\Delta_2},d) = e(f, \boxed{R_2})$:

      The Verification equation is:  $E(\vec{C}_{\Delta_2}, (h)) = E((f), \vec{C}_{R_2}) \cdot E((\pi_{6,1}), \vec{g}_1)\cdot E((\pi_{6,2}), \vec{g}_2)\cdot E((\pi_{6,3}), \vec{g}_3)$

      with
      \begin{itemize}
      \item[$\pi_{6,1} = $] $d^{r_{\Delta_2}} \cdot f^{-r_{R_2}}$
      \item[$\pi_{6,2} = $] $d^{s_{\Delta_2}} \cdot f^{-s_{R_2}}$
      \item[$\pi_{6,3} = $] $d^{t_{\Delta_2}} \cdot f^{-t_{R_2}}$
      \end{itemize}

      Using the proof $\vec{\pi}_{15}$ we can update the proof elements:
      \begin{itemize}
      \item[$\pi'_{6,1} = $] $\pi_{6,1} \cdot d^{r'_{\Delta_2}} \cdot f^{-r'_{R_2}} = \pi_{7,1} \cdot \pi_{15,1}^{\theta_2'}$
      \item[$\pi'_{6,2} = $] $\pi_{6,2} \cdot d^{s'_{\Delta_2}} \cdot f^{-s'_{R_2}} = \pi_{7,2} \cdot \pi_{15,2}^{\theta_2'}$
      \item[$\pi'_{6,3} = $] $\pi_{6,3} \cdot d^{t'_{\Delta_2}} \cdot f^{-t'_{R_2}} = \pi_{7,3} \cdot \pi_{15,3}^{\theta_2'}$
      \end{itemize}
      
      %7
    \item Equation 7:

      $e(\boxed{\Delta_3},d) \cdot e(\boxed{M}, d^{-1})= e(\boxed{R_1}, h) \cdot e(\boxed{R_2}, h)$:
      
      The Verification equation is:  $E(\vec{C}_{\Delta_3}, (d)) \cdot E(\vec{C}_M, (d^{-1}))= E(\vec{C}_{R_1}, (h)) \cdot E(\vec{C}_{R_2}, (h)) \cdot E((\pi_{7,1}), \vec{g}_1)\cdot E((\pi_{7,2}), \vec{g}_2)\cdot E((\pi_{7,3}), \vec{g}_3)$
      
      with
      \begin{itemize}
      \item[$\pi_{7,1} = $] $d^{r_{\Delta_3}} \cdot d^{-r_M} \cdot h^{-r_{R_1}} \cdot h^{-r_{R_2}}$
      \item[$\pi_{7,2} = $] $d^{s_{\Delta_3}} \cdot d^{-s_M} \cdot h^{-s_{R_1}} \cdot h^{-s_{R_2}}$
      \item[$\pi_{7,3} = $] $d^{t_{\Delta_3}} \cdot d^{-t_M} \cdot h^{-t_{R_1}} \cdot h^{-t_{R_2}}$
      \end{itemize}   

      Using the proof $\vec{\pi}_{13}$ we can update the proof elements:
      \begin{itemize}
      \item[$\pi'_{7,1} = $] $\pi_{7,1} \cdot d^{r'_{\Delta_3}} \cdot h^{-r'_{R_1}} \cdot h^{-r'_{R_2}} = \pi_{7,1} \cdot \pi_{13,1}^{-(\theta_1'+\theta_2')}$
      \item[$\pi'_{7,2} = $] $\pi_{7,2} \cdot d^{s'_{\Delta_3}} \cdot h^{-s'_{R_1}} \cdot h^{-s'_{R_2}} = \pi_{7,2} \cdot \pi_{13,2}^{-(\theta_1'+\theta_2')}$
      \item[$\pi'_{7,3} = $] $\pi_{7,3} \cdot d^{t'_{\Delta_3}} \cdot h^{-t'_{R_1}} \cdot h^{-t'_{R_2}} = \pi_{7,3} \cdot \pi_{13,3}^{-(\theta_1'+\theta_2')}$
      \end{itemize}

      
      %8
    \item Equation 8:

      $e(C_1/\boxed{\Delta_1}, d) = e (\boxed{D_1},d) \cdot e(f^{-1}, \boxed{S_1})$

      The Verification equation is: $ E((C_1)/\vec{C}_{\Delta_1}, (d)) = E(\vec{C}_{D_1},(d)) \cdot E((f^{-1}), \vec{C}_{S_1})\cdot E((\pi_{8,1}), \vec{g}_1)\cdot E((\pi_{8,2}), \vec{g}_2)\cdot E((\pi_{8,3}), \vec{g}_3)$

      with
      \begin{itemize}
      \item[$\pi_{8,1} = $] $d^{-r_{\Delta_1}} \cdot d^{-r_{D_1}} \cdot f^{r_{S_1}}$
      \item[$\pi_{8,2} = $] $d^{-s_{\Delta_1}} \cdot d^{-s_{D_1}} \cdot f^{s_{S_1}}$
      \item[$\pi_{8,3} = $] $d^{-t_{\Delta_1}} \cdot d^{-t_{D_1}} \cdot f^{t_{S_1}}$
      \end{itemize}
      Using the proof $\vec{\pi}_{14}$ we can update the proof elements:
      \begin{itemize}
      \item[$\pi'_{8,1} = $] $\pi_{8,1} \cdot d^{-r'_{\Delta_1}} \cdot f^{r'_{S_1}} = \pi_{8,1} \cdot \pi_{14,1}^{-\theta_1'}$
      \item[$\pi'_{8,2} = $] $\pi_{8,2} \cdot d^{-s'_{\Delta_1}} \cdot f^{s'_{S_1}} = \pi_{8,2} \cdot \pi_{14,2}^{-\theta_1'}$
      \item[$\pi'_{8,3} = $] $\pi_{8,3} \cdot d^{-t'_{\Delta_1}} \cdot f^{t'_{S_1}} = \pi_{8,3} \cdot \pi_{14,3}^{-\theta_1'}$
      \end{itemize}

      %9
    \item Equation 9:

      $e(C_2/\boxed{\Delta_2}, d) = e (\boxed{D_2},d) \cdot e(g^{-1}, \boxed{S_2})$

      The Verification equation is: $ E((C_2)/\vec{C}_{\Delta_2}, (d)) = E(\vec{C}_{D_2},(d)) \cdot E((g^{-1}), \vec{C}_{S_2})\cdot E((\pi_{9,1}), \vec{g}_1)\cdot E((\pi_{9,2}), \vec{g}_2)\cdot E((\pi_{9,3}), \vec{g}_3)$

      with
      \begin{itemize}
      \item[$\pi_{9,1} = $] $d^{-r_{\Delta_2}} \cdot d^{-r_{D_2}} \cdot g^{r_{S_2}}$
      \item[$\pi_{9,2} = $] $d^{-s_{\Delta_2}} \cdot d^{-s_{D_2}} \cdot g^{s_{S_2}}$
      \item[$\pi_{9,3} = $] $d^{-t_{\Delta_2}} \cdot d^{-t_{D_2}} \cdot g^{t_{S_2}}$
      \end{itemize}
      Using the proof $\vec{\pi}_{15}$ we can update the proof elements:
      
      \begin{itemize}
      \item[$\pi'_{9,1} = $] $\pi_{9,1} \cdot d^{-r'_{\Delta_2}} \cdot f^{-r'_{S_2}} = \pi_{9,1} \cdot \pi_{15,1}^{-\theta_2'}$
      \item[$\pi'_{9,2} = $] $\pi_{9,2} \cdot d^{-s'_{\Delta_2}} \cdot f^{-s'_{S_2}} = \pi_{9,2} \cdot \pi_{15,2}^{-\theta_2'}$
      \item[$\pi'_{9,3} = $] $\pi_{9,3} \cdot d^{-t'_{\Delta_2}} \cdot f^{-t'_{S_2}} = \pi_{9,3} \cdot \pi_{15,3}^{-\theta_2'}$
      \end{itemize}

      %10
    \item Equation 10:

      $e(C_3/\boxed{\Delta_3}, d) = e (\boxed{D_3},d) \cdot e(\boxed{S_1}, h^{-1})\cdot e(\boxed{S_2}, h^{-1})$

      The Verification equation is: $ E((C_3/\vec{C}_{\Delta_3}), (d)) = E(\vec{C}_{D_3},(d)) \cdot E((h^{-1}), \vec{C}_{S_1}) \cdot E((h^{-1}), \vec{C}_{S_2}) \cdot E((\pi_{10,1}), \vec{g}_1)\cdot E((\pi_{10,2}), \vec{g}_2)\cdot E((\pi_{10,3}), \vec{g}_3)$

      with
      \begin{itemize}
      \item[$\pi_{10,1} = $] $d^{-r_{\Delta_3}} \cdot d^{-r_{D_3}} \cdot h^{r_{S_1}} \cdot h^{r_{S_2}}$
      \item[$\pi_{10,2} = $] $d^{-s_{\Delta_3}} \cdot d^{-s_{D_3}} \cdot h^{s_{S_1}} \cdot h^{s_{S_2}}$
      \item[$\pi_{10,3} = $] $d^{-t_{\Delta_3}} \cdot d^{-t_{D_3}} \cdot h^{t_{S_1}} \cdot h^{t_{S_2}}$
      \end{itemize}

      Using the proof $\vec{\pi}_{13}$ we can update the proof elements:

      \begin{itemize}
      \item[$\pi'_{10,1} = $] $\pi_{10,1} \cdot d^{-r'_{\Delta_3}} \cdot h^{r'_{S_1}} \cdot h^{r'_{S_2}} = \pi_{10,1} \cdot \pi_{13,1}^{\theta_1'+\theta_2'}$
      \item[$\pi'_{10,2} = $] $\pi_{10,1} \cdot d^{-s'_{\Delta_3}} \cdot h^{s'_{S_1}} \cdot h^{s'_{S_2}} = \pi_{10,2} \cdot \pi_{13,2}^{\theta_1'+\theta_2'}$
      \item[$\pi'_{10,3} = $] $\pi_{10,1} \cdot d^{-t'_{\Delta_3}} \cdot h^{t'_{S_1}} \cdot h^{t'_{S_2}} = \pi_{10,3} \cdot \pi_{13,3}^{\theta_1'+\theta_2'}$
      \end{itemize}
      
    \end{enumerate}

    
  \end{enumerate}

  
  
\item[Second Stage]: Rerandomize the commitments and the proofs.

  For each commitment $\vec{C}_{X}$ or $\vec{C}'_X$, we randomize it with $\tilde{vec{C}}_X = \vec{C}_X \cdot g_1^{r_X} \cdot g_2^{s_X} \cdot g_3^{t_X}$.
  \begin{enumerate}

    %1
  \item The proof of the quadratic equation: $e(d,\boxed{d^b}) = e(\boxed{d^b},\boxed{d^b})$
    
    The Verification equation is: $E(\iota(d), \vec{C}_{d^b}) = E(\vec{C}_{d^b}, \vec{C}_{d^b}) \cdot E(\vec{\pi}_{1,1}, \vec{g}_1)\cdot E(\vec{\pi}_{1,2}, \vec{g}_2)\cdot E(\vec{\pi_{1,3}}, \vec{g}_3)$

    with
    \begin{itemize}
    \item[$\vec{\pi}_{1,1} = $] $(d^b)^{2r_{d^b}}\cdot (\vec{g}_1^{r_{d^b}} \cdot \vec{g}_2^{s_{d^b}} \cdot \vec{g}_3^{t_{d^b}})^{r_{d^b}} \cdot (d^{-1})^{r_{d^b}}$
    \item[$\vec{\pi}_{1,2} = $] $(d^b)^{2s_{d^b}}\cdot (\vec{g}_1^{r_{d^b}} \cdot \vec{g}_2^{s_{d^b}} \cdot \vec{g}_3^{t_{d^b}})^{s_{d^b}} \cdot (d^{-1})^{s_{d^b}}$
    \item[$\vec{\pi}_{1,3} = $] $(d^b)^{2t_{d^b}}\cdot (\vec{g}_1^{r_{d^b}} \cdot \vec{g}_2^{s_{d^b}} \cdot \vec{g}_3^{t_{d^b}})^{t_{d^b}} \cdot (d^{-1})^{t_{d^b}}$
    \end{itemize}

    The new proofs of the new equations can be generated as:

    \begin{itemize}
    \item[$\tilde{\vec{\pi}}_{1,1} = $] $\vec{\pi}_{1,1} \cdot \vec{C}_{d^b}^{2\tilde{r}_{d^b}}\cdot (\vec{g}_1^{\tilde{r}_{d^b}} \cdot \vec{g}_2^{\tilde{s}_{d^b}} \cdot \vec{g}_3^{\tilde{t}_{d^b}})^{\tilde{r}_{d^b}} \cdot (d^{-1})^{\tilde{r}_{d^b}}$
    \item[$\tilde{\vec{\pi}}_{1,2} = $] $\vec{\pi}_{1,2} \cdot \vec{C}_{d^b}^{2\tilde{s}_{d^b}}\cdot (\vec{g}_1^{\tilde{r}_{d^b}} \cdot \vec{g}_2^{\tilde{s}_{d^b}} \cdot \vec{g}_3^{\tilde{t}_{d^b}})^{\tilde{s}_{d^b}} \cdot (d^{-1})^{\tilde{s}_{d^b}}$
    \item[$\tilde{\vec{\pi}}_{1,3} = $] $\vec{\pi}_{1,3} \cdot \vec{C}_{d^b}^{2\tilde{t}_{d^b}}\cdot (\vec{g}_1^{\tilde{r}_{d^b}} \cdot \vec{g}_2^{\tilde{s}_{d^b}} \cdot \vec{g}_3^{\tilde{t}_{d^b}})^{\tilde{t}_{d^b}} \cdot (d^{-1})^{\tilde{t}_{d^b}}$
    \end{itemize}



    %2
  \item Equation 2:

    $e(\boxed{\Delta_1'},d) = e(C_1', \boxed{d^b})$:
    
    The Verification equation is:  $E(\vec{C}_{\Delta_1'}, d) = E(C_1'), \vec{C}_{d^b}) \cdot E(\pi_{2,1}, \vec{g}_1)\cdot E(\pi_{2,2}, \vec{g}_2)\cdot E(\pi_{2,3}, \vec{g}_3)$
    
    with
    \begin{itemize}
    \item[$\pi_{2,1} = $] $d^{r_{\Delta_1'}} \cdot C_1'^{-r_{d^b}}$
    \item[$\pi_{2,2} = $] $d^{s_{\Delta_1'}} \cdot C_1'^{-s_{d^b}}$
    \item[$\pi_{2,3} = $] $d^{t_{\Delta_1'}} \cdot C_1'^{-t_{d^b}}$
    \end{itemize}

    The new proofs of the equations are:
    
    \begin{itemize}
    \item[$\tilde{\pi}_{2,1} = $] $\pi'_{2,1} \cdot d^{\tilde{r}_{\Delta_1'}}\cdot C_1'^{-\tilde{r}_{d^b}}$   
    \item[$\tilde{\pi}_{2,2} = $] $\pi'_{2,2} \cdot d^{\tilde{s}_{\Delta_1'}}\cdot C_1'^{-\tilde{s}_{d^b}}$   
    \item[$\tilde{\pi}_{2,3} = $] $\pi'_{2,3} \cdot d^{\tilde{t}_{\Delta_1'}}\cdot C_1'^{-\tilde{t}_{d^b}}$ 
    \end{itemize}



    %3
  \item Equation 3:
    
    $e(\boxed{\Delta_2’},d) = e(C_2', \boxed{d^b})$:
    
    The Verification equation is:  $E(\vec{C}_{\Delta_2}’, d) = E(C_2', \vec{C}_{d^b}) \cdot E(\pi_{4,1}, \vec{g}_1)\cdot E(\pi_{4,2}, \vec{g}_2)\cdot E(\pi_{4,3}, \vec{g}_3)$
    
    with
    \begin{itemize}
    \item[$\pi_{3,1} = $] $d^{r_{\Delta_2’}} \cdot C_2'^{-r_{d^b}}$ 
    \item[$\pi_{3,2} = $] $d^{s_{\Delta_2'}} \cdot C_2’^{-s_{d^b}}$ 
    \item[$\pi_{3,3} = $] $d^{t_{\Delta_2’}} \cdot C_2'^{-t_{d^b}}$ 
    \end{itemize}
    
    The new proofs of the equations are:
    
    \begin{itemize}
    \item[$\tilde{\pi}_{3,1} = $] $\pi_{3,1}' \cdot d^{\tilde{r}_{\Delta_2'}} \cdot C_2'^{-\tilde{r}_{d^b}}$ 
    \item[$\tilde{\pi}_{3,2} = $] $\pi_{3,2}' \cdot d^{\tilde{s}_{\Delta_2’}} \cdot C_2'^{-\tilde{s}_{d^b}}$    
    \item[$\tilde{\pi}_{3,3} = $] $\pi_{3,2}' \cdot d^{\tilde{t}_{\Delta_2'}} \cdot C_2'^{-\tilde{t}_{d^b}}$ 
    \end{itemize}

    
    %4
  \item Equation 4:

    $e(\boxed{\Delta_3’},d) = e(C_3', \boxed{d^b})$:
    
    The Verification equation is:  $E(\vec{C}_{\Delta_3’}, d) = E(C_3', \vec{C}_{d^b}) \cdot E(\pi_{5,1}, \vec{g}_1)\cdot E(\pi_{5,2}, \vec{g}_2)\cdot E(\pi_{5,3}, \vec{g}_3)$
    
    with
    \begin{itemize}
    \item[$\pi_{4,1} = $] $d^{r_{\Delta_3’}} \cdot C_3'^{-r_{d^b}}$
    \item[$\pi_{4,2} = $] $d^{s_{\Delta_3'}} \cdot C_3'^{-s_{d^b}}$
    \item[$\pi_{4,3} = $] $d^{t_{\Delta_3’}} \cdot C_3'^{-t_{d^b}}$
    \end{itemize}

    The new proofs of the equations are:

    \begin{itemize}
    \item[$\tilde{\pi}_{4,1} = $] $\pi_{4,1}' \cdot d^{\tilde{r}_{\Delta_3’}} \cdot C_3'^{-\tilde{r}_{d^b}}$
    \item[$\tilde{\pi}_{4,2} = $] $\pi_{4,1}' \cdot d^{\tilde{s}_{\Delta_3'}} \cdot C_3’^{-\tilde{s}_{d^b}}$
    \item[$\tilde{\pi}_{4,3} = $] $\pi_{4,1}' \cdot d^{\tilde{t}_{\Delta_3’}} \cdot C_3'^{-\tilde{t}_{d^b}}$
    \end{itemize}

    %5
  \item $e(\boxed{\Delta_1},d) = e(f, \boxed{R_1})$:
    
    The Verification equation is:  $E(\vec{C}_{\Delta_1'}, d) = E(f, \vec{C}_{R_1’}) \cdot E(\pi_{5,1}, \vec{g}_1)\cdot E(\pi_{5,2}, \vec{g}_2)\cdot E(\pi_{5,3}, \vec{g}_3)$
    
    with
    \begin{itemize}
    \item[$\pi_{5,1} = $] $d^{r_{\Delta_1'}} \cdot f^{-r_{R_1’}}$
    \item[$\pi_{5,2} = $] $d^{s_{\Delta_1’}} \cdot f^{-s_{R_1'}}$
    \item[$\pi_{5,3} = $] $d^{t_{\Delta_1'}} \cdot f^{-t_{R_1’}}$
    \end{itemize}

    The new proofs of the equations are:
    \begin{itemize}
    \item[$\tilde{\pi}_{5,1} = $] $\pi_{5,1}' \cdot d^{\tilde{r}_{\Delta_1'}} \cdot f^{-\tilde{r}_{R_1’}}$
    \item[$\tilde{\pi}_{5,2} = $] $\pi_{5,2}' \cdot d^{\tilde{s}_{\Delta_1’}} \cdot f^{-\tilde{s}_{R_1'}}$
    \item[$\tilde{\pi}_{5,3} = $] $\pi_{5,3}' \cdot d^{\tilde{t}_{\Delta_1'}} \cdot f^{-\tilde{t}_{R_1’}}$
    \end{itemize}

    %6
  \item $e(\boxed{\Delta_2'},d) = e(f, \boxed{R_2’})$:

    The Verification equation is:  $E(\vec{C}_{\Delta_2'}, h) = E(f, \vec{C}_{R_2'}) \cdot E(\pi_{6,1}, \vec{g}_1)\cdot E(\pi_{6,2}, \vec{g}_2)\cdot E(\pi_{6,3}, \vec{g}_3)$

    with
    \begin{itemize}
    \item[$\pi_{6,1} = $] $d^{r_{\Delta_2'}} \cdot f^{-r_{R_2'}}$
    \item[$\pi_{6,2} = $] $d^{s_{\Delta_2'}} \cdot f^{-s_{R_2'}}$
    \item[$\pi_{6,3} = $] $d^{t_{\Delta_2'}} \cdot f^{-t_{R_2'}}$
    \end{itemize}

    The new proofs of the equations are:
    \begin{itemize}
    \item[$\tilde{\pi}_{6,1} = $] $\pi_{6,1}' \cdot d^{\tilde{r}_{\Delta_2'}} \cdot f^{-\tilde{r}_{R_2'}}$
    \item[$\tilde{\pi}_{6,2} = $] $\pi_{6,2}' \cdot d^{\tilde{s}_{\Delta_2'}} \cdot f^{-\tilde{s}_{R_2'}}$
    \item[$\tilde{\pi}_{6,3} = $] $\pi_{6,3}' \cdot d^{\tilde{t}_{\Delta_2'}} \cdot f^{-\tilde{t}_{R_2'}}$
    \end{itemize}

    %7
  \item $e(\boxed{\Delta_3'},d) \cdot e(\boxed{M}, d^{-1})= e(\boxed{R_1'}, h) \cdot e(\boxed{R_2'}, h)$:
    
    The Verification equation is:  $E(\vec{C}_{\Delta_3'}, d) \cdot E(\vec{C}_M, d^{-1})= E(\vec{C}_{R_1'}, h) \cdot E(\vec{C}_{R_2'}, h) \cdot E(\pi_{7,1}, \vec{g}_1)\cdot E(\pi_{7,2}, \vec{g}_2)\cdot E(\pi_{7,3}, \vec{g}_3)$
    
    with
    \begin{itemize}
    \item[$\pi_{7,1} = $] $d^{r_{\Delta_3'}} \cdot d^{-r_M} \cdot h^{-r_{R_1'}} \cdot h^{-r_{R_2'}}$
    \item[$\pi_{7,2} = $] $d^{s_{\Delta_3'}} \cdot d^{-s_M} \cdot h^{-s_{R_1'}} \cdot h^{-s_{R_2'}}$
    \item[$\pi_{7,3} = $] $d^{t_{\Delta_3'}} \cdot d^{-t_M} \cdot h^{-t_{R_1'}} \cdot h^{-t_{R_2'}}$
    \end{itemize}   

    The new proofs of the equations are:
    \begin{itemize}
    \item[$\tilde{\pi}_{7,1} = $] $\pi_{7,1}' \cdot d^{\tilde{r}_{\Delta_3'}} \cdot d^{-\tilde{r}_M} \cdot h^{-\tilde{r}_{R_1'}} \cdot h^{-\tilde{r}_{R_2'}}$
    \item[$\tilde{\pi}_{7,2} = $] $\pi_{7,2}' \cdot d^{\tilde{s}_{\Delta_3'}} \cdot d^{-\tilde{s}_M} \cdot h^{-\tilde{s}_{R_1'}} \cdot h^{-\tilde{s}_{R_2'}}$
    \item[$\tilde{\pi}_{7,3} = $] $\pi_{7,3}' \cdot d^{\tilde{t}_{\Delta_3'}} \cdot d^{-\tilde{t}_M} \cdot h^{-\tilde{t}_{R_1'}} \cdot h^{-\tilde{t}_{R_2'}}$
    \end{itemize}

    
    %8
  \item $e(C_1'/\boxed{\Delta_1'}, d) = e (\boxed{D_1},d) \cdot e(f^{-1}, \boxed{S_1'})$

    The Verification equation is: $ E(\iota(C_1')/\vec{C}_{\Delta_1'}, d) = E(\vec{C}_{D_1},d) \cdot E(f^{-1}, \vec{C}_{S_1'})\cdot E(\pi_{8,1}, \vec{g}_1)\cdot E(\pi_{8,2}, \vec{g}_2)\cdot E(\pi_{8,3}, \vec{g}_3)$

    with
    \begin{itemize}
    \item[$\pi_{8,1} = $] $d^{-r_{\Delta_1'}} \cdot d^{-r_{D_1}} \cdot f^{r_{S_1'}}$
    \item[$\pi_{8,2} = $] $d^{-s_{\Delta_1'}} \cdot d^{-s_{D_1}} \cdot f^{s_{S_1'}}$
    \item[$\pi_{8,3} = $] $d^{-t_{\Delta_1'}} \cdot d^{-t_{D_1}} \cdot f^{t_{S_1'}}$
    \end{itemize}

    The new proofs of the equations are:
    
    \begin{itemize}
    \item[$\tilde{\pi}_{8,1} = $] $\pi_{8,1}' \cdot d^{-\tilde{r}_{\Delta_1'}} \cdot d^{-\tilde{r}_{D_1}} \cdot f^{\tilde{r}_{S_1'}}$
    \item[$\tilde{\pi}_{8,2} = $] $\pi_{8,2}' \cdot d^{-\tilde{s}_{\Delta_1'}} \cdot d^{-\tilde{s}_{D_1}} \cdot f^{\tilde{s}_{S_1'}}$
    \item[$\tilde{\pi}_{8,3} = $] $\pi_{8,3}' \cdot d^{-\tilde{t}_{\Delta_1'}} \cdot d^{-\tilde{t}_{D_1}} \cdot f^{\tilde{t}_{S_1'}}$
    \end{itemize}

    
    %9
  \item $e(C_2'/\boxed{\Delta_2'}, d) = e (\boxed{D_2},d) \cdot e(g^{-1}, \boxed{S_2'})$

    The Verification equation is: $ E(\iota(C_2')/\vec{C}_{\Delta_2'}, d) = E(\vec{C}_{D_2}, d) \cdot E(g^{-1}, \vec{C}_{S_2'})\cdot E(\pi_{9,1}, \vec{g}_1)\cdot E(\pi_{9,2}, \vec{g}_2)\cdot E(\pi_{9,3}, \vec{g}_3)$

    with
    \begin{itemize}
    \item[$\pi_{9,1} = $] $d^{-r_{\Delta_2'}} \cdot d^{-r_{D_2}} \cdot g^{r_{S_2'}}$
    \item[$\pi_{9,2} = $] $d^{-s_{\Delta_2'}} \cdot d^{-s_{D_2}} \cdot g^{s_{S_2'}}$
    \item[$\pi_{9,3} = $] $d^{-t_{\Delta_2'}} \cdot d^{-t_{D_2}} \cdot g^{t_{S_2'}}$
    \end{itemize}

    The new proofs of the equations are:
    
    \begin{itemize}
    \item[$\tilde{\pi}_{9,1} = $] $\pi_{9,1}' \cdot d^{-\tilde{r}_{\Delta_2'}} \cdot d^{-\tilde{r}_{D_2}} \cdot g^{-\tilde{r}_{S_2'}}$
    \item[$\tilde{\pi}_{9,2} = $] $\pi_{9,2}' \cdot d^{-\tilde{s}_{\Delta_2'}} \cdot d^{-\tilde{s}_{D_2}} \cdot g^{-\tilde{s}_{S_2'}}$
    \item[$\tilde{\pi}_{9,3} = $] $\pi_{9,3}' \cdot d^{-\tilde{t}_{\Delta_2'}} \cdot d^{-\tilde{t}_{D_2}} \cdot g^{-\tilde{t}_{S_2'}}$
    \end{itemize}

    %10
  \item $e(C_3'/\boxed{\Delta_3'}, d) = e (\boxed{D_3},d) \cdot e(\boxed{S_1'}, h^{-1}) \cdot e(\boxed{S_2'}, h^{-1})$

    The Verification equation is: $ E(\iota(C_3')/\vec{C}_{\Delta_3'}, d) = E(\vec{C}_{D_3},d) \cdot E(h^{-1}, \vec{C}_{S_1'}) \cdot E(h^{-1}, \vec{C}_{S_2'}) \cdot E(\pi_{10,1}, \vec{g}_1)\cdot E(\pi_{10,2}, \vec{g}_2)\cdot E(\pi_{10,3}, \vec{g}_3)$

    with
    \begin{itemize}
    \item[$\pi_{10,1} = $] $d^{-r_{\Delta_3'}} \cdot d^{-r_{D_3}} \cdot h^{r_{S_1'}} \cdot h^{r_{S_2'}}$
    \item[$\pi_{10,2} = $] $d^{-s_{\Delta_3'}} \cdot d^{-s_{D_3}} \cdot h^{s_{S_1'}} \cdot h^{s_{S_2'}}$
    \item[$\pi_{10,3} = $] $d^{-t_{\Delta_3'}} \cdot d^{-t_{D_3}} \cdot h^{t_{S_1'}} \cdot h^{t_{S_2'}}$
    \end{itemize}

    The new proofs of the equations are:

    \begin{itemize}
    \item[$\tilde{\pi}_{10,1} = $] $\pi_{10,1}' \cdot d^{-\tilde{r}_{\Delta_3'}} \cdot d^{-\tilde{r}_{D_3}} \cdot h^{\tilde{r}_{S_1'}} \cdot h^{\tilde{r}_{S_2'}}$
    \item[$\tilde{\pi}_{10,2} = $] $\pi_{10,1}' \cdot d^{-\tilde{s}_{\Delta_3'}} \cdot d^{-\tilde{s}_{D_3}} \cdot h^{\tilde{s}_{S_1'}} \cdot h^{\tilde{s}_{S_2'}}$
    \item[$\tilde{\pi}_{10,3} = $] $\pi_{10,1}' \cdot d^{-\tilde{t}_{\Delta_3'}} \cdot d^{-\tilde{t}_{D_3}} \cdot h^{\tilde{t}_{S_1'}} \cdot h^{\tilde{t}_{S_2'}}$
    \end{itemize}
    
    
    %11
  \item $e(\alpha, d/\boxed{d^b}) = e(g_z, \boxed{\Sigma_1}) \cdot e(g_r, \boxed{\Sigma_2}) \cdot \prod_{i=1}^3 e(g_i, \boxed{D_i})$

    The Verification equation is: $E(a, \iota(h)/\vec{C}_{b}) = E(g_z, \vec{C}_{\Sigma_1}) \cdot E(g_r, \vec{C}_{\Sigma_2}) \cdot \prod_{i=1}^3 E(g_i, \vec{C}_{D_i}) \cdot E(\pi_{12,1}, \vec{g}_1)\cdot E(\pi_{12,2}, \vec{g}_2)\cdot E((\pi_{12,3}), \vec{g}_3)$

    with
    \begin{itemize}
    \item[$\pi_{12,1} = $] $\alpha^{-r_{d^b}} \cdot g_z^{-r_{\Sigma_1}} \cdot g_r^{-r_{\Sigma_2}} \cdot \prod_{i=1}^3 g_i^{-r_{D_i}}$
    \item[$\pi_{12,2} = $] $\alpha^{-s_{d^b}} \cdot g_z^{-s_{\Sigma_1}} \cdot g_r^{-s_{\Sigma_2}} \cdot \prod_{i=1}^3 g_i^{-s_{D_i}}$
    \item[$\pi_{12,3} = $] $\alpha^{-t_{d^b}} \cdot g_z^{-t_{\Sigma_1}} \cdot g_r^{-t_{\Sigma_2}} \cdot \prod_{i=1}^3 g_i^{-t_{D_i}}$
    \end{itemize}

    The new proofs of the equations are:
    \begin{itemize}
    \item[$\tilde{\pi}_{12,1} = $] $\pi_{12,1} \cdot \alpha^{-\tilde{r}_{d^b}} \cdot g_z^{-\tilde{r}_{\Sigma_1}} \cdot g_r^{-\tilde{r}_{\Sigma_2}} \cdot \prod_{i=1}^3 g_i^{-\tilde{r}_{D_i}}$
    \item[$\tilde{\pi}_{12,2} = $] $\pi_{12,2} \cdot \alpha^{-\tilde{s}_{d^b}} \cdot g_z^{-\tilde{s}_{\Sigma_1}} \cdot g_r^{-\tilde{s}_{\Sigma_2}} \cdot \prod_{i=1}^3 g_i^{-\tilde{s}_{D_i}}$
    \item[$\tilde{\pi}_{12,3} = $] $\pi_{12,3} \cdot \alpha^{-\tilde{t}_{d^b}} \cdot g_z^{-\tilde{t}_{\Sigma_1}} \cdot g_r^{-\tilde{t}_{\Sigma_2}} \cdot \prod_{i=1}^3 g_i^{-\tilde{t}_{D_i}}$
    \end{itemize}

    %12
  \item $e(\beta, d/\boxed{d^b}) = e(h_z, \boxed{\Sigma_1}) \cdot e(h_u, \boxed{\Sigma_3}) \cdot \prod_{i=1}^3 e(h_i, \boxed{D_i})$

    The Verification equation is: $E(b, \iota(h)/\vec{C}_{b}) = E(h_z, \vec{C}_{\Sigma_1}) \cdot E(h_u, \vec{C}_{\Sigma_3}) \cdot \prod_{i=1}^3 E(h_i, \vec{C}_{D_i}) \cdot E(\pi_{13,1}, \vec{g}_1)\cdot E(\pi_{13,2}, \vec{g}_2)\cdot E(\pi_{13,3}, \vec{g}_3)$

    with
    \begin{itemize}
    \item[$\pi_{13,1} = $] $\beta^{-r_{d^b}} \cdot h_z^{-r_{\Sigma_1}} \cdot h_u^{-r_{\Sigma_3}} \cdot \prod_{i=1}^3 h_i^{-r_{D_i}}$
    \item[$\pi_{13,2} = $] $\beta^{-s_{d^b}} \cdot h_z^{-s_{\Sigma_1}} \cdot h_u^{-s_{\Sigma_3}} \cdot \prod_{i=1}^3 h_i^{-s_{D_i}}$
    \item[$\pi_{13,3} = $] $\beta^{-t_{d^b}} \cdot h_z^{-t_{\Sigma_1}} \cdot h_u^{-t_{\Sigma_3}} \cdot \prod_{i=1}^3 h_i^{-t_{D_i}}$
    \end{itemize}

    The new proofs of the equations are:
    \begin{itemize}
    \item[$\tilde{\pi}_{13,1} = $] $\pi_{13,1} \cdot \beta^{-\tilde{r}_{d^b}} \cdot h_z^{-\tilde{r}_{\Sigma_1}} \cdot h_u^{-\tilde{r}_{\Sigma_3}} \cdot \prod_{i=1}^3 h_i^{-\tilde{r}_{D_i}}$
    \item[$\tilde{\pi}_{13,2} = $] $\pi_{13,1} \cdot \beta^{-\tilde{s}_{d^b}} \cdot h_z^{-\tilde{s}_{\Sigma_1}} \cdot h_u^{-\tilde{s}_{\Sigma_3}} \cdot \prod_{i=1}^3 h_i^{-\tilde{s}_{D_i}}$
    \item[$\tilde{\pi}_{13,3} = $] $\pi_{13,1} \cdot \beta^{-\tilde{t}_{d^b}} \cdot h_z^{-\tilde{t}_{\Sigma_1}} \cdot h_u^{-\tilde{t}_{\Sigma_3}} \cdot \prod_{i=1}^3 h_i^{-\tilde{t}_{D_i}}$
    \end{itemize}

    %13
  \item \label{relationhd}
    $e(\boxed{d^b},h) = e(\boxed{h^b}, d)$: 

    The Verification equation is: $E(\vec{C}_b, h) = E(\vec{C}_{h^b}, d) \cdot E(\pi_{13,1}, \vec{g}_1)\cdot E(\pi_{13,2}, \vec{g}_2)\cdot E(\pi_{13,3}, \vec{g}_3)$

    with
    \begin{itemize}
    \item[$\pi_{13,1} = $] $h^{r_{d^b}} \cdot d^{-r_{h^b}}$
    \item[$\pi_{13,2} = $] $h^{s_{d^b}} \cdot d^{-s_{h^b}}$
    \item[$\pi_{13,3} = $] $h^{t_{d^b}} \cdot d^{-t_{h^b}}$
    \end{itemize}

    We can rerandomize this proof using the following formulas:

    \begin{itemize}
    \item[$\tilde{\pi}_{13,1} = $] $h^{r_{d^b}+\tilde{r}_{d^b}} \cdot d^{r_{h^b}+\tilde{r}_{h^b}} = \pi_{13,1} \cdot h^{\tilde{r}_{d^b}} \cdot d^{\tilde{r}_{h^b}}$
    \item[$\tilde{\pi}_{13,2} = $] $h^{s_{d^b}+\tilde{s}_{d^b}} \cdot d^{s_{h^b}+\tilde{s}_{h^b}} = \pi_{13,2} \cdot h^{\tilde{s}_{d^b}} \cdot d^{\tilde{s}_{h^b}}$
    \item[$\tilde{\pi}_{13,3} = $] $h^{t_{d^b}+\tilde{t}_{d^b}} \cdot d^{t_{h^b}+\tilde{t}_{h^b}} = \pi_{13,3} \cdot h^{\tilde{t}_{d^b}} \cdot d^{\tilde{t}_{h^b}}$
    \end{itemize}


    %14  
  \item \label{relationfd}
    $e(\boxed{F},d) = e(f, \boxed{d^b})$

    The Verification equation is: $E(\vec{C}_F, d) = E(f, \vec{C}_{d^b}) \cdot E(\pi_{14,1}, \vec{g}_1)\cdot E(\pi_{14,2}, \vec{g}_2)\cdot E(\pi_{14,3}, \vec{g}_3)$

    with

    \begin{itemize}
    \item[$\pi_{14,1} = $] $d^{r_F} \cdot f^{-r_{d^b}}$    
    \item[$\pi_{14,2} = $] $d^{s_F} \cdot f^{-s_{d^b}}$    
    \item[$\pi_{14,3} = $] $d^{t_F} \cdot f^{-t_{d^b}}$
    \end{itemize}

    The new proofs of the equations are:
    \begin{itemize}
    \item[$\tilde{\pi}_{14,1} = $] $\tilde{\pi}_{14,1} \cdot d^{\tilde{r}_F} \cdot f^{-\tilde{r}_{d^b}}$    
    \item[$\tilde{\pi}_{14,2} = $] $\tilde{\pi}_{14,2} \cdot d^{\tilde{s}_F} \cdot f^{-\tilde{s}_{d^b}}$    
    \item[$\tilde{\pi}_{14,3} = $] $\tilde{\pi}_{14,3} \cdot d^{\tilde{t}_F} \cdot f^{-\tilde{t}_{d^b}}$
    \end{itemize}


    %15
  \item \label{relationgd}
    $e(\boxed{G},d) = e(g, \boxed{d^b})$

    The Verification equation is: $E(\vec{C}_G, d) = E(g, \vec{C}_{d^b}) \cdot E(\pi_{15,1}, \vec{g}_1)\cdot E(\pi_{15,2}, \vec{g}_2)\cdot E(\pi_{15,3}, \vec{g}_3)$

    with

    \begin{itemize}
    \item[$\pi_{15,1} = $] $d^{r_G} \cdot g^{-r_{d^b}}$    
    \item[$\pi_{15,1} = $] $d^{s_G} \cdot g^{-s_{d^b}}$    
    \item[$\pi_{15,1} = $] $d^{t_G} \cdot g^{-t_{d^b}}$
    \end{itemize}

    The new proofs of the equations are:
    
    \begin{itemize}
    \item[$\tilde{\pi}_{15,1} = $] $\pi_{15,1} \cdot d^{\tilde{r}_G} \cdot g^{-\tilde{r}_{d^b}}$    
    \item[$\tilde{\pi}_{15,1} = $] $\pi_{15,1} \cdot d^{\tilde{s}_G} \cdot g^{-\tilde{s}_{d^b}}$    
    \item[$\tilde{\pi}_{15,1} = $] $\pi_{15,1} \cdot d^{\tilde{t}_G} \cdot g^{-\tilde{t}_{d^b}}$
    \end{itemize}

    
  \end{enumerate}
\end{description}

\subsection{Efficiency}
From the efficiency point of view, there are in total 108 group elements.

For the efficientcy reason, we can replace
\begin{itemize}
\item $(\vec{\pi}_3,\vec{\pi}_9)$ by $(\vec{\pi}_3 \cdot \vec{\pi}_9)$
\item $(\vec{\pi}_4,\vec{\pi}_{10})$ by $(\vec{\pi}_4 \cdot\vec{\pi}_{10})$
\item $(\vec{\pi}_5,\vec{\pi}_{11})$ by $(\vec{\pi}_5 \cdot\vec{\pi}_{11})$
\item $(\vec{\pi}_6,\vec{\pi}_{12})$ by $(\vec{\pi}_6 \cdot\vec{\pi}_{12})$
\item $(\vec{\pi}_7,\vec{\pi}_{13})$ by $(\vec{\pi}_7 \cdot\vec{\pi}_{13})$
\end{itemize}

Thus we can reduce the number of group elements down to $108-5 \cdot 3 = 93$.

\begin{enumerate}
\item[$\tilde{\vec{\pi}}^{2,8}$]:
  \begin{enumerate}
  \item $\tilde{\pi}^{2,8}_1 = \pi^{2,8}_1 \cdot \pi_{14,1}^{\theta_1'} \cdot d^{\tilde{r}_{\Delta_1'}} \cdot C_1'^{-\tilde{r}_{d^b}} \cdot \pi_{14,1}^{-\theta_1'} \cdot d^{-\tilde{r}_{\Delta_1'}} \cdot d^{-\tilde{r}_{D_1}} \cdot f^{\tilde{r}_{S_1'}} = \pi^{2,8}_1 \cdot C_1'^{-\tilde{r}_{d^b}} \cdot d^{-\tilde{r}_{D_1}}$
  \item $\tilde{\pi}^{2,8}_2 = \pi^{2,8}_2 \cdot \pi_{14,2}^{\theta_1'} \cdot d^{\tilde{s}_{\Delta_1'}} \cdot C_1'^{-\tilde{s}_{d^b}} \cdot \pi_{14,2}^{-\theta_1'} \cdot d^{-\tilde{s}_{\Delta_1'}} \cdot d^{-\tilde{s}_{D_1}} \cdot f^{\tilde{s}_{S_1'}} = \pi^{2,8}_2 \cdot C_1'^{-\tilde{s}_{d^b}} \cdot d^{-\tilde{s}_{D_1}}$
  \item $\tilde{\pi}^{2,8}_3 = \pi^{2,8}_3 \cdot \pi_{14,3}^{\theta_1'} \cdot d^{\tilde{t}_{\Delta_1'}} \cdot C_1'^{-\tilde{t}_{d^b}} \cdot \pi_{14,3}^{-\theta_1'} \cdot d^{-\tilde{t}_{\Delta_1'}} \cdot d^{-\tilde{t}_{D_1}} \cdot f^{\tilde{t}_{S_1'}} = \pi^{2,8}_3 \cdot C_1'^{-\tilde{t}_{d^b}} \cdot d^{-\tilde{t}_{D_1}}$
  \end{enumerate}
\item[$\tilde{\vec{\pi}}^{3,9}$]:
  \begin{enumerate}
  \item $\tilde{\pi}^{3,9}_1 = \pi^{3,9}_1 \cdot \pi_{15,1}^{\theta_2'} \cdot d^{\tilde{r}_{\Delta_2'}} \cdot C_2'^{-\tilde{r}_{d^b}} \cdot \pi_{15,1}^{-\theta_2'} \cdot d^{-\tilde{r}_{\Delta_2'}} \cdot d^{-\tilde{r}_{D_2}} \cdot g^{-\tilde{r}_{S_2'}} = \pi^{3,9}_1 \cdot C_2^{-\tilde{r}_{d^b}} \cdot d^{-\tilde{r}_{D_2}} \cdot g^{-\tilde{r}_{S_2'}}$
  \item $\tilde{\pi}^{3,9}_2 = \pi^{3,9}_2 \cdot \pi_{15,2}^{\theta_2'} \cdot d^{\tilde{s}_{\Delta_2'}} \cdot C_2'^{-\tilde{s}_{d^b}} \cdot \pi_{15,2}^{-\theta_2'} \cdot d^{-\tilde{s}_{\Delta_2'}} \cdot d^{-\tilde{s}_{D_2}} \cdot g^{-\tilde{s}_{S_2'}} = \pi^{3,9}_2 \cdot C_2^{-\tilde{r}_{d^b}} \cdot d^{-\tilde{s}_{D_2}} \cdot g^{-\tilde{s}_{S_2'}}$
  \item $\tilde{\pi}^{3,9}_3 = \pi^{3,9}_3 \cdot \pi_{15,3}^{\theta_2'} \cdot d^{\tilde{t}_{\Delta_2'}} \cdot C_2'^{-\tilde{t}_{d^b}} \cdot \pi_{15,3}^{-\theta_2'} \cdot d^{-\tilde{t}_{\Delta_2'}} \cdot d^{-\tilde{t}_{D_2}} \cdot g^{-\tilde{t}_{S_2'}} = \pi^{3,9}_3 \cdot C_2^{-\tilde{r}_{d^b}} \cdot d^{-\tilde{t}_{D_2}} \cdot g^{-\tilde{t}_{S_2'}}$
  \end{enumerate}
\item[$\tilde{\vec{\pi}}^{4,10}$]:
  \begin{enumerate}
  \item $\tilde{\pi}^{4,10}_1 = \pi^{4,10}_1 \cdot \pi_{13,1}^{-(\theta_1'+\theta_2')} \cdot d^{\tilde{r}_{\Delta_3'}} \cdot C_3'^{-\tilde{r}_{d^b}} \cdot \pi_{13,1}^{\theta_1'+ \theta_2'} \cdot d^{-\tilde{r}_{\Delta_3'}} \cdot d^{-\tilde{r}_{D_3}} \cdot h^{\tilde{r}_{S_1'}} \cdot h^{\tilde{r}_{S_2}'} = \pi^{5,11}_1 \cdot C_3^{-\tilde{r}_{d^b}} \cdot d^{-\tilde{r}_{D_3}} \cdot h^{\tilde{r}_{S_1'}} \cdot h^{\tilde{r}_{S_2}'}$
  \item $\tilde{\pi}^{4,10}_2 = \pi^{4,10}_2 \cdot \pi_{13,2}^{-(\theta_1'+\theta_2')} \cdot d^{\tilde{s}_{\Delta_3'}} \cdot C_3'^{-\tilde{s}_{d^b}} \cdot \pi_{13,2}^{\theta_1'+ \theta_2'} \cdot d^{-\tilde{s}_{\Delta_3'}} \cdot d^{-\tilde{s}_{D_3}} \cdot h^{\tilde{s}_{S_1'}} \cdot h^{\tilde{s}_{S_2}'} = \pi^{5,11}_2 \cdot C_3^{-\tilde{s}_{d^b}} \cdot d^{-\tilde{s}_{D_3}} \cdot h^{\tilde{s}_{S_1'}} \cdot h^{\tilde{s}_{S_2}'}$
  \item $\tilde{\pi}^{4,10}_3 = \pi^{4,10}_3 \cdot \pi_{13,3}^{-(\theta_1'+\theta_2')} \cdot d^{\tilde{t}_{\Delta_3'}} \cdot C_3'^{-\tilde{t}_{d^b}} \cdot \pi_{13,3}^{\theta_1'+ \theta_2'} \cdot d^{-\tilde{t}_{\Delta_3'}} \cdot d^{-\tilde{t}_{D_3}} \cdot h^{\tilde{t}_{S_1'}} \cdot h^{\tilde{t}_{S_2}'} = \pi^{5,11}_3 \cdot C_3^{-\tilde{t}_{d^b}} \cdot d^{-\tilde{t}_{D_3}} \cdot h^{\tilde{t}_{S_1'}} \cdot h^{\tilde{t}_{S_2}'}$
  \end{enumerate}
\item[$\tilde{\vec{\pi}}^{5,11}$]:
  \begin{enumerate}
  \item $\tilde{\pi}^{5,11}_1 = \pi^{5,12}_1 \cdot \pi_{14,1}^{\theta_1'} \cdot d^{r_{\Delta_1'}} \cdot f^{-r_{R_1}'} \cdot \alpha^{-\tilde{r}_{d^b}} \cdot g_z^{-\tilde{r}_{\Sigma_1}} \cdot g_r^{-\tilde{r}_{\Sigma_2}}\cdot \prod_{i = 1}^3 g_i^{-\tilde{r}_{D_i}}$
  \item $\tilde{\pi}^{5,11}_2 = \pi^{5,12}_2 \cdot \pi_{14,2}^{\theta_1'} \cdot d^{s_{\Delta_1'}} \cdot f^{-s_{R_1}'} \cdot \alpha^{-\tilde{s}_{d^b}} \cdot g_z^{-\tilde{s}_{\Sigma_1}} \cdot g_r^{-\tilde{s}_{\Sigma_2}}\cdot \prod_{i = 1}^3 g_i^{-\tilde{s}_{D_i}}$
  \item $\tilde{\pi}^{5,11}_3 = \pi^{5,12}_3 \cdot \pi_{14,3}^{\theta_1'} \cdot d^{t_{\Delta_1'}} \cdot f^{-t_{R_1}'} \cdot \alpha^{-\tilde{t}_{d^b}} \cdot g_z^{-\tilde{t}_{\Sigma_1}} \cdot g_r^{-\tilde{t}_{\Sigma_2}}\cdot \prod_{i = 1}^3 g_i^{-\tilde{t}_{D_i}}$
  \end{enumerate}
\item[$\tilde{\vec{\pi}}^{6,12}$]:
  \begin{enumerate}
  \item $\tilde{\pi}^{6,12}_1 = \pi^{7,13}_1 \cdot \pi_{15,1}^{\theta_2'} \cdot d^{r_{\Delta_2'}} \cdot f^{-r_{R_2'}} \cdot \beta^{-\tilde{r}_{d^b}} \cdot h_z^{-\tilde{r}_{\Sigma_1}} \cdot h_u^{-\tilde{r}_{\Sigma_3}}\cdot \prod_{i = 1}^3 h_i^{-\tilde{r}_{D_i}}$
  \item $\tilde{\pi}^{6,12}_2 = \pi^{7,13}_2 \cdot \pi_{15,2}^{\theta_2'} \cdot d^{s_{\Delta_2'}} \cdot f^{-s_{R_2'}} \cdot \beta^{-\tilde{s}_{d^b}} \cdot h_z^{-\tilde{s}_{\Sigma_1}} \cdot h_u^{-\tilde{s}_{\Sigma_3}}\cdot \prod_{i = 1}^3 h_i^{-\tilde{s}_{D_i}}$
  \item $\tilde{\pi}^{6,12}_3 = \pi^{7,13}_3 \cdot \pi_{15,3}^{\theta_2'} \cdot d^{t_{\Delta_2'}} \cdot f^{-t_{R_2'}} \cdot \beta^{-\tilde{t}_{d^b}} \cdot h_z^{-\tilde{t}_{\Sigma_1}} \cdot h_u^{-\tilde{t}_{\Sigma_3}}\cdot \prod_{i = 1}^3 h_i^{-\tilde{t}_{D_i}}$
  \end{enumerate}
\end{enumerate}


  \end{section}
  
\begin{section}{Instantiation of Chase \etal based on the SXDH assumption}
  \label{chaseSXDH}
  To compare with our construction, we also give in this section an instantiation of Chase \etal~\cite{DBLP:conf/eurocrypt/ChaseKLM12} based on the SXDH assumption.

\begin{description}
\item[\boldmath$RCCA3.\Setup(\lambda)$:] This algorithm generates the public and secret keys of our RCCA encryption scheme.
  \begin{enumerate}
  \item Pick bilinear groups $(\G, \hat{\G}, \G_T)$ of prime order $p$ and the bilinear map $e$ on this pair of groups. We choose a group generator $f \in \G$ and a random value $x \sample \mathbb{Z}_p$, then compute $g = f^x$.
  \item Choose a random group generator $d \in \G$.
  \item Choose a random group generator $\hat{d} \in \hat{\G}$.
  \item Choose $g_1,g_2 \sample \G$ and $\hat{g}_1, \hat{g}_2 \in \hat{\G}$ then set $\vec{u}_1 = (g_1, g_2) \in \G^2$, $\vec{u}_2 \sample \G^2$, $\hat{\vec{u}}_1 = (\hat{g}_1, \hat{g}_2) \in \hat{\G}^2$ and $\hat{\vec{u}}_2 \sample \hat{\G}^2 $
  \item Set up the keys for the underlying encryption scheme $\pk_{enc} = (f,g)$ and $\sk_{enc} = (x)$.
  \item Choose two random group generators $(g_z, g_r)$ and $6$ random exponents $\{\chi_i, \gamma_i\}_{i = 1}^2, \zeta, \rho\sample \Z_p$, then compute $g_i = g_z^{\chi_i}g_r^{\gamma_i})$ for $ i \in \{1,2\}$ and $\alpha = g_z^\zeta g_r^\rho$
  \item Set up the keys for the underlying signature scheme
    $$\vk_{sig} = (g_z, g_r, g_1, g_2, \alpha)$$
    and
    $$\sk_{sig} = (\vk_{sig}, \zeta, \rho, \{\chi_i, \gamma_i\}_{i = 1}^2).$$
  \item $\PK = (d,\pk_{enc}, \vk_{sig}, \sigma_{crs})$.
  \item $\SK = (x)$
  \end{enumerate}

\item[\boldmath$RCCA3.\Enc(\PK,m)$:] This algorithm takes as input a message and the public key of the underlying encryption scheme, outputs the corresponding ciphertext of our RCCA encryption scheme.
  \begin{enumerate}
  \item Choose a random exponents $\theta \sample \Z_p$ and set $r = \theta$.
  \item Compute the ciphertext $\vec{C} = (C_1, C_2)$:
    \begin{align*}
      C_1 &= f^{\theta} & C_2 &= m \cdot g^{\theta}
    \end{align*}

  \item Prove the knowledge of the witness $\vec{w} = (m, r, \vec{D}, S, \vec{\sigma})$ which verifies that
    \begin{align*}
      Enc_{BBS}(pk_{enc}, m; r) = \vec{C} \vee (\vec{C} = ReRand(\vec{D}; S) \wedge Verify(vk_{sig}, \vec{D}) = \True)
    \end{align*}
    
  \item Define the bit $b = 1$ and a commitment $\vec{C}_b = (1,d^b)\cdot \vec{g}_1^{r_b} \cdot \vec{g}_2^{s_b}$, $\hat{\vec{C}}_b = (1,\hat{d}^b) \cdot \hat{\vec{g}}_1^{r_{\hat{b}}} \cdot \hat{\vec{g}}_2^{s_{\hat{b}}}$ and also a $WI-NIZK$ proof $(\pi_{b,1}, \pi_{b,2}) \in (\G^2 \times \hat{\G}^2)^2$ of the pairing product equation 
  
  \begin{align}
    e(d, \boxed{\hat{d}^b}) &= e(\boxed{d^b}, \boxed{\hat{d}^b}) \tag{1}\\
    e(\boxed{d^b}, d) &= e(\boxed{d^b}, \boxed{\hat{d}^b}) \tag{2}
  \end{align}
  
  which ensures that $b \in \{0,1\}$.

    %  \item Then generate commitements $\{\vec{C}_{\Gamma_i}\}_{i = 1}^3$ of the variables $\Gamma_i = C_i^b$ for $i \in \{1,2,3\}$ and the corresponding proofs:
    %    \begin{align}
    %      e(C_i,\boxed{h^b}) &= e(h, \boxed{\Gamma_i}) ,&
    %      \forall i \in \{1,2,3\}
    %    \end{align}
  \item We first prove the left side of the OR statement, we generate commitments $(\vec{C}_{\hat{R}})$ of the variable $\hat{R} = \hat{d}^{\theta b}$ and $\vec{C}_{M}$ commitment of $M = m^b$ and commitments $\{\vec{C}_{\Delta_i}\}_{i=1}^2$ of the variables $\{\Delta_i = C_i^b\}_{i=1}^2$. Recall that $\{C_i\}_{i=1}^2$, $R$ and $\{\Delta_i\}_{i=1}^2$ verify the following equations:
    \begin{align}
      e(C_i, \boxed{\hat{d}^b}) &= e(\boxed{\Delta_i}, \hat{d})  &\forall i \in \{1,2\} \tag{3,4}\\
      e(\boxed{\Delta_1}, \hat{d}) &= e(f, \boxed{\hat{R}}) \tag{5}\\
      e(\boxed{\Delta_2}, \hat{d}) \cdot e(\boxed{M}, \hat{d}^{-1}) &= e(g, \boxed{\hat{R}}) \tag{6}
    \end{align}



    %signature part
  \item Then we prove the right side of the OR statement:
    \begin{enumerate}  
    \item The $ReRand$ component: define $(D_1, D_2) = (1_\G, 1_\G)$ and $S = 1_\G$.
    \item Remind that actually these variables are of the following forms in the security proof, but in the case $b=1$ they all become $1_\G$.
      $$(D_1, D_2, D_3) = (f^{(\theta_1+\theta_1')\cdot (1-b)}, m^{1-b} \cdot g^{(\theta_2+\theta_2')\cdot (1-b)}$$
      and $$S_1 = d^{\theta_1'\cdot (1-b)}$$
    \item Then compute the commitments $\{\vec{C}_{D_i}\}_{i=1}^2$ of $\{D_i\}_{i= 1}^2$ and $\vec{C}_{\hat{S}}$ commitments of $\hat{S}$.
    \item And also compute the proofs of following equations:
      \begin{align}
        e(C_1/\boxed{\Delta_1}, \hat{d}) &= e (\boxed{D_1}, \hat{d}) \cdot e(f^{-1}, \boxed{\hat{S}}) \tag{7}\\
        e(C_2/\boxed{\Delta_2}, \hat{d}) &= e (\boxed{D_2}, \hat{d}) \cdot e(g^{-1}, \boxed{\hat{S}}) \tag{8}
      \end{align}

    \item Then the signature component (Remind that $b = 1$): define
      $$\vec{\sigma}  = (\Sigma_1, \Sigma_2) = (z^{1-b}, r^{1-b}) = (1_\G, 1_\G),$$
      then compute their commitments $(\vec{C}_{\Sigma_1}, \vec{C}_{\Sigma_2})$.
    \item We generate the proof of the following linear pairing equations:
      \begin{align} 
        e(\alpha, d/\boxed{d^b}) &= e(g_z, \boxed{\Sigma_1}) \cdot e(g_r, \boxed{\Sigma_2}) \cdot \prod_{i=1}^3 e(g_i, \boxed{D_i}) \tag{9}
      \end{align}

    \end{enumerate}

  \item To allow the re-randomization of the ciphertext, we need to compute the commitments $\vec{C}_F$, $\vec{C}_G$ to the variables :

    \begin{align*}
    F &= f^b, & G&=g^b
    \end{align*}

    and their corresponding proofs:
    \begin{align}
      e(\boxed{F},d) &= e(f,\boxed{d^b}) & e(\boxed{G}, d) &= e(g, \boxed{d^b})\tag{10,11}
    \end{align}

    
  \item We put all these proofs together to get $\vec{\pi}$.
  \item The ciphertext of the RCCA-scheme is
    $$(\vec{C} = (C_1, C_2, C_3), \vec{C}_{d^b}, \vec{C}_{\hat{d}^b}, \vec{C}_{M}, \vec{C}_{\hat{R}}, \{\vec{C}_{D_i}\}_{i = 1}^2, \vec{C}_{\hat{S}}, \{\vec{C}_{\Sigma_i}\}_{i = 1}^2,\{\vec{C}_{\Delta_i}\}_{i=1}^2, \vec{C}_F, \vec{C}_G, \vec{\pi}) \in 49\G \times 20 \hat{\G}$$
    
  \end{enumerate}

\item[\boldmath{$RCCA3.\Dec(\PK,\SK, \vec{C})$}]:
  \begin{enumerate}
  \item Parse $\vec{C}$ as $(C_1, C_2, C_3), \vec{C}_{d^b}, \vec{C}_{\hat{d}^b}, \vec{C}_{M}, \vec{C}_{\hat{R}}, \{\vec{C}_{D_i}\}_{i = 1}^2, \vec{C}_{\hat{S}}, \{\vec{C}_{\Sigma_i}\}_{i = 1}^2,\{\vec{C}_{\Delta_i}\}_{i=1}^2, \vec{C}_F, \vec{C}_G, \vec{\pi})$.
  \item Parse $\SK$ as $x$
  \item Verify that all proofs are correct.
  \item If any proof fails then return $\bot$. otherwise return $C_2/(C_1^x)$ 
  \end{enumerate}

\item[\boldmath{$RCCA3.\Rerand(\PK, \vec{C}, )$}]:  
  For the randomization, we will proceed in two stages. Firstly we sample two random values $\theta' \gets \mathbb{Z}_p$. The new variables are $C_1' = C_1 \cdot f^{\theta}$ and $C_2' = C_2 \cdot h^{\theta'}$. Then using the proof of the equations $(10, 11)$, to adapt the new proofs corresponding to the new ciphertext instance $\vec{C}' = (C_1', C_2')$.

  %  For the randomization we first generate randomness for each commitment. For a variable $X$, we generate the new randomness $(\tilde{r}_X, \tilde{s}_X, \tilde{t}_X)$, the new commitment will be $\tilde{\vec{C}}_X = \iota(X) \cdot \vec{g}_1^{r_X+\tilde{r}_X} \cdot \vec{g}_2^{s_X+\tilde{s}_X} \cdot \vec{g}_3^{t_X+\tilde{t}_X}$.
  
For the second stage, we randomize all the commitments and the GS proofs without changing the ciphertext part $\vec{C}' = (C_1', C_2')$.

In this algorithm, for the variable $X$, we denote its commitment by $\vec{C}_X = (1, X) \cdot \vec{g}_1^{r_X} \cdot \vec{g}_2^{s_X}$ its new commitment from the first stage by $\vec{C}_X' = \vec{C}_X \cdot \vec{g}_1^{r'_X} \cdot \vec{g}_2^{s'_X}$ and denote the new randomness introduced in the second step by $(\tilde{r}_X, \tilde{s}_X)$.



\end{description}


%\input{proofs}


\end{section}

\begin{section}{Proof of the CCA2 security of the Public Verifiable Structure-Preserving Encryption scheme}
  \label{ProofSPPE}
  \begin{myTh}
  The scheme $SPPE$ provides IND-CCA2 security under the SXDH assumption.
\end{myTh}


\begin{proof}

  This will be a game based proof. From the first game which is the definition of the CCA2 security game to the last game, in which the adversary can trivially not have any advantage. In the $i$-th game, we define the advantage of the adversary by $S_i$.

  \begin{description}
  \item[\textsf{Game} $0$ :] This is the real game, the adversary is against the CCA2 security game. We give the adversary the public key $\PK$ of the encryption scheme which contains the proof vectors $(\vec{u}_1, \vec{u}_2)$ which verifies:
    \begin{align*}
      \vec{u}_1 &= (\hat{g}, \hat{h}) \in \hat{\G}^2\\
      \vec{u}_2 &= (\hat{g}^{\rho_u}, \hat{h}^{\rho_u'}) \in \hat{\G}^2
    \end{align*}

    The adversary has the access to the decryption oracle.
    
    Then during the challenge phase, the adversary chooses two messages $(m_0, m_1) \in \G^2$, then submits to the Challenger and obtains a challenge ciphertext
    \begin{align*}
      \vec{C}^* = (\SVK^*, \com^*, \open^*, C_0^*, C_1^*, C_2^*, \vec{C}_{\theta}^*, \vec{\pi}^*, \vec{\sigma}^*)
    \end{align*}
    especially we have:
    \begin{align*}
      C_0^* &= m_b \cdot X^{\theta^*} & C_1^* &= g_1^{\theta^*} & C_2^* &= g_2^{\theta^*}.
    \end{align*}

    The adversary has access to the decryption oracle except the ciphertext $\vec{C}^*$ after the challenge phase.

    At the end, the adversary outputs a bit $b'$, its advantage against the RCCA security game is defined by the winning probability of $S_0 = |\PR[b' = b] - \frac{1}{2}|$.

  \item[\textsf{Game} $1$ :] In this game, the challenger generates the signing key and verification key pair and the commitment $\com^*$ of the signature's verification key at the beginning of the game. Since this does not change the view of the adversary then we have $S_0 = S_1$.

  \item[\textsf{Game} $2$ :] In the $\KeyGen$ algorithm, we modify the generation of the public key. Instead of generate $(\vec{u}_1, \vec{u}_2)$ as in the \textsf{Game} 0, we define:
    \begin{align*}
      \vec{u}_1 &= (\hat{g}, \hat{h}) \in \hat{\G}^2\\
      \vec{u}_2 &= (\hat{g}^{\rho_u}, \hat{h}^{\rho_u'}) \cdot (1, (\com^*)^{-1}) \in \hat{\G}^2
    \end{align*}

    Since $\vec{u}_2$ is always distributed uniformly over $\hat{\G}^2$, this change does not affect the view of the adversary. Thus we have $S_2 = S_1$.


  \item[\textsf{Game} $3$ :] In this game, we define a failure event $F_4$: the ciphertext submitted by the adversary to the decryption oracle during the first query phase(before the challenge phase) contains the commitment $\com$ which verifies $\com = \com^*$.

    If the event $F_3$ happens, then the experiment halts and outputs a random bit. Since $\com^*$ is chosen uniformly in the space $\hat{\G}$, and remains independent from the adversary's view until the challenge phase, then we have $|S_3 - S_2| \leq \PR[F_3] \leq q_D/p$ where $q_D$ represents the number of decryption queries before the challenge phase and $p$ is the order of the group $\hat{\G}$.

  \item[\textsf{Game} $4$ :] In this game, we modify the decryption oracle during the second query phase, let us denote the event $F_4$: the ciphertext submitted by the adversary to the decryption oracle during the second query phase contains the commitment and the open $(\com, \open)$ which verifies $\com = \com^*$ but $\open \neq \open^*$.

    The experiment halts if $F_4$ occurs and outputs a random bit. Thus we have $|S_4- S_3| \leq \PR[F_4]$.

    And if $F_4$ occurs, we can easily construct an adversary $\mathcal{B}$ of the target collision-resistance of the underlying structure-preserving commitment $SPC$ which contradicts the Double Pairing assumption.

    Thus we have $|S_4 - S_3| \leq \PR[F_4] \leq adv_{\mathcal{B}}^{TCR-CR}(\lambda) \leq adv_{\mathcal{B}}^{DP}(\lambda)$.

  \item[\textsf{Game} $6$ :] We modify again the decryption oracle during the second query phase.
    During the second query phase, the ciphertext submitted by the adversary contains $(\com, \open)$ such that $\com = \com^*$ and $\open = \open^*$ but $(C_0, C_1, C_2, \pi_1, \pi_2) \neq (C_0^*, C_1^*, C_2^*, \pi_1^*, \pi_2^*)$, we denote this event $F_6$. If $F_6$ occurs, the experiment halts and outputs a random bit. Thus we have $|S_6 - S_5| \leq \PR[F_6]$. The event $F_6$ is contradict the strong unforgeability of the underlying one-time signature. Thus we have $|S_6 - S_5| \leq adv_{\mathcal{B}}^{SUF-OTS}(\lambda) \leq adv_{\mathcal{B}}^{DP}(\lambda)$.
    

  \item[\textsf{Game} $6$ :] We modify again the decryption oracle. For the decryption query
    $$\vec{C} = (\SVK, \com, \open, C_0, C_1, C_2, \vec{C}_{\theta}, \vec{\pi}, \vec{\sigma}),$$
    let us denote the event $F_6$ as the following condition happens but $F_5$ does not occur:
    \begin{eqnarray} \label
      (\com,\open)=(\com^*,\open^*)  \quad \wedge \quad (C_0,C_1,C_2,\pi_1,\pi_2) = (C_0^*,C_1^*,C_2^*,\pi_1^*,\pi_2^*) 
      \quad \wedge \quad \vec{C}_{\theta} \neq \vec{C}_{\theta}^* .
    \end{eqnarray}

    We have $|S_6- S_5| = \PR[F_6]$. We now prove that $\PR[F_6] = 0$. For a given $(C_1^*,C_2^*,\pi_1^*,\pi_2^*) \in \G^4$, there exists only one commitment $\vec{C}_{\theta}^* \in \hat{\G}^2$ that satisfies the equalities:
    \begin{eqnarray*}
      E(g_1,\vec{C}_{\theta}) &=& E(C_1 , \vec{u}_{\com}) \cdot E(\pi_1,\vec{u}_1) \\
      E(g_2,\vec{C}_{\theta}) &=& E(C_2 , \vec{u}_{\com}) \cdot E(\pi_2,\vec{u}_1)
    \end{eqnarray*}
    Because the two equations give us the equalities:
    \begin{eqnarray*}  
      E(g_1,\vec{C}_{\theta}^\star) &=& E(g_1^{\theta^\star} , \vec{u}_{\com}) \cdot E(g_1^{r^\star},\vec{u}_1) = E(g_1 , \vec{u}_{\com}^{\theta^\star}) \cdot E(g_1,\vec{u}_1^{r^\star}) \\
      E(g_2,\vec{C}_{\theta}^\star) &=& E(g_2^{\theta^\star} , \vec{u}_{\com}) \cdot E(g_2^{r^\star},\vec{u}_1) =E(g_2 , \vec{u}_{\com}^{\theta^\star}) \cdot E(g_2,\vec{u}_1^{r^\star})
    \end{eqnarray*}

    Thus we have $|S_6- S_5| = \PR[F_6] = 0$
    
    
  \item[\textsf{Game} $7$ :] In this game, we modify the distribution of the public keys. We compute the public keys $(\vec{u}_1, \vec{u}_2)$ in the following way:
    \begin{align*}
      \vec{u}_1 &= (\hat{g}, \hat{h}) \in \hat{\G}^2\\
      \vec{u}_1 &= (\hat{g}^{\rho_u}, \hat{h}^{\rho_u}) \cdot (1, \com^{-1}) \in \hat{\G}^2
    \end{align*}

    Since the Challenger does not use $\rho_u$ or $\rho_u'$ in the security game, thus an adversary who can make difference between \textsf{Game} $6$ and \textsf{Game} $7$, is an adversary against the DDH assumption in $\hat{\G}$. Thus we have $|S_7 - S_6| \leq adv_{\mathcal{B}}^{DDH}(\lambda)$.

  \item[\textsf{Game} $8$ :] In this game, instead of generate the proof $(\vec{\pi}_{\theta^*}, \vec{\pi}_{\theta^*})$ using the witness $\theta^*$, we generate a random value $r \sample \mathbb{Z}_p^*$ and generate $(\vec{C}_{\theta^*}, \vec{\pi}_{\theta^*}, \vec{\pi}_{\theta^*})$ as following:
    \begin{align*}
      \vec{C}_{\theta^*} &= \vec{u}_1^r & \vec{\pi}_{\theta^*} &= g_1^r \cdot C_1^{*-\rho_u} & \vec{\pi}_{\theta^*} &= g_2^r \cdot C_2^{*-\rho_u}
    \end{align*}

    Notice that even the proof elements are generated without using the witness $\theta^* = log_{g_1}(C_1^*) = log_{g_2}(C_2^*)$, the distribution of the proof is still remain the same as in the original proof. In fact, let us define $\tilde{r} = r - \rho_u \cdot \theta^*$, we have:
    \begin{align*}
      \vec{C}_{\theta}^* &= \vec{u}_{\com}^{\theta^*} \cdot \vec{u}_1^{\tilde{r}} &  \vec{\pi}_{\theta^*,1,1} &= g_1^{\tilde{r}} & \vec{\pi}_{\theta^*,2,1} &= g_2^{\tilde{r}}
    \end{align*}

    Thus we have $S_8 = S_7$.

  \item[\textsf{Game} $9$ :] In this game, we modify the ciphertext generation in the challenge phase. Instead of compute the ciphertext using the public key $(X, g_1, g_2)$, we generate it with the secret key $(x_1, x_2)$:
    \begin{align*}
      C_1^* &= g_1^{\theta^*} & C_2^* &= g_2^{\theta^*} & C_0 &= M_b\cdot C_1^{*x_1} \cdot C_2^{*x_2}  
    \end{align*}
    Since the ciphertext remains exactly the same as in the \textsf{Game} $8$. Thus this modification does not change the view of the adversary, which means $S_9 = S_8$.


  \item[\textsf{Game} $10$ :] In this game, we modify again the ciphertext generation in the challenge phase. Recall that since the \textsf{Game} $9$, we don't use anymore $\theta^{*}$ to generate $(\vec{C}_{\theta^*}, \vec{\pi}_{\theta^*}, \vec{\pi}_{\theta^*})$, then we generate two random values $(\theta_1, \theta_2) \sample \mathbb{Z}_p^2$ and compute the ciphertext as following:
    \begin{align*}
      C_1 &= g_1^{\theta_1} & C_2 &= g_2^{\theta_2} & C_0 &= m_b \cdot C_1^{x_1} \cdot C_2^{x_2}
    \end{align*}

    As we don't use anymore $\theta_1$ nor $\theta_2$ in the whole game, we can easily construct a reduction from an adversary who can make difference between \textsf{Game} $10$ and \textsf{Game} $9$ to an adversary against the DDH assumption. Thus we have $|S_{10} - S_9| \leq adv_{\mathcal{B}}^{DDH}(\lambda)$.
    
  \end{description}

  Notice that in the final game, the ciphertext component is as follows:
  \begin{align*}
    C_1 &= g_1^{\theta^*} & C_2 &= g_2^{\theta^*+ \theta'} & C_0 &= m_b \cdot X_1^{\theta^*} \cdot g_2^{x_2 \cdot \theta'}
  \end{align*}

  As $x_2$ is completely independent of the adversary's view, $C_0$ can be seen as a one-time pad of the message $m_b$. Thus the adversary does not have any information about the bit $b$. Then we have $S_{10} = 0$.
For summary, we have
\begin{align*}
  adv_{\mathcal{A}}^{CCA2}(\lambda) &= S_0\\
  &\leq S_{10} + 2 \cdot adv_{\mathcal{B}}^{DDH}(\lambda) + 2 \cdot adv_{\mathcal{B}}^{DP}(\lambda) + q_D/p\\
  &= 2 \cdot adv_{\mathcal{B}}^{DDH}(\lambda) + 2 \cdot adv_{\mathcal{B}}^{DP}(\lambda) + q_D/p \in negl(\lambda)
\end{align*}

\end{proof}

\end{section}

\end{appendices}


\end{document}




