For the simplicity, we choose the symmetric setting of the bilinear group and our construction is based on the DLIN assumption.

\begin{description}
\item[\boldmath$Setup(\lambda)$:] This algorithm generates the public and secret keys of our RCCA encryption scheme.
  \begin{enumerate}
  \item Pick bilinear group $(\G, \G_T)$ of prime order $p$ and the bilinear map $e$ on this pair of groups with generators $(f,g,h) \sample \G^3$ which verify $f^x = g^y = h$.
  \item Choose a random group generator $d \in \G$.
  \item Choose $g_1,g_2 \sample \G$ and set $\vec{g}_1 = (g_1,1,g) \in \G^3$, $\vec{g}_2 = (1,g_2,g) \in \G^3$ and $\vec{g}_3 \sample \G^3$.
  \item Set up the keys for the underlying encryption scheme $\pk_{enc} = (f,g,h)$ and $\sk_{enc} = (x,y)$.
  \item Choose four random group generators $(g_z, g_r, h_z, h_u)$ and twelve random exponents $\{\chi_i, \gamma_i, \delta_i\}_{i = 1}^3, \zeta, \rho, \phi \sample \Z_p$, then compute $(g_i,h_i) = (g_z^{\chi_i}g_r^{\gamma_i}, h_z^{\chi_i}h_u^{\delta_i})$ for $ i \in \{1,2,3\}$ and $(\alpha,\beta) = (g_z^\zeta g_r^\rho, h_z^\zeta h_u^\phi)$
  \item Set up the keys for the underlying signature scheme
    $$\vk_{sig} = (g_z, h_z, g_r, h_u, \{g_i, h_i\}_{i = 1}^3, \alpha ,\beta)$$
    and
    $$\sk_{sig} = (\vk_{sig}, \zeta, \rho, \phi, \{\chi_i, \gamma_i, \delta_i\}_{i = 1}^3).$$
  \item $\PK = (d,\pk_{enc}, \vk_{sig}, \sigma_{crs})$
  \end{enumerate}

\item[\boldmath$Enc(\PK,m)$:] This algorithm takes as input a message and the public key of the underlying encryption scheme, outputs the corresponded ciphertext of our RCCA encryption scheme.
  \begin{enumerate}
  \item Choose two random exponents $(\theta_1, \theta_2) \sample \Z_p$ and compute $ \vec{r} = (\theta_1, \theta_2)$.
  \item Compute the ciphertext $\vec{C} = (C_1, C_2, C_3)$:
    \begin{align*}
      C_1 &= f^{\theta_1} & C_2 &= g^{\theta_2} & C_3 &= m \cdot h^{\theta_1+\theta_2}
    \end{align*}

  \item Recall that we want prove the knowledge of the witness $\vec{w} = (m, \vec{r}, \vec{D}, \vec{S}, \vec{\sigma})$ which verifies that
    \begin{align*}
      Enc_{BBS}(pk_{enc}, m; \vec{r}) = \vec{C} \vee (\vec{C} = ReRand(\vec{D}; \vec{S}) \wedge Verify(vk_{sig}, \vec{D}) = \True)
    \end{align*}
    
  \item Define the bit $b = 1$ and a Groth-Sahai commitment $\vec{C}_b = (1,1,d^b)\cdot \vec{g}_1^{r_b} \cdot \vec{g}_2^{s_b} \cdot \vec{g}_3^{t_b}$ and also a $NIWI$ proof $\pi_b \in \G^6$ of the pairing product equation 
  
  \begin{align}
  e(d,\boxed{d^b}) &= e(\boxed{d^b},\boxed{d^b}) \tag{1}
  \end{align}
  
  which ensures that $b \in \{0,1\}$.

    %  \item Then generate commitements $\{\vec{C}_{\Gamma_i}\}_{i = 1}^3$ of the variables $\Gamma_i = C_i^b$ for $i \in \{1,2,3\}$ and the corresponded proofs:
    %    \begin{align}
    %      e(C_i,\boxed{h^b}) &= e(h, \boxed{\Gamma_i}) ,&
    %      \forall i \in \{1,2,3\}
    %    \end{align}
  \item We first prove the left side of the OR statement, we generate commitments $(\vec{C}_{R_1}, \vec{C}_{R_2})$ of the variables $(R_1 = d^{\theta_1b}, R_2 = d^{\theta_2b})$ and $\vec{C}_{M}$ commitment of $M = m^b$ and commitments $\{\vec{C}_{\Delta_i}\}_{i=1}^3$ of the variables $\{\Delta_i = C_i^b\}_{i=1}^3$. Recall that $\{C_i\}_{i=1}^3$, $\{R_1,R_2\}$ and $\{\Delta_i\}_{i=1}^3$ verify the following equations:
    \begin{align}
      e(C_i,\boxed{d^b}) &= e(\boxed{\Delta_i}, d)  &\forall i \in \{1,2,3\} \tag{2,3,4}\\
      e(\boxed{\Delta_1},d) &= e(f, \boxed{R_1}) \tag{5}\\
      e(\boxed{\Delta_2},d) &= e(g, \boxed{R_2}) \tag{6}\\
      e(\boxed{\Delta_3},d) \cdot e(\boxed{M},d^{-1}) &= e(\boxed{R_1}, h) \cdot e(\boxed{R_2},h) \tag{7}
    \end{align}



    %signature part
  \item Then we prove the right side of the OR statement:
    \begin{enumerate}  
    \item The $ReRand$ component: define $(D_1, D_2, D_3) = (1_\G, 1_\G, 1_\G)$ and $(S_1,S_2) = (1_\G, 1_\G)$.
    \item Remind that actually these variables are of the following forms in the security proof, but in the case $b=1$ they all become $1_\G$.
      $$(D_1, D_2, D_3) = (f^{(\theta_1+\theta_1')\cdot (1-b)},g^{(\theta_2+\theta_2')\cdot (1-b)},m^{1-b} \cdot h^{(\theta_1+\theta_1'+\theta_2+\theta_2')\cdot (1-b)})$$
      and $$(S_1, S_2) = (d^{\theta_1'\cdot (1-b)}, d^{\theta_2'\cdot (1-b)})$$
    \item Then compute the commitments $\{\vec{C}_{D_i}\}_{i=1}^3$ of $\{D_i\}_{i= 1}^3$ and $(\vec{C}_{S_1},\vec{C}_{S_2})$ commitments of $S_1,S_2$.
    \item And also compute the proofs of following equations:
      \begin{align}
        e(C_1/\boxed{\Delta_1}, d) &= e (\boxed{D_1},d) \cdot e(f^{-1}, \boxed{S_1}) \tag{8}\\
        e(C_2/\boxed{\Delta_2}, d) &= e (\boxed{D_2},d) \cdot e(g^{-1}, \boxed{S_2}) \tag{9}\\
        e(C_3/\boxed{\Delta_3}, d) &= e (\boxed{D_3},d) \cdot e(\boxed{S_1},h^{-1}) \cdot e(\boxed{S_2},h^{-1}) \tag{10}
      \end{align}

    \item Then the signature component (Remind that $b = 1$): define
      $$\vec{\sigma}  = (\Sigma_1, \Sigma_2, \Sigma_3) = (z^{1-b},r^{1-b},u^{1-b}) = (1_\G, 1_\G, 1_\G),$$
      then compute their commitments $(\vec{C}_{\Sigma_1}, \vec{C}_{\Sigma_2}, \vec{C}_{\Sigma_3})$.
    \item We generate the proof of the following linear pairing equations:
      \begin{align} 
        e(\alpha, d/\boxed{d^b}) &= e(g_z, \boxed{\Sigma_1}) \cdot e(g_r, \boxed{\Sigma_2}) \cdot \prod_{i=1}^3 e(g_i, \boxed{D_i}) \tag{11}\\
        e(\beta, d/\boxed{d^b}) &= e(h_z, \boxed{\Sigma_1}) \cdot e(h_u, \boxed{\Sigma_3}) \cdot \prod_{i=1}^3 e(h_i, \boxed{D_i}) \tag{12}
      \end{align}

    \end{enumerate}

  \item To allow the re-randomization of the ciphertext, we need to compute the commitments $\vec{C}_F$, $\vec{C}_G$ to the variables :

    \begin{align*}
    H &= h^b & F &= f^b, & G&=g^b
    \end{align*}

    and their corresponding proofs:
    \begin{align}
      e(\boxed{d^b}, h) &= e(\boxed{H},d) & e(\boxed{F},d) &= e(f,\boxed{d^b}) & e(\boxed{G}, d) &= e(g, \boxed{d^b})\tag{13, 14, 15}
    \end{align}

    
  \item We put all these proofs together to get $\vec{\pi}$.
  \item The ciphertext of the RCCA-scheme is
    $$(\vec{C} = (C_1, C_2, C_3), \vec{C}_{H}, \vec{C}_{d^b}, \vec{C}_{M}, \{\vec{C}_{R_1}\}_{i= 1}^2, \{\vec{C}_{D_i}\}_{i = 1}^3, \{\vec{C}_{S_i}\}_{i = 1}^2, \{\vec{C}_{\Sigma_i}\}_{i = 1}^3,\{\vec{C}_{\Delta_i}\}_{i=1}^3, \vec{C}_F, \vec{C}_G, \vec{\pi})$$
    
  \end{enumerate}

\item[\boldmath{$ReRandom(PK, C)$}:]For the randomization, we will proceed in two stages. Firstly we sample two random values $(\theta_1', \theta_2') \gets \mathbb{Z}_p$. The new variables are $C_1' = C_1 \cdot f^{\theta_1'}$, $C_2' = C_2 \cdot g^{\theta_2'}$ and $C_3' = C_3 \cdot h^{\theta_1'+\theta_2'}$. Then using the proof of the equations $(13, 14, 15)$, to adapt the new proofs corresponding to the new ciphertext instance $\vec{C}' = (C_1', C_2', C_3')$.

%  For the randomization we first generate randomness for each commitment. For a variable $X$, we generate the new randomness $(\tilde{r}_X, \tilde{s}_X, \tilde{t}_X)$, the new commitment will be $\tilde{\vec{C}}_X = \iota(X) \cdot \vec{g}_1^{r_X+\tilde{r}_X} \cdot \vec{g}_2^{s_X+\tilde{s}_X} \cdot \vec{g}_3^{t_X+\tilde{t}_X}$.

For the second stage, we randomize all the commitments and the GS proofs without changing the ciphertext part $\vec{C}' = (C_1', C_2', C_3')$.

In this algorithm, for the variable $X$, we denote its commitment by $\vec{C}_X = (1, 1, X) \cdot \vec{g}_1^{r_X} \cdot \vec{g}_2^{s_X} \cdot \vec{g}_3^{t_X}$ its new commitment from the first stage by $\vec{C}_X' = \vec{C}_X \cdot \vec{g}_1^{r'_X} \cdot \vec{g}_2^{s'_X} \cdot \vec{g}_3^{t'_X}$ and denote the new randomness introduced in the second step by $(\tilde{r}_X, \tilde{s}_X, \tilde{t}_X)$.


The detailed proof elements are in the appendix~\ref{Rerandomization}. With this instantiation, the ciphertext of the RCCA encryption scheme
 $$(\vec{C} = (C_1, C_2, C_3), \vec{C}_{H}, \vec{C}_{d^b}, \vec{C}_{M}, \{\vec{C}_{R_1}\}_{i= 1}^2, \{\vec{C}_{D_i}\}_{i = 1}^3, \{\vec{C}_{S_i}\}_{i = 1}^2, \{\vec{C}_{\Sigma_i}\}_{i = 1}^3,\{\vec{C}_{\Delta_i}\}_{i=1}^3, \vec{C}_F, \vec{C}_G, \vec{\pi})$$
has $93\G$ group elements.


\end{description}


\section{Efficiency}
From the efficiency point of view, there are in total 108 group elements.

For the efficientcy reason, we can replace
\begin{itemize}
\item $(\vec{\pi}_3,\vec{\pi}_9)$ by $(\vec{\pi}_3 \cdot \vec{\pi}_9)$
\item $(\vec{\pi}_4,\vec{\pi}_{10})$ by $(\vec{\pi}_4 \cdot\vec{\pi}_{10})$
\item $(\vec{\pi}_5,\vec{\pi}_{11})$ by $(\vec{\pi}_5 \cdot\vec{\pi}_{11})$
\item $(\vec{\pi}_6,\vec{\pi}_{12})$ by $(\vec{\pi}_6 \cdot\vec{\pi}_{12})$
\item $(\vec{\pi}_7,\vec{\pi}_{13})$ by $(\vec{\pi}_7 \cdot\vec{\pi}_{13})$
\end{itemize}

Thus we can reduce the number of group elements down to $108-5 \cdot 3 = 93$.

\begin{enumerate}
\item[$\tilde{\vec{\pi}}^{2,8}$]:
  \begin{enumerate}
  \item $\tilde{\pi}^{2,8}_1 = \pi^{2,8}_1 \cdot \pi_{14,1}^{\theta_1'} \cdot d^{\tilde{r}_{\Delta_1'}} \cdot C_1'^{-\tilde{r}_{d^b}} \cdot \pi_{14,1}^{-\theta_1'} \cdot d^{-\tilde{r}_{\Delta_1'}} \cdot d^{-\tilde{r}_{D_1}} \cdot f^{\tilde{r}_{S_1'}} = \pi^{2,8}_1 \cdot C_1'^{-\tilde{r}_{d^b}} \cdot d^{-\tilde{r}_{D_1}}$
  \item $\tilde{\pi}^{2,8}_2 = \pi^{2,8}_2 \cdot \pi_{14,2}^{\theta_1'} \cdot d^{\tilde{s}_{\Delta_1'}} \cdot C_1'^{-\tilde{s}_{d^b}} \cdot \pi_{14,2}^{-\theta_1'} \cdot d^{-\tilde{s}_{\Delta_1'}} \cdot d^{-\tilde{s}_{D_1}} \cdot f^{\tilde{s}_{S_1'}} = \pi^{2,8}_2 \cdot C_1'^{-\tilde{s}_{d^b}} \cdot d^{-\tilde{s}_{D_1}}$
  \item $\tilde{\pi}^{2,8}_3 = \pi^{2,8}_3 \cdot \pi_{14,3}^{\theta_1'} \cdot d^{\tilde{t}_{\Delta_1'}} \cdot C_1'^{-\tilde{t}_{d^b}} \cdot \pi_{14,3}^{-\theta_1'} \cdot d^{-\tilde{t}_{\Delta_1'}} \cdot d^{-\tilde{t}_{D_1}} \cdot f^{\tilde{t}_{S_1'}} = \pi^{2,8}_3 \cdot C_1'^{-\tilde{t}_{d^b}} \cdot d^{-\tilde{t}_{D_1}}$
  \end{enumerate}
\item[$\tilde{\vec{\pi}}^{3,9}$]:
  \begin{enumerate}
  \item $\tilde{\pi}^{3,9}_1 = \pi^{3,9}_1 \cdot \pi_{15,1}^{\theta_2'} \cdot d^{\tilde{r}_{\Delta_2'}} \cdot C_2'^{-\tilde{r}_{d^b}} \cdot \pi_{15,1}^{-\theta_2'} \cdot d^{-\tilde{r}_{\Delta_2'}} \cdot d^{-\tilde{r}_{D_2}} \cdot g^{-\tilde{r}_{S_2'}} = \pi^{3,9}_1 \cdot C_2^{-\tilde{r}_{d^b}} \cdot d^{-\tilde{r}_{D_2}} \cdot g^{-\tilde{r}_{S_2'}}$
  \item $\tilde{\pi}^{3,9}_2 = \pi^{3,9}_2 \cdot \pi_{15,2}^{\theta_2'} \cdot d^{\tilde{s}_{\Delta_2'}} \cdot C_2'^{-\tilde{s}_{d^b}} \cdot \pi_{15,2}^{-\theta_2'} \cdot d^{-\tilde{s}_{\Delta_2'}} \cdot d^{-\tilde{s}_{D_2}} \cdot g^{-\tilde{s}_{S_2'}} = \pi^{3,9}_2 \cdot C_2^{-\tilde{r}_{d^b}} \cdot d^{-\tilde{s}_{D_2}} \cdot g^{-\tilde{s}_{S_2'}}$
  \item $\tilde{\pi}^{3,9}_3 = \pi^{3,9}_3 \cdot \pi_{15,3}^{\theta_2'} \cdot d^{\tilde{t}_{\Delta_2'}} \cdot C_2'^{-\tilde{t}_{d^b}} \cdot \pi_{15,3}^{-\theta_2'} \cdot d^{-\tilde{t}_{\Delta_2'}} \cdot d^{-\tilde{t}_{D_2}} \cdot g^{-\tilde{t}_{S_2'}} = \pi^{3,9}_3 \cdot C_2^{-\tilde{r}_{d^b}} \cdot d^{-\tilde{t}_{D_2}} \cdot g^{-\tilde{t}_{S_2'}}$
  \end{enumerate}
\item[$\tilde{\vec{\pi}}^{4,10}$]:
  \begin{enumerate}
  \item $\tilde{\pi}^{4,10}_1 = \pi^{4,10}_1 \cdot \pi_{13,1}^{-(\theta_1'+\theta_2')} \cdot d^{\tilde{r}_{\Delta_3'}} \cdot C_3'^{-\tilde{r}_{d^b}} \cdot \pi_{13,1}^{\theta_1'+ \theta_2'} \cdot d^{-\tilde{r}_{\Delta_3'}} \cdot d^{-\tilde{r}_{D_3}} \cdot h^{\tilde{r}_{S_1'}} \cdot h^{\tilde{r}_{S_2}'} = \pi^{5,11}_1 \cdot C_3^{-\tilde{r}_{d^b}} \cdot d^{-\tilde{r}_{D_3}} \cdot h^{\tilde{r}_{S_1'}} \cdot h^{\tilde{r}_{S_2}'}$
  \item $\tilde{\pi}^{4,10}_2 = \pi^{4,10}_2 \cdot \pi_{13,2}^{-(\theta_1'+\theta_2')} \cdot d^{\tilde{s}_{\Delta_3'}} \cdot C_3'^{-\tilde{s}_{d^b}} \cdot \pi_{13,2}^{\theta_1'+ \theta_2'} \cdot d^{-\tilde{s}_{\Delta_3'}} \cdot d^{-\tilde{s}_{D_3}} \cdot h^{\tilde{s}_{S_1'}} \cdot h^{\tilde{s}_{S_2}'} = \pi^{5,11}_2 \cdot C_3^{-\tilde{s}_{d^b}} \cdot d^{-\tilde{s}_{D_3}} \cdot h^{\tilde{s}_{S_1'}} \cdot h^{\tilde{s}_{S_2}'}$
  \item $\tilde{\pi}^{4,10}_3 = \pi^{4,10}_3 \cdot \pi_{13,3}^{-(\theta_1'+\theta_2')} \cdot d^{\tilde{t}_{\Delta_3'}} \cdot C_3'^{-\tilde{t}_{d^b}} \cdot \pi_{13,3}^{\theta_1'+ \theta_2'} \cdot d^{-\tilde{t}_{\Delta_3'}} \cdot d^{-\tilde{t}_{D_3}} \cdot h^{\tilde{t}_{S_1'}} \cdot h^{\tilde{t}_{S_2}'} = \pi^{5,11}_3 \cdot C_3^{-\tilde{t}_{d^b}} \cdot d^{-\tilde{t}_{D_3}} \cdot h^{\tilde{t}_{S_1'}} \cdot h^{\tilde{t}_{S_2}'}$
  \end{enumerate}
\item[$\tilde{\vec{\pi}}^{5,11}$]:
  \begin{enumerate}
  \item $\tilde{\pi}^{5,11}_1 = \pi^{5,12}_1 \cdot \pi_{14,1}^{\theta_1'} \cdot d^{r_{\Delta_1'}} \cdot f^{-r_{R_1}'} \cdot \alpha^{-\tilde{r}_{d^b}} \cdot g_z^{-\tilde{r}_{\Sigma_1}} \cdot g_r^{-\tilde{r}_{\Sigma_2}}\cdot \prod_{i = 1}^3 g_i^{-\tilde{r}_{D_i}}$
  \item $\tilde{\pi}^{5,11}_2 = \pi^{5,12}_2 \cdot \pi_{14,2}^{\theta_1'} \cdot d^{s_{\Delta_1'}} \cdot f^{-s_{R_1}'} \cdot \alpha^{-\tilde{s}_{d^b}} \cdot g_z^{-\tilde{s}_{\Sigma_1}} \cdot g_r^{-\tilde{s}_{\Sigma_2}}\cdot \prod_{i = 1}^3 g_i^{-\tilde{s}_{D_i}}$
  \item $\tilde{\pi}^{5,11}_3 = \pi^{5,12}_3 \cdot \pi_{14,3}^{\theta_1'} \cdot d^{t_{\Delta_1'}} \cdot f^{-t_{R_1}'} \cdot \alpha^{-\tilde{t}_{d^b}} \cdot g_z^{-\tilde{t}_{\Sigma_1}} \cdot g_r^{-\tilde{t}_{\Sigma_2}}\cdot \prod_{i = 1}^3 g_i^{-\tilde{t}_{D_i}}$
  \end{enumerate}
\item[$\tilde{\vec{\pi}}^{6,12}$]:
  \begin{enumerate}
  \item $\tilde{\pi}^{6,12}_1 = \pi^{7,13}_1 \cdot \pi_{15,1}^{\theta_2'} \cdot d^{r_{\Delta_2'}} \cdot f^{-r_{R_2'}} \cdot \beta^{-\tilde{r}_{d^b}} \cdot h_z^{-\tilde{r}_{\Sigma_1}} \cdot h_u^{-\tilde{r}_{\Sigma_3}}\cdot \prod_{i = 1}^3 h_i^{-\tilde{r}_{D_i}}$
  \item $\tilde{\pi}^{6,12}_2 = \pi^{7,13}_2 \cdot \pi_{15,2}^{\theta_2'} \cdot d^{s_{\Delta_2'}} \cdot f^{-s_{R_2'}} \cdot \beta^{-\tilde{s}_{d^b}} \cdot h_z^{-\tilde{s}_{\Sigma_1}} \cdot h_u^{-\tilde{s}_{\Sigma_3}}\cdot \prod_{i = 1}^3 h_i^{-\tilde{s}_{D_i}}$
  \item $\tilde{\pi}^{6,12}_3 = \pi^{7,13}_3 \cdot \pi_{15,3}^{\theta_2'} \cdot d^{t_{\Delta_2'}} \cdot f^{-t_{R_2'}} \cdot \beta^{-\tilde{t}_{d^b}} \cdot h_z^{-\tilde{t}_{\Sigma_1}} \cdot h_u^{-\tilde{t}_{\Sigma_3}}\cdot \prod_{i = 1}^3 h_i^{-\tilde{t}_{D_i}}$
  \end{enumerate}
\end{enumerate}

%\input{proofs}
