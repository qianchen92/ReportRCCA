\subsection*{General Context}

For the simple needs of safe and private communication, cryptography is very important in our current life. 
Starting with Shanon in 1949 in his paper Communication theory of secrecy systems~\cite{shannon-otp},
we begin to formally define the properties we wanted for the cryptographic protocols and prove these properties based on some hardness assumptions or complexity assumptions.
As one of the most useful cryptographic protocol, the encryption scheme is widely used in the construction of more complex cryptographic systems.
Thus we are motivated to define the most adapted security notion for different encryption schemes. 
From the very basic One-Wayness Chosen-Plaintext Attack(OW-CPA) model which means with the ciphertext no one can get the original message
to the most secure Indistinguishable Chosen-Ciphertext Attack(IND-CCA) model
in which no one can distinguish with non-negligible probability ciphertext of two different messages even with access to the decryption oracle(except the trivial case in which the adversary require the oracle to decrypt the ciphertext which is required to distinguish). 

One of the most important property of the encryption scheme is the malleability, 
which means with a valid ciphertext $c$ corresponding to the message $m$ we can produce another ciphertext $c'$ of a plaintext $T(m)$ which is related to the original one with the function $T$ without knowing $m$.
However, there exits a trivial IND-CCA type attack for a such encryption scheme: given the ciphertext $C$ of one of the messages $m_0, m_1$, we can choose a function $T$ such that $T(m_0) \neq T(m_1)$,
then modify the ciphertext and ask the decryption oracle to decrypt the new ciphertext $C$ corresponding to the plaintext $T(m_b)$ where $b \in \{0,1\}$ is chosen by the challenger unknown to us,
then as $T(m_0) \neq T(m_1)$ we can easily distinguish which message is corresponded to the given cipherttext.

\subsection*{Problem studied}
My internship focused on encryption schemes which are only allowed to be rerandomizable.
This property can be formally defined as resistant to Replayable Chosen-Ciphertext Attack(RCCA).
A such scheme has many applications, as in the voting system,
each voter encrypts his message using a rerandomizable encryption scheme and submits it to the server(who collects all the votes).
The adversary may want to force the voter to provide a proof of the message of its own vote.
But if before publishing the votes, the server re-randomizes all the votes,
then even the voter can not prove the message with its randomness.
Thus the adversary can not any more achieve its goal.

However, the previous constructions~\cite{DBLP:conf/ctrsa/GolleJJS04} suffered from the Chosen-Chipertext attack or based on some non standard assumptions~\cite{DBLP:conf/crypto/PrabhakaranR07} or not very efficient~\cite{DBLP:conf/eurocrypt/ChaseKLM12}.
The aim of my internship is to construct and prove with standard and well studied assumptions efficient encryption scheme which verifies the RCCA security.
We try to improve the efficiency to get some usable protocol in the practical point of view.

\subsection*{Proposed Contributions}
In our constructions, all the ciphertexts are elements in two groups $\G$ and $\hat{\G}$.
The size of one element in $\hat{\G}$ is approximately two times the size of one element in $\G$.
The contributions of my internship are the following:
We first give an efficient instantiation of the general controlled-malleable encryption scheme proposed by~\cite{DBLP:conf/eurocrypt/ChaseKLM12} which ciphertext has $93\G$ elements(or $49\G \times 20\hat{\G}$ elements).
Then we use another approach to get an efficient computational RCCA encryption in which the the ciphertext size is only $39\G + 20 \hat{\G}$ elements.
As a sub-result, we also have constructed a public verifiable structure-preserving CCA encryption which means all the public key and ciphertext are group elements and during the encryption and decryption procedures we don't need the hash function.
Such an encryption scheme is very useful, because hash function is not adapted for the proof system.
Our scheme is more efficient than the existing construction $16\G + 11\hat{\G}$ against $321\G$~\cite{DBLP:conf/pkc/AbeDKNO13}.

\subsection*{Arguments Supporting Their Validity}
For the validity of our constructions,
every construction has been proven with standard complexity assumptions which are well studied.
And we also give their efficiency by measuring their ciphertext size and comparing with existing schemes to show that we achieve efficiency improvements.

\subsection*{Summary and Future Work}
During my internship, I have proposed several efficiency improvements for the construction of the cryptographic scheme.
This contribution can be considered as improvements both in the efficiency and the constructions of the new schemes with some practical properties for the further constructions of more complex cryptographic systems.

However, several questions are left open. 
We especially studied the re-randomizable encryption scheme, which is a subset of homomorphic encryption scheme.
Can we use the similar ideas for a wider class of homomorphic encryption schemes?
And, even in our RCCA scheme, the re-randomization is computational.
A natural open question is can we have RCCA scheme which rerandomization which is statistical unlinkable.

