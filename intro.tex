\subsection*{General Context}

For the simple needs of communicate safely and privately, cryptography is very important in our current life. 
Start with Shanon in 1949 in his paper Communication theory of secrecy systems~\cite{shannon-otp},
we begin to formally define the properties we wanted for the cryptographic protocols and prove these properties based on some hardness assumptions or complexity assumptions.
As one of most useful cryptographic protocol, the encryption scheme is widely used in the construction or more complex cryptographic system.
Thus we are motivated to define the most adapted security notion for the encryption scheme. 
From the very basic One-Wayness Chosen-Plaintext Attack(OW-CPA) to the most secure Indistinguishable Chosen-Ciphertext Attack(IND-CCA) model. 

One of the most important property of the encryption scheme is the malleability, 
which means with a valid ciphertext we can produce another ciphertext of a plaintext which is related to the original one without knowing it.
This property necessarily produce some information leakages,
thus it is forbidden by the most secure definition(IND-CCA).
But recently, these properties are seen to be a potentially useful feature that can be exploited.

\subsection*{Problem studied}
My internship focused on the encryption scheme which only allowed to be re-randomizable.
This property can be formally defined as resist of Replayeable Chosen-Ciphertext Attack(RCCA).
In some recent work of constructing more advanced cryptographic protocols like receipt-free voting system~\cite{cryptoeprint:2015:629} and reverse firewall~\cite{DBLP:conf/crypto/DodisMS16},
RCCA encryption scheme are required in such schemes,
The first application requires a weak re-randomization RCCA encryption scheme,
and the second application requires a strong re-randomization RCCA encryption scheme.
The aim of my internship is to construct and prove efficient encryption scheme which are suitable for the above schemes.
The main motivation is that the previous works on constructing such scheme are more or less not efficient.
We try to improve there efficiency to get some usable protocol in the practical point of view.

\subsection*{Proposed Contributions}
The contributions of my internship are the following: 
the construction and proof of a efficient weak RCCA encryption which is adapted to the receipt-free voting system,
efficient instantiation of the general controlled-malleable encryption scheme proposed by~\cite{DBLP:conf/eurocrypt/ChaseKLM12}.
Then we use another approach to get a very efficient strong RCCA encryption in which the the ciphertext size is only .
As a sub-result, we also have constructed a public verifiable structure-preserving CCA encryption which is more efficient than the existing construction.

\subsection*{Arguments Supporting Their Validity}
For the validity of our construction,
every construction has been proven for the security model with standard complexity assumptions which are well studied and general believed.
And we also give their efficiency by counting their ciphertext size and compare with existing schemes to show that we achieve efficiency improvement.

\subsection*{Summary and Future Work}
During my internship, I have proposed several efficiency improvements for the construction of the cryptographic scheme,
This contribution can be considered as improvement both in the efficiency and the construction of the new scheme with some practical properties for the further construction of more complex cryptographic system.

However, several questions are left open. 
We especially studied the re-randomizable encryption scheme, which is a subset of homomorphic encryption scheme,
can we use the similar idea of the efficiency improvement for a wider class of homomorphic encryption scheme.
And, even in our strong RCCA scheme, the re-randomization is computational.
A natural open question is can we achieve strong RCCA scheme which re-randomization is statistical.

