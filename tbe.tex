In this section, we first give an instantiation of weak CCA tag-based encryption based on SXDH assumption,
then combined with a linearly homomorphic signature scheme,
we can construct a partially randomizable RCCA encryption scheme.

\begin{subsection}{Tag based-encryption}

	In this section, we first describe a tag based-encryption which is selective-ID CCA1 tag-based encryption scheme. The construction of this scheme is proposed by Cash \etal~\cite{DBLP:journals/joc/CashKS09} in the form of an public key encryption scheme.

	\begin{description}

	\item[$\boldmath{\Setup(1^\lambda)}$] :
	\begin{enumerate}
		\item Choose a group $\G$ of prime order $p$ in which the DDH problem is hard and choose randomly a group generator $g \sample \G$.
		\item Generate three random values $(x_1, \tilde{x}_1, x_2) \sample \mathbb{Z}_p^3$.
		\item Compute $X_1
		 \gets g^{x_1}$, $\tilde{X}_1 \gets g^{\tilde{x}_1}$.
		\item Set $\PPP \gets (g, \G)$
		\item Compute $\PK \gets (X_1, \tilde{X}_1, X_2)$ and $\SK \gets (x_1, \tilde{x}_1, x_2)$
	\end{enumerate}
	
	\item[$\boldmath{\Enc(\PPP, \PK, m, \tau)}$]: where $m \in \G$ and $\tau \in \mathbb{Z}_p$
	\begin{enumerate}
	\item Generate a random value $r \sample \mathbb{Z}_p$.
	\item Compute 
	\begin{align}
	R &= g^r, & Z &= (X_1^\tau \tilde{X}_1)^r & C &= X_2^r \cdot m
	\end{align}
	\item Output the cipher-text $\vec{C} = (R, Z, C) \in \G^3$.

	\end{enumerate}

	\item[$\boldmath{\Dec(\PPP, \SK, \vec{C}, \tau)}$]: where $\vec{C} \in \G^3$
	\begin{enumerate}
	\item Parse $\vec{C}$ as $(R, Z, C)$ and $\SK$ as $(x_1, \tilde{x}_1, x_2)$.
	\item Verify that  if $Y^{x_1 \tau + \tilde{x}_1} = Z$.
	\item If not reject, otherwise, compute:
	\begin{align}
	m &= C/ R^{x_2}
	\end{align}
	\item Output the message $m$.
	\end{enumerate}
	
\end{description}

\end{subsection}

\begin{subsection}{Combine the tag-based encryption with the Linearly Homomorphic Structure-Preserving Signature(LHSPS)}

	In this section, we follow the general idea of construction CCA2 security encryption from a selective-tag based CCA1 encryption scheme.
	To achieve the partially randomization property, we use the LHSPS in the construction of our public key encryption scheme.
	
	We present our construction for the partially randomizable encryption scheme as following:
	
	\begin{description}
	\item[$\boldmath{RCCA1.\Setup(1^\lambda)}$]:
		\begin{enumerate}
		\item Generate the public parameters for the LHSPS scheme $\PPP_{LHSPS} \gets LHSPS.\Setup(1^\lambda, \ell = 4)$. Where $\ell$ presents the length of the message will be signed.
		\item Generate the public parameters for the tag-based encryption scheme $\PPP_{TBE} \gets TBE.\Setup(1^\lambda)$.
		\item Generate a collision resistant hash function $H : \hat{\G}^4 \to \mathbb{Z}_p$. 
		\item Output the public parameter $\PPP \gets (\PPP_{LHSPS}, \PPP_{TBE})$
		\end{enumerate}
		
	\item[$\boldmath{RCCA1.\KeyGen(\PPP)}$]:
		\begin{enumerate}
		\item Parse $\PPP$ as $(\PPP_{LHSPS}, \PPP_{TBE})$.
		\item Using $\PPP_{TBE}$ to generate the public keys and secret keys $(\PK, \SK) \gets TBE.\KeyGen(\PPP_{TBE})$ of the underlying tag-based encryption scheme.
		\end{enumerate}
		
	\item[$\boldmath{RCCA1.\KeyGen(\PPP)}$]:
		\begin{enumerate}
		\item Generate the signing key-verification key pair for the underlying signature scheme 
		$$(\SSK, \SVK) \gets LHSPS. \KeyGen(\PPP_{LHSPS}) \in (\mathbb{Z}_p^8 \times \G^4).$$
		\item Hash the verification key of the underlying signature scheme $\tau \gets H(\SVK)$.
		\item Generate the encryption of the message $m$ of the underlying tag-based encryption scheme $\vec{C} \gets TBE.\Enc()$
		\end{enumerate}
		
	\end{description}
\end{subsection}