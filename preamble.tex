        \usepackage[backend=bibtex]{biblatex}
\bibliography{main}
\usepackage{fullpage}
%\usepackage[top=1.5cm, bottom=1.5cm, left=1.5cm, right=1.5cm]{geometry}
\usepackage[english]{babel}
\usepackage[utf8]{inputenc}
%\usepackage[latin1]{inputenc} 
\usepackage[T1]{fontenc}
\usepackage{lmodern}
\usepackage[onelanguage,boxed]{algorithm2e}
\usepackage{graphicx}
\usepackage{float}
\usepackage{amsmath}
\usepackage{amsfonts}
\usepackage{amsthm}
\usepackage{color}
\usepackage[usenames,dvipsnames]{xcolor}
\usepackage[toc,page]{appendix}
\usepackage{listings}
\usepackage{tikz}
\usepackage{subfigure}
\usepackage{wrapfig}
\usepackage{hyperref}
\hypersetup{colorlinks,linkcolor=black,urlcolor=blue}
\usepackage[font=small,labelfont=bf]{caption}
\newtheorem {myDef} {Definition}
\newtheorem {myTh}{Theorem}
\lstset
{
  language=[Objective]Caml,
  basicstyle=\footnotesize,       % the size of the fonts that are used for the code
  numbers=left,                   % where to put the line-numbers
  numberstyle=\footnotesize,      % the size of the fonts that are used for the line-numbers
  stepnumber=1,                   % the step between two line-numbers. If it is 1 each line will be numbered
  numbersep=5pt,                  % how far the line-numbers are from the code
  backgroundcolor=\color{white},  % choose the background color. You must add \usepackage{color}
  showspaces=false,               % show spaces adding particular underscores
  showstringspaces=false,         % underline spaces within strings
  showtabs=false,                 % show tabs within strings adding particular underscores
  frame=leftline,           % adds a frame around the code
  tabsize=2,          % sets default tabsize to 2 spaces
  captionpos=b,           % sets the caption-position to bottom
  breaklines=true,        % sets automatic line breaking
  breakatwhitespace=false,    % sets if automatic breaks should only happen at whitespace
  escapeinside={\%*}{*)},          % if you want to add a comment within your code
  basicstyle=\ttfamily,
  keywordstyle=\color{BurntOrange}\ttfamily,
  stringstyle=\color{red}\ttfamily,
  commentstyle=\color{LimeGreen}\ttfamily,
  morecomment=[l][\color{RoyalBlue}]{\#}
}
\lstset
{
  emph={Stack},
  emphstyle={\color{DarkOrchid}}
}



\newcommand{\A}{\mathcal{A}}
\newcommand{\B}{\mathcal{B}}
\newcommand{\D}{\mathcal{D}}
\newcommand{\G}{\mathbb{G}}
\newcommand{\Z}{\mathbb{Z}}
\newcommand{\PR}{\operatorname{Pr}}
\newcommand{\PP}{\mathsf{P}}  
\newcommand{\VV}{\mathsf{V}}  
\newcommand{\K}{\mathsf{K}}  
\newcommand{\SIM}{\mathsf{S}}  
\newcommand{\lbl}{\mathsf{lbl}} 
\newcommand{\PPE}{\mathrm{PPE}} 
\newcommand{\SK}{\mathsf{SK}}
\newcommand{\PK}{\mathsf{PK}}
\newcommand{\VK}{\mathsf{VK}}
\newcommand{\SSK}{\mathsf{SSK}}
\newcommand{\SVK}{\mathsf{SVK}}
\newcommand{\sk}{\mathsf{sk}}
\newcommand{\ck}{\mathsf{ck}}
\newcommand{\tk}{\mathsf{tk}}
\newcommand{\msk}{\mathsf{msk}}
\newcommand{\vk}{\mathsf{vk}}
\newcommand{\ovk}{\mathsf{ovk}}
\newcommand{\pk}{\mathsf{pk}}
\newcommand{\opk}{\mathsf{opk}}
\newcommand{\osk}{\mathsf{osk}}
\newcommand{\com}{\mathsf{com}}
\newcommand{\open}{\mathsf{open}}
\newcommand{\True}{\mathsf{True}}
\newcommand{\False}{\mathsf{False}}
\newcommand{\BF}{\mathbf}
\newcommand{\sample}{\stackrel{{\scriptscriptstyle \mkern4mu R}}{\gets}}
\newcommand{\etal}{\textit{el. al.}}
\newcommand{\eg}{\textrm{e.g.} }
\newcommand{\ie}{\textrm{i.e.} }
\newcommand{\wrt}{\textrm{w.r.t.} }
\newcommand{\st}{\textrm{s.t.} }
\newcommand{\resp}{\textrm{resp.} }
\providecommand{\tprod}{{\textstyle\prod}}
\newcommand{\Setup}{{\mathsf{Setup}}}
\newcommand{\KeyGen}{{\mathsf{KeyGen}}}
\newcommand{\Enc}{{\mathsf{Enc}}}
\newcommand{\Dec}{{\mathsf{Dec}}}
\newcommand{\Rerand}{{\mathsf{ReRandom}}}
\newcommand{\Sig}{{\mathsf{Sign}}}
\newcommand{\Verif}{{\mathsf{Verify}}}
\newcommand{\Prove}{{\mathsf{Prove}}}
\newcommand{\Com}{{\mathsf{Commit}}}
\newcommand{\PPP}{\mathsf{PP}}
\newcommand{\Forge}{\mathsf{Forge}}
\newcommand{\Adv}{\mathcal{A}}
